% copyright 2020 Edmundo Carmona Antoranz
% Released under the terms of Creative Commons Attribution-ShareAlike 4.0 International Public License

\section{API conflicts}

Before explaining the details you have to care for when dealing with {\bf API conflicts}, let's consider what {\bf API changes} mean for
merges {\it without} conflicts.

\subsection{Example 15}
\label{example_15}

Let me show you the common ancestor:
\begin{lstlisting}[style=python_style,
	basicstyle=\small,
	caption={\bf example 15} - common ancestor]
#!/usr/bin/python

import sys

colors = {"black": "black mirror",
          "white": "white noise",
          "blue": "blue sky"}

def getPhrase(color):
    color = color.lower()
    if color not in colors:
        sys.stderr.write("There is no phrase defined for color %s\n" % color)
        sys.exit(1)
    phrase = colors[color]
    phrase = "%s: %s" % (color, phrase)
    return phrase

for color in sys.argv[1:]:
    print(getPhrase(color))
\end{lstlisting}

We will be printing the phrases for multiple colors now, not just a single one.

On {\bf branchA} the API is changed so that it can be specified whether the {\bf original color} will be included in the phrase:
\begin{lstlisting}[style=python_style,
	basicstyle=\small,
	caption={\bf example 15} - {\bf branchA}]
#!/usr/bin/python

import sys

colors = {"black": "black mirror",
          "white": "white noise",
          "blue": "blue sky"}

def getPhrase(color, includeColor):
    color = color.lower()
    if color not in colors:
        sys.stderr.write("There is no phrase defined for color %s\n" % color)
        sys.exit(1)
    phrase = colors[color]
    if (includeColor):
        phrase = "%s: %s" % (color, phrase)
    return phrase

for color in sys.argv[1:]:
    print(getPhrase(color, True))
\end{lstlisting}

In the real world there is probably no point in having this API change, but it's ok. Bear with it for the sake of explanation. Let's
take a look at {\bf branchB}.

\begin{lstlisting}[style=python_style,
	basicstyle=\small,
	caption={\bf example 15} - {\bf branchB}]
#!/usr/bin/python

import sys

colors = {"black": "black mirror",
          "white": "white noise",
          "blue": "blue sky"}

def getPhrase(color):
    color = color.lower()
    if color not in colors:
        sys.stderr.write("There is no phrase defined for color %s\n" % color)
        sys.exit(1)
    phrase = colors[color]
    phrase = "%s: %s" % (color, phrase)
    return phrase

# I love color white
print(getPhrase("white"))
for color in sys.argv[1:]:
    print(getPhrase(color))
\end{lstlisting}

Now the phrase for color white will always be printed before all the colors that are used as parameters. What will happen if we
tried to merge? Will we get a conflict or won't we? What do you think?

\begin{lstlisting}[style=console_style,
	basicstyle=\small,
	caption={\bf example 15} - {\bf git merge}]
$ git merge example15/branchB --no-edit
Auto-merging example.py
Merge made by the 'recursive' strategy.
 example.py | 2 ++
 1 file changed, 2 insertions(+)
\end{lstlisting}

Nice! And the end result?

\begin{lstlisting}[style=python_style,
	basicstyle=\small,
	caption={\bf example 15} - final result]
#!/usr/bin/python

import sys

colors = {"black": "black mirror",
          "white": "white noise",
          "blue": "blue sky"}

def getPhrase(color, includeColor):
    color = color.lower()
    if color not in colors:
        sys.stderr.write("There is no phrase defined for color %s\n" % color)
        sys.exit(1)
    phrase = colors[color]
    if (includeColor):
        phrase = "%s: %s" % (color, phrase)
    return phrase

# I love color white
print(getPhrase("white"))
for color in sys.argv[1:]:
    print(getPhrase(color, True))
\end{lstlisting}

You {\bf do} see the problem, right? Let me show you if you haven't caught it yet:

\begin{lstlisting}[style=console_style,
	basicstyle=\small,
	caption={\bf example 15} - we have a problem]
$ ./example.py blue
Traceback (most recent call last):
  File "./example.py", line 20, in <module>
    print(getPhrase("white"))
TypeError: getPhrase() takes exactly 2 arguments (1 given)
\end{lstlisting}

Oh, this line, added on {\bf branchB}, hasn't been adapted to the new {\bf API change} coming from {\bf branchA}:

\begin{lstlisting}[style=python_style,
	basicstyle=\small,
	firstnumber=19,
	caption={\bf example 15} - this is the problem]
# I love color white
print(getPhrase("white"))
\end{lstlisting}

And that is because, when the API was changed on {\bf branchA}, the {\it preexisting calls on that branch} were adjusted.... but the
calls that were {\it added on the other branch} won't magically get those adjustments when merging. Even if there are no conflicts,
{\bf API changes can be tricky to deal with} when merging code.

\subsection{Example 16}
\label{example_16}

Now, what will happen when you have {\bf conflicting changes on the API}? It's the same situation we just saw, but on steroids
because you will have to consider code that is modified from all the branches that are being merged, not just one.

Let's use another example {\it based} on the example we just used. Here's the full conflicted file:

\begin{lstlisting}[style=python_style,
	basicstyle=\small,
	caption={\bf example 16} - full conflicted file]
#!/usr/bin/python

import sys

RESET=chr(0x1b) + "[m"
colors = {"black": {"phrase": "black mirror", "fg": chr(0x1b) + "[0;7m"}, # reverse on color
          "white": {"phrase": "white noise", "fg": chr(0x1b) + "[1m"},
          "blue": {"phrase": "blue sky", "fg": chr(0x1b) + "[1;34m"}}

<<<<<<< HEAD
def getPhrase(color, showColor):
||||||| 46e6753
def getPhrase(color):
=======
def getPhrase(color, useColor):
>>>>>>> example16/branchB
    """
    Get the phrase that corresponds to one color
    
    Parameters
    ----------
    color: color to get the phrase for. It can be used in any combination of uppercase/lowercase letters
<<<<<<< HEAD
    showColor: whether to display the original color name before the phrase or not
||||||| 46e6753
=======
    useColor: we can change the color of the phrase when writing on the output
>>>>>>> example16/branchB
    """
    color = color.lower()
    if color not in colors:
        sys.stderr.write("There is no phrase defined for color %s\n" % color)
        sys.exit(1)
<<<<<<< HEAD
    phrase = colors[color]
    if showColor:
        phrase = "%s: %s" % (color, phrase)
||||||| 46e6753
    phrase = colors[color]
    phrase = "%s: %s" % (color, phrase)
=======
    phrase = colors[color]["phrase"]
    phrase = "%s: %s" % (color, phrase)
    if useColor:
        phrase = colors[color]["fg"] + phrase + RESET
>>>>>>> example16/branchB
    return phrase

for color in sys.argv[1:]:
    print(getPhrase(color, True))
\end{lstlisting}

Each branch added a parameter to {\bf getPhrase()}... and the name can be confusing, right? Are they the same parameter (same {\it intent})
with differing names on the two branches? That's yet {\bf another} layer of difficulty that you might have to deal with. Luckily for us,
there's {\bf docstrings} for the function and on the second {\bf CB} we can see what the parameter from each branch is about and we can
conclude that they are definitely not the same:

\begin{lstlisting}[style=python_style,
	firstnumber=17,
	basicstyle=\small,
	caption={\bf example 16} - {\bf CB\#2}]
    """
    Get the phrase that corresponds to one color
    
    Parameters
    ----------
    color: color to get the phrase for. It can be used in any combination of uppercase/lowercase letters
<<<<<<< HEAD
    showColor: whether to display the original color name before the phrase or not
||||||| 46e6753
=======
    useColor: we can change the color of the phrase when writing on the output
>>>>>>> example16/branchB
    """
\end{lstlisting}

The one in {\bf UB}, {\bf showColor}, is used to tell if the original color will be printed before the phrase. The one in {\bf LB},
{\bf useColor}, is used to change text color on the output for an added visual effect, and as usual, we need to keep both {\bf intents}.
On the first {\bf CB}, what parameter will we add first? From my point of view, either one is fine but you might have rules to follow
for that... every situation is different. I will do this to solve the first two conflicts in a single shot:

\begin{lstlisting}[style=python_style,
	firstnumber=10,
	basicstyle=\small,
	caption={\bf example 16} - solution to first 2 {\bf CB}s]
def getPhrase(color, showColor, colorOnOutput):
    """
    Get the phrase that corresponds to one color
    
    Parameters
    ----------
    color: color to get the phrase for. It can be used in any combination of uppercase/lowercase letters
    showColor: whether to display the original color name before the phrase or not
    colorOnOutput: we can change the color of the phrase when writing on the output
    """
\end{lstlisting}

I changed the name of the parameter coming from {\bf the other branch} so that the difference between the parameters is more
evident. That means that I will {\it also} have to modify the places where {\bf useColor} (the original name of that parameter) is in
the function. Let's take a look at the next {\bf CB} starting on line 24\footnote{Line 24 {\bf after} the previous two conflicts are solved}:

\begin{lstlisting}[style=python_style,
	firstnumber=24,
	basicstyle=\small,
	caption={\bf example 16} - {\bf CB\#3}]
<<<<<<< HEAD
    phrase = colors[color]
    if showColor:
        phrase = "%s: %s" % (color, phrase)
||||||| 46e6753
    phrase = colors[color]
    phrase = "%s: %s" % (color, phrase)
=======
    phrase = colors[color]["phrase"]
    phrase = "%s: %s" % (color, phrase)
    if useColor:
        phrase = colors[color]["fg"] + phrase + RESET
>>>>>>> example16/branchB
\end{lstlisting}

\subsubsection{dMU}
Added a conditional in line 26 to control if we will include the original color name on the phrase.

\subsubsection{dML}
Modified the way we get the phrase associated to the color because the dictionary structure has changed (line 32) and added a
conditional on line 34 to modify the phrase so that the output color on the terminal is changed.

\subsubsection{Resolution}
Given that {\bf LB} is more different from {\bf MB} than {\bf UB}, I will work from {\bf LB}:

\begin{lstlisting}[style=python_style,
	firstnumber=31,
	basicstyle=\small,
	caption={\bf example 16} - Step 1 - {\bf LB\#3}]
=======
    phrase = colors[color]["phrase"]
    phrase = "%s: %s" % (color, phrase)
    if useColor:
        phrase = colors[color]["fg"] + phrase + RESET
>>>>>>> example16/branchB
\end{lstlisting}

Then that is needed to be done is add the conditional to include the original color and adjust indentation:

\begin{lstlisting}[style=python_style,
	firstnumber=31,
	basicstyle=\small,
	caption={\bf example 16} - Step 2 - {\bf LB\#3}]
=======
    phrase = colors[color]["phrase"]
    if showColor:
        phrase = "%s: %s" % (color, phrase)
    if useColor:
        phrase = colors[color]["fg"] + phrase + RESET
>>>>>>> example16/branchB
\end{lstlisting}

And then we are done, right? {\bf Gotcha!!!} Did you remember to adjust the name of the parameter that is coming from
{\bf the other branch}? I changed it from {\bf useColor} to {\bf colorOnOutput} while working on solving {\bf CB}s 1 and 2, so we need to also adjust that on this {\bf CB}:

\begin{lstlisting}[style=python_style,
	firstnumber=31,
	basicstyle=\small,
	caption={\bf example 16} - Step 2 - {\bf LB\#3}]
=======
    phrase = colors[color]["phrase"]
    if showColor:
        phrase = "%s: %s" % (color, phrase)
    if colorOnOutput:
        phrase = colors[color]["fg"] + phrase + RESET
>>>>>>> example16/branchB
\end{lstlisting}

So, the final resolution of the file would be:

\begin{lstlisting}[style=python_style,
	basicstyle=\small,
	caption={\bf example 16} - solved conflict]
#!/usr/bin/python

import sys

RESET=chr(0x1b) + "[m"
colors = {"black": {"phrase": "black mirror", "fg": chr(0x1b) + "[0;7m"}, # reverse on color
          "white": {"phrase": "white noise", "fg": chr(0x1b) + "[1m"},
          "blue": {"phrase": "blue sky", "fg": chr(0x1b) + "[1;34m"}}

def getPhrase(color, showColor, colorOnOutput):
    """
    Get the phrase that corresponds to one color
    
    Parameters
    ----------
    color: color to get the phrase for. It can be used in any combination of uppercase/lowercase letters
    showColor: whether to display the original color name before the phrase or not
    colorOnOutput: we can change the color of the phrase when writing on the output
    """
    color = color.lower()
    if color not in colors:
        sys.stderr.write("There is no phrase defined for color %s\n" % color)
        sys.exit(1)
    phrase = colors[color]["phrase"]
    if showColor:
        phrase = "%s: %s" % (color, phrase)
    if colorOnOutput:
        phrase = colors[color]["fg"] + phrase + RESET
    return phrase

for color in sys.argv[1:]:
    print(getPhrase(color, True))
\end{lstlisting}

And now we are ready, right? {\bf Gotcha again!!!} Did you happen to notice the call on line 32?

\begin{lstlisting}[style=python_style,
	firstnumber=31,
	basicstyle=\small,
	caption={\bf example 16} - call that needs to be adusted]
for color in sys.argv[1:]:
    print(getPhrase(color, True))
\end{lstlisting}

It has only 2 parameters... but we should have 3, right? Why is that? Oh, it's one of those cases where git's {\it smartness} is playing
against us. If git notices that both branches applied the same change, it just takes it in {\bf as is} from one of the branches and
generates no conflict. In this case both branches added the second parameter and the value is {\bf True} for both branches, so it's
the same change coming from both branches. But we need to adjust it so that both parameters get a {\bf True} value and then it will
be solved. All in all, the very final resolution of the conflict\footnote{For real this time, I swear!} would be:

\begin{lstlisting}[style=python_style,
	basicstyle=\small,
	caption={\bf example 16} - solved conflict {\bf for real}]
#!/usr/bin/python

import sys

RESET=chr(0x1b) + "[m"
colors = {"black": {"phrase": "black mirror", "fg": chr(0x1b) + "[0;7m"}, # reverse on color
          "white": {"phrase": "white noise", "fg": chr(0x1b) + "[1m"},
          "blue": {"phrase": "blue sky", "fg": chr(0x1b) + "[1;34m"}}

def getPhrase(color, showColor, colorOnOutput):
    """
    Get the phrase that corresponds to one color
    
    Parameters
    ----------
    color: color to get the phrase for. It can be used in any combination of uppercase/lowercase letters
    showColor: whether to display the original color name before the phrase or not
    colorOnOutput: we can change the color of the phrase when writing on the output
    """
    color = color.lower()
    if color not in colors:
        sys.stderr.write("There is no phrase defined for color %s\n" % color)
        sys.exit(1)
    phrase = colors[color]["phrase"]
    if showColor:
        phrase = "%s: %s" % (color, phrase)
    if colorOnOutput:
        phrase = colors[color]["fg"] + phrase + RESET
    return phrase

for color in sys.argv[1:]:
    print(getPhrase(color, True, True))
\end{lstlisting}

That was some work, right?

\subsection{Exercises}

\subsubsection{Exercise 7}

From \hyperref[git_repo]{git repo}, checkout revision {\bf 4284497396} and merge {\bf cdb5330a9b}\footnote{These are the
parents of revision {\bf dcd6a8c09a}}. Solve all conflicts. Solution is \hyperref[exercise_07]{here}.

