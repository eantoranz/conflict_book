% copyright 2020 Edmundo Carmona Antoranz
% Released under the terms of Creative Commons Attribution-ShareAlike 4.0 International Public License

\section{Choose your starting point}

\subsection{Example 5}
\label{example_05}

From \hyperref[git_repo]{git repo}, checkout {\bf 4a12f89865} and merge {\bf aaf633c2ad}.
\footnote{These are the parents of {\bf f4f8dfe127}}.

\begin{lstlisting}[style=c_style,
	basicstyle=\tiny,
	firstnumber=686,
	caption={\bf example 5} - {\bf CB} in {\bf builtin/gc.c}]
<<<<<<< HEAD
if (gc_write_commit_graph &&
    write_commit_graph_reachable(get_object_directory(),
				 !quiet && !daemonized ? COMMIT_GRAPH_WRITE_PROGRESS : 0,
				 NULL))
	return 1;
||||||| 9c9b961d7e
if (gc_write_commit_graph &&
    write_commit_graph_reachable(get_object_directory(),
				 !quiet && !daemonized ? COMMIT_GRAPH_PROGRESS : 0,
				 NULL))
	return 1;
=======
prepare_repo_settings(the_repository);
if (the_repository->settings.gc_write_commit_graph == 1)
	write_commit_graph_reachable(get_object_directory(),
				     !quiet && !daemonized ? COMMIT_GRAPH_PROGRESS : 0,
				     NULL);
>>>>>>> aaf633c2ad
\end{lstlisting}\footnote{This and all of the following code snippets {\it in this chapter} will have
the code shifted {\it a tab} to the left due to spacing issues but when you are working you should
keep the original tabs as seen in the original code).}

Following the same technique we have used so far, start working from {\bf UB} (with markers, for clarity):

\begin{lstlisting}[style=c_style,
	basicstyle=\tiny,
	firstnumber=686,
	caption={\bf example 5} - {\bf UB} in {\bf CB}]
<<<<<<< HEAD
if (gc_write_commit_graph &&
    write_commit_graph_reachable(get_object_directory(),
				 !quiet && !daemonized ? COMMIT_GRAPH_WRITE_PROGRESS : 0,
				 NULL))
	return 1;
||||||| 9c9b961d7e
\end{lstlisting}

Let's get the {\bf dML}:

\begin{lstlisting}[style=c_style,
	basicstyle=\tiny,
	firstnumber=692,
	caption={\bf example 5} - {\bf MB} and {\bf LB} in {\bf CB}]
||||||| 9c9b961d7e
if (gc_write_commit_graph &&
    write_commit_graph_reachable(get_object_directory(),
				 !quiet && !daemonized ? COMMIT_GRAPH_PROGRESS : 0,
				 NULL))
	return 1;
=======
prepare_repo_settings(the_repository);
if (the_repository->settings.gc_write_commit_graph == 1)
	write_commit_graph_reachable(get_object_directory(),
				     !quiet && !daemonized ? COMMIT_GRAPH_PROGRESS : 0,
				     NULL);
>>>>>>> aaf633c2ad
\end{lstlisting}

{\bf dML}: A call to {\bf prepare\_repo\_settings()} was prepended (line 699), the conditional statement was modified
so that the {\it standalone} variable {\bf gc\_write\_commit\_graph} from line 694 is now part of a {\bf struct * } on
line 700. The call to {\bf write\_commit\_graph\_reachable()} originally used as part of the conditional statement (line 694) is
now part of the {\bf if-true} section of the conditional (line 701) so adjusted formatting and finally removed the {\bf return}
statement (present on line 697).

Let's replicate all those changes on {\bf UB}.

\begin{lstlisting}[style=c_style,
	basicstyle=\tiny,
	firstnumber=686,
	caption={\bf example 5} - Step 1 - Prepend call]
<<<<<<< HEAD
prepare_repo_settings(the_repository);
if (gc_write_commit_graph &&
    write_commit_graph_reachable(get_object_directory(),
				 !quiet && !daemonized ? COMMIT_GRAPH_WRITE_PROGRESS : 0,
				 NULL))
	return 1;
||||||| 9c9b961d7e
\end{lstlisting}

\begin{lstlisting}[style=c_style,
	basicstyle=\tiny,
	firstnumber=686,
	caption={\bf example 5} - Step 2 - adjust conditional]
<<<<<<< HEAD
prepare_repo_settings(the_repository);
if (the_repository->settings.gc_write_commit_graph == 1)
    write_commit_graph_reachable(get_object_directory(),
				 !quiet && !daemonized ? COMMIT_GRAPH_WRITE_PROGRESS : 0,
				 NULL);
	return 1;
||||||| 9c9b961d7e
\end{lstlisting}

\begin{lstlisting}[style=c_style,
	basicstyle=\tiny,
	firstnumber=686,
	caption={\bf example 5} - Step 3 - adjust formatting]
<<<<<<< HEAD
prepare_repo_settings(the_repository);
if (the_repository->settings.gc_write_commit_graph == 1)
	write_commit_graph_reachable(get_object_directory(),
				     !quiet && !daemonized ? COMMIT_GRAPH_WRITE_PROGRESS : 0,
				     NULL);
	return 1;
||||||| 9c9b961d7e
\end{lstlisting}

\begin{lstlisting}[style=c_style,
	basicstyle=\tiny,
	firstnumber=686,
	caption={\bf example 5} - Step 4 - Remove return]
<<<<<<< HEAD
prepare_repo_settings(the_repository);
if (the_repository->settings.gc_write_commit_graph == 1)
	write_commit_graph_reachable(get_object_directory(),
				     !quiet && !daemonized ? COMMIT_GRAPH_WRITE_PROGRESS : 0,
				     NULL);
||||||| 9c9b961d7e
\end{lstlisting}

Final result:
\begin{lstlisting}[style=c_style,
	basicstyle=\tiny,
	firstnumber=681,
	caption={\bf example 5} - final result]
if (pack_garbage.nr > 0) {
	close_object_store(the_repository->objects);
	clean_pack_garbage();
}

prepare_repo_settings(the_repository);
if (the_repository->settings.gc_write_commit_graph == 1)
	write_commit_graph_reachable(get_object_directory(),
				     !quiet && !daemonized ? COMMIT_GRAPH_WRITE_PROGRESS : 0,
				     NULL);

if (auto_gc && too_many_loose_objects())
\end{lstlisting}

\subsection{Example 5... from LB}

All in all, 4 modifications had to be carried out on the {\bf UB} to solve the conflict. Now, we are {\bf not}
forced to work from {\bf UB}. We can choose to work from {\bf LB} instead.

Let's start over:

\begin{lstlisting}[style=c_style,
	basicstyle=\tiny,
	firstnumber=698,
	caption={\bf example 5} - {\bf LB} in {\bf CB}]
=======
prepare_repo_settings(the_repository);
if (the_repository->settings.gc_write_commit_graph == 1)
	write_commit_graph_reachable(get_object_directory(),
				     !quiet && !daemonized ? COMMIT_GRAPH_PROGRESS : 0,
				     NULL);
>>>>>>> aaf633c2ad
\end{lstlisting}

Let's do the {\bf dMU} analysis this time:

\begin{lstlisting}[style=c_style,
	basicstyle=\tiny,
	firstnumber=686,
	caption={\bf example 5} - {\bf UB} and {\bf MB} in {\bf CB}]
<<<<<<< HEAD
if (gc_write_commit_graph &&
    write_commit_graph_reachable(get_object_directory(),
				 !quiet && !daemonized ? COMMIT_GRAPH_WRITE_PROGRESS : 0,
				 NULL))
	return 1;
||||||| 9c9b961d7e
if (gc_write_commit_graph &&
    write_commit_graph_reachable(get_object_directory(),
				 !quiet && !daemonized ? COMMIT_GRAPH_PROGRESS : 0,
				 NULL))
	return 1;
=======
\end{lstlisting}

{\bf dMU}: The original {\bf COMMIT\_GRAPH\_PROGRESS} on line 695 is now {\bf COMMIT\_GRAPH\_WRITE\_PROGRESS} on line 689.
\footnote{Remember this is {\bf dMU} so we analyze from the {\bf MB} to the {\bf UB}, so the analysis has to be done
going {\bf up}!}

Let's replicate this change on {\bf LB}:

\begin{lstlisting}[style=c_style,
	basicstyle=\tiny,
	firstnumber=698,
	caption={\bf example 5} - Step 1]
=======
prepare_repo_settings(the_repository);
if (the_repository->settings.gc_write_commit_graph == 1)
	write_commit_graph_reachable(get_object_directory(),
				     !quiet && !daemonized ? COMMIT_GRAPH_WRITE_PROGRESS : 0,
				     NULL);
>>>>>>> aaf633c2ad
\end{lstlisting}

And we are done! Final result:
\begin{lstlisting}[style=c_style,
	basicstyle=\tiny,
	firstnumber=681,
	caption={\bf example 5} - final result]
if (pack_garbage.nr > 0) {
	close_object_store(the_repository->objects);
	clean_pack_garbage();
}

prepare_repo_settings(the_repository);
if (the_repository->settings.gc_write_commit_graph == 1)
	write_commit_graph_reachable(get_object_directory(),
				     !quiet && !daemonized ? COMMIT_GRAPH_WRITE_PROGRESS : 0,
				     NULL);

if (auto_gc && too_many_loose_objects())
\end{lstlisting}

Which is exactly like the previous conflict resolution we tried. The only thing is that we had to work {\bf less} because
we started from the block of code that was more different from {\bf MB}. Less changes to detect and bring
over... and therefore less opportunities to mess up.

