% copyright 2020 Edmundo Carmona Antoranz
% Released under the terms of Creative Commons Attribution-ShareAlike 4.0 International Public License

\chapter{Exercises}

\section{Exercise 1}
\label{exercise_01}
\subsection*{Problem}
Take branch {\bf exercise1/branchA } from the \hyperref[exercises_repo]{\bf exercises repo}. There is a short list of
irregular verbs inside {\bf irregular.txt} in alphabetical order. Add these two verbs: {\bf drink} and {\bf know}. Then
take branch {\bf example1/branchB } and add these other 2 verbs: {\bf lose } and {\bf keep }. Merge the two branches.

\subsection*{Conflict}
You should see a conflict like this:
\begin{lstlisting}[style=console_style, numbers=left, caption={\bf Exercise 1} - conflict]
# List of irregular verbs
# simple form<tab>past tense<tab>past participle
catch	caught	caught
draw	drew	drawn
drink	drank	drinken
eat	ate	eaten
fight	fought	fought
fly	flew	flown
grow	grew	grown
hang	hung	hung
<<<<<<< HEAD
know	knew	known
=======
keep	kept	kept
>>>>>>> exercise1/branchB
let	let	let
lose	lost	lost
read	read	read
run	ran	run
sleep	slept	slept
\end{lstlisting}

You might have gotten the conflct section branches in reverse order (first {\bf keep}, then {\bf know}) if you were on {\bf branchB} 
and then tried to merge {\bf branchA}. Notice how the conflict is only related to {\bf know/keep}. The other two verbs,
{\bf drink} and {\bf lose}, were merged correctly, even though they come from different branches.

\subsection*{Resolution}
Given that the verbs are ordered alphabetically, the resulting file looks like this:
\begin{lstlisting}[style=console_style, numbers=left, caption={\bf Exercise 1} - Resolution]
# List of irregular verbs
# simple form<tab>past tense<tab>past participle
catch	caught	caught
draw	drew	drawn
drink	drank	drinken
eat	ate	eaten
fight	fought	fought
fly	flew	flown
grow	grew	grown
hang	hung	hung
keep	kept	kept
know	knew	known
let	let	let
lose	lost	lost
read	read	read
run	ran	run
sleep	slept	slept
\end{lstlisting}
Given that the list is sorted alphabetically, {\bf keep} had to be placed before {\bf know}.


\section{Exercise 2}
\label{exercise_02}
From the \hyperref[exercises_repo]{\bf exercises repo}, merge branches {\bf exercise2/branchA} and {\bf exercise2/branchB}.

\subsection*{Conflict}
\begin{lstlisting}[style=python_style, caption={\bf Exercise 2} - conflict]
#!/usr/bin/python

import sys

colors = {"black": "black mirror",
          "white":  "white noise",
          "blue": "blue sky"}

def getPhrase(color):
<<<<<<< HEAD
    phrase = colors[color.lower()]
=======
    if color not in colors:
        sys.stderr.write("Got no phrase for color %s\n" % color)
        sys.exit(1)
    phrase = colors[color]
>>>>>>> exercise2/branchB
    return phrase

print(getPhrase(sys.argv[1]))
\end{lstlisting}

\subsection*{Resolution}
Correct resolution {\bf might} be something like this:

\begin{lstlisting}[style=python_style, caption={\bf Exercise 2} - resolution]
#!/usr/bin/python

import sys

colors = {"black": "black mirror",
          "white":  "white noise",
          "blue": "blue sky"}

def getPhrase(color):
    if color.lower() not in colors:
        sys.stderr.write("Got no phrase for color %s\n" % color)
        sys.exit(1)
    phrase = colors[color.lower()]
    return phrase

print(getPhrase(sys.argv[1]))
\end{lstlisting}

If you didn't use the {\bf lower()} call on line 10, it's ok... for the time being. I will explain the rationale to
add that call later on. You could have done other things like save color to be lowercase before going into the {\bf if} on
line 10 but we will be touching on the things that we could/should do as we go through other topics. Rest assured, this is just a {\bf teaser} of a conflict.

\section{Exercise 3}
\label{exercise_03}
From \hyperref[git_repo]{git's repo}, checkout revision {\bf fe870600fe} and merge {\bf 1bdca81641}
\footnote{These are the parents of revision {\bf f8cb64e3d4}}. Solve both conflicts (there are 2 conflicting files,
one conflict section on each one of the files).

\subsection*{Conflict in path.c}
\begin{lstlisting}[style=c_style, firstnumber=852, caption={\bf Exercise 3} - conflict in {\bf path.c}]
	if (is_git_directory(".")) {
<<<<<<< HEAD
		set_git_dir(".", 0);
		check_repository_format();
||||||| 51ebf55b93
		set_git_dir(".");
		check_repository_format();
=======
		set_git_dir(".");
		check_repository_format(NULL);
>>>>>>> 1bdca81641
		return path;
	}
\end{lstlisting}
I don't think it is too difficult to realize what happened on each branch. On {\bf HEAD} the call to {\bf set\_git\_dir()} got a second
parameter on line 854. On {\bf the other branch}, the call to {\bf check\_repository\_format()} got a new NULL parameter on line 861.
That points to having this as the conflict resolution:

\begin{lstlisting}[style=c_style, firstnumber=852, caption={\bf Exercise 3} - resolution to conflict in {\bf path.c}]
	if (is_git_directory(".")) {
		set_git_dir(".", 0);
		check_repository_format(NULL);
		return path;
	}
\end{lstlisting}

\subsection*{Conflict in builtin/pack-objects.c}
\begin{lstlisting}[style=c_style, basicstyle=\small, firstnumber=880, caption={\bf Exercise 3} - conflict in {\bf builtin/pack-objects.c}]
	len = encode_in_pack_object_header(header, sizeof(header),
					   OBJ_REF_DELTA, size);
	hashwrite(out, header, len);
<<<<<<< HEAD
	hashwrite(out, base_oid.hash, 20);
||||||| 51ebf55b93
	hashwrite(out, base_sha1, 20);
=======
	hashwrite(out, base_sha1, the_hash_algo->rawsz);
>>>>>>> 1bdca81641
	copy_pack_data(out, reuse_packfile, w_curs, cur, next - cur);
	return;
\end{lstlisting}

In this case, on {\bf HEAD} the second parameter to the {\bf hashwrite()} call was changed from {\bf base\_sha1} on
line 886 to {\bf base\_oid.hash} on line 884. On {\bf the ther branch}, the {\it third parameter} to the same call was
changed from {\bf 20} on line 886 to {\bf the\_hash\_algo-$>$rawsz} on line 888. This all points to this as the conflict
resolution:

\begin{lstlisting}[style=c_style, firstnumber=880, basicstyle=\small, caption={\bf Exercise 3} - conflict in {\bf builtin/pack-objects.c}]
	len = encode_in_pack_object_header(header, sizeof(header),
					   OBJ_REF_DELTA, size);
	hashwrite(out, header, len);
	hashwrite(out, base_oid.hash, the_hash_algo->rawsz);
	copy_pack_data(out, reuse_packfile, w_curs, cur, next - cur);
	return;
\end{lstlisting}

If you compare with revision {\bf f8cb64e3d4}, you should get no meaningful difference.

As an {\bf additional} exercise, imagine what the task would have been {\it if} {\bf diff3} had not been used.

\section{Exercise 4 - a git conflict}
\label{exercise_04}
From \hyperref[git_repo]{git's repo}, checkout {\bf d9d65e9f6a} and merge {\bf b57e8119e6}.
\footnote{These are the parents of revision {\bf 01f8d78887}}.

\subsection*{CB in contrib/completion/git-completion.bash}
\begin{lstlisting}[style=c_style,
	firstnumber=2614,
	basicstyle=\tiny,
	caption={\bf Exercise 4} - CB in {\bf contrib/completion/git-completion.bash}]
<<<<<<< HEAD
	local subcommands="add status init deinit update summary foreach sync absorbgitdirs"
||||||| d62dad7a7d
	local subcommands="add status init deinit update summary foreach sync"
=======
	local subcommands="add status init deinit update set-branch summary foreach sync"
>>>>>>> b57e8119e6
\end{lstlisting}

\subsection*{Working from {\bf UB} (keeping markers for clarity)}
\begin{lstlisting}[style=c_style,
	firstnumber=2614,
	basicstyle=\tiny,
	caption={\bf Exercise 4} - UB in {\bf contrib/completion/git-completion.bash}]
<<<<<<< HEAD
	local subcommands="add status init deinit update summary foreach sync absorbgitdirs"
||||||| d62dad7a7d
\end{lstlisting}

\subsection*{Get {\bf dML}}
\begin{lstlisting}[style=c_style,
	firstnumber=2616,
	basicstyle=\tiny,
	caption={\bf Exercise 4} - {\bf MB} and {\bf LB} in {\bf contrib/completion/git-completion.bash}]
||||||| d62dad7a7d
	local subcommands="add status init deinit update summary foreach sync"
=======
	local subcommands="add status init deinit update set-branch summary foreach sync"
>>>>>>> b57e8119e6
\end{lstlisting}

{\bf dML}: {\bf set-branch} is added between {\bf update} and {\bf summary} as a {\bf subcommand}.

\subsection*{Apply {\bf dML} on {\bf UB}}
\begin{lstlisting}[style=c_style,
	firstnumber=2614,
	basicstyle=\tiny,
	caption={\bf Exercise 4} - UB in {\bf contrib/completion/git-completion.bash}]
<<<<<<< HEAD
	local subcommands="add status init deinit update set-branch summary foreach sync absorbgitdirs"
||||||| d62dad7a7d
\end{lstlisting}

\subsection*{Result}
\begin{lstlisting}[style=c_style,
	firstnumber=2610,
	basicstyle=\tiny,
	caption={\bf Exercise 4} - Conflict Resolution]
_git_submodule ()
{
	__git_has_doubledash && return

	local subcommands="add status init deinit update set-branch summary foreach sync absorbgitdirs"
	local subcommand="$(__git_find_on_cmdline "$subcommands")"
\end{lstlisting}


\section{Exercise 5 - another git conflict}
\label{exercise_05}
From \hyperref[git_repo]{git's repo}, checkout revision {\bf cf054f817a} and merge revision {\bf caf388caa1}.
\footnote{These are the parents of revision {\bf 9b6606f43d}}.

I will let you work on this conflict on your own. There are 5 CBs in 3 files. When you are done, compare
with {\bf 9b6606f43d} and you should get no meaningful differences.

\section{Exercise 6}
\label{exercise_06}

From \hyperref[git_repo]{git repo}, checkout revision {\bf 0194c9ad72} and merge revision {\bf aa46a0da30}. There will be
a {\bf CB} in {\bf ref-filter.c}:

\begin{lstlisting}[style=c_style,
	firstnumber=843,
	caption={\bf Exercise 6} - {\bf CB} in {\bf ref-filter.c}]
<<<<<<< HEAD
/*
 * Given an object name, read the object data and size, and return a
 * "struct object".  If the object data we are returning is also borrowed
 * by the "struct object" representation, set *eaten as well---it is a
 * signal from parse_object_buffer to us not to free the buffer.
 */
static void *get_obj(const struct object_id *oid, struct object **obj, unsigned long *sz, int *eaten)
{
	enum object_type type;
	void *buf = read_object_file(oid, &type, sz);

	if (buf)
		*obj = parse_object_buffer(the_repository, oid, type, *sz,
					   buf, eaten);
	else
		*obj = NULL;
	return buf;
}

||||||| e3331758f1
/*
 * Given an object name, read the object data and size, and return a
 * "struct object".  If the object data we are returning is also borrowed
 * by the "struct object" representation, set *eaten as well---it is a
 * signal from parse_object_buffer to us not to free the buffer.
 */
static void *get_obj(const struct object_id *oid, struct object **obj, unsigned long *sz, int *eaten)
{
	enum object_type type;
	void *buf = read_object_file(oid, &type, sz);

	if (buf)
		*obj = parse_object_buffer(oid, type, *sz, buf, eaten);
	else
		*obj = NULL;
	return buf;
}

=======
>>>>>>> aa46a0da30
\end{lstlisting}

It's a conflict ouf of deleted code.

\subsection*{dMU}
Modified the call to {\bf parse\_object\_buffer()} to have {\bf the\_repository} as the first argument on line 856.

\subsection*{dML}
Deleted the whole section of code.

\subsection*{Resolution}
Should we delete the whole {\bf CB} and move on with our lives? I really hope you had a Pavlov reflex saying {\bf NOOOOO!!!!!}
Good! All the hard work is paying off. We need to find out what happened with that code. The hard way is to find out
in what revision that code was deleted to see if it was moved. First, we do the {\bf git blame --reverse}:

\begin{lstlisting}[style=console_style,
	caption={\bf Exercise 6} - {\bf git blame --reverse} on {\bf ref-filter.c}]
.
.
.
04f6ee1a58e  807)       if (buf)
04f6ee1a58e  808)               *obj = parse_object_buffer(oid, type, *sz, buf, eaten);
04f6ee1a58e  809)       else
04f6ee1a58e  810)               *obj = NULL;
\end{lstlisting}

The last revision when the line was {\bf present} was on {\bf 04f6ee1a58e}. Let's check what follows that revision:
\begin{lstlisting}[style=console_style,
	caption={\bf Exercise 6} - {\bf git log --oneline} on {\bf ref-filter.c}]
$ git log --oneline --graph 04f6ee1a58e..aa46a0da30 -- ref-filter.c
* aa46a0da30 ref-filter: use oid_object_info() to get object
* e2255179f6 ref-filter: merge get_obj and get_object
\end{lstlisting}

And checking those two revisions we find that the code was {\it apparently} {\bf deleted} in {\bf e2255179f6}.

\begin{lstlisting}[style=c_style,
	firstnumber=843,
	caption={\bf Exercise 6} - {\bf git show e2255179f6}]
$ git show e2255179f6 --pretty=
diff --git a/ref-filter.c b/ref-filter.c
index 8db7ca95b1..2b401a17c4 100644
--- a/ref-filter.c
+++ b/ref-filter.c
@@ -797,24 +797,6 @@ int verify_ref_format(struct ref_format *format)
        return 0;
 }
 
-/*
- * Given an object name, read the object data and size, and return a
- * "struct object".  If the object data we are returning is also borrowed
- * by the "struct object" representation, set *eaten as well---it is a
- * signal from parse_object_buffer to us not to free the buffer.
- */
-static void *get_obj(const struct object_id *oid, struct object **obj, unsigned long *sz, int *eaten)
-{
-       enum object_type type;
-       void *buf = read_object_file(oid, &type, sz);
-
-       if (buf)
-               *obj = parse_object_buffer(oid, type, *sz, buf, eaten);
-       else
-               *obj = NULL;
-       return buf;
-}
-
 static int grab_objectname(const char *name, const struct object_id *oid,
                           struct atom_value *v, struct used_atom *atom)
 {
@@ -1437,21 +1419,25 @@ static const char *get_refname(struct used_atom *atom, struct ref_array_item *re
 }
 
 static int get_object(struct ref_array_item *ref, const struct object_id *oid,
-                      int deref, struct object **obj, struct strbuf *err)
+                     int deref, struct object **obj, struct strbuf *err)
 {
        /* parse_object_buffer() will set eaten to 0 if free() will be needed */
        int eaten = 1;
        int ret = 0;
        unsigned long size;
-       void *buf = get_obj(oid, obj, &size, &eaten);
+       enum object_type type;
+       void *buf = read_object_file(oid, &type, &size);
        if (!buf)
                ret = strbuf_addf_ret(err, -1, _("missing object %s for %s"),
                                      oid_to_hex(oid), ref->refname);
-       else if (!*obj)
-               ret = strbuf_addf_ret(err, -1, _("parse_object_buffer failed on %s for %s"),
-                                     oid_to_hex(oid), ref->refname);
-       else
-               grab_values(ref->value, deref, *obj, buf, size);
+       else {
+               *obj = parse_object_buffer(oid, type, size, buf, &eaten);
+               if (!*obj)
+                       ret = strbuf_addf_ret(err, -1, _("parse_object_buffer failed on %s for %s"),
+                                             oid_to_hex(oid), ref->refname);
+               else
+                       grab_values(ref->value, deref, *obj, buf, size);
+       }
        if (!eaten)
                free(buf);
        return ret;
\end{lstlisting}

In the second {\bf hunk} we can see that the call to {\bf parse\_object\_buffer} was {\bf added} back. In other words,
the call was {\bf moved}, and that call is {\it still} in our conflicted file, in line 1520:
\begin{lstlisting}[style=c_style,
	firstnumber=1519,
	caption={\bf Exercise 6} - section of {\bf ref-filter.c}]
	if (oi->info.contentp) {
		*obj = parse_object_buffer(&oi->oid, oi->type, oi->size, oi->content, &eaten);
		if (!obj) {
			if (!eaten)
				free(oi->content);
			return strbuf_addf_ret(err, -1, _("parse_object_buffer failed on %s for %s"),
					       oid_to_hex(&oi->oid), ref->refname);
		}
		grab_values(ref->value, deref, *obj, oi->content, oi->size);
	}
\end{lstlisting}

So all we need to do, as our {\bf dMU} requests, is to add {\bf the\_repository} as the first argument
to the call on line 1520:
\begin{lstlisting}[style=c_style,
	firstnumber=1519,
	caption={\bf Exercise 6} - adjusted section of {\bf ref-filter.c}]
	if (oi->info.contentp) {
		*obj = parse_object_buffer(the_repository, &oi->oid, oi->type, oi->size, oi->content, &eaten);
		if (!obj) {
			if (!eaten)
				free(oi->content);
			return strbuf_addf_ret(err, -1, _("parse_object_buffer failed on %s for %s"),
					       oid_to_hex(&oi->oid), ref->refname);
		}
		grab_values(ref->value, deref, *obj, oi->content, oi->size);
	}
\end{lstlisting}

We remove the original {\bf CB}:
\begin{lstlisting}[style=c_style,
	firstnumber=840,
	caption={\bf Exercise 6} - remove {\bf CB} in {\bf ref-filter.c}]
	return 0;
}

static int grab_objectname(const char *name, const struct object_id *oid,
			   struct atom_value *v, struct used_atom *atom)
{
\end{lstlisting}

And if you now check differences with revision {\bf c83149ace6}, you should get none.

\section{Exercise 7}
\label{exercise_07}

From \hyperref[git_repo]{git repo}, checkout revision {\bf 4284497396} and merge {\bf cdb5330a9b}\footnote{These are the
parents of revision {\bf dcd6a8c09a}}. Solve all conflicts.

\subsection*{CB in diff.h}

There's one API conflict in {\bf diff.h}

\begin{lstlisting}[style=c_style,
	basicstyle=\small,
	firstnumber=443,
	caption={\bf Exercise 7} - {\bf CB} in {\bf diff.h}]
<<<<<<< HEAD
void diff_no_index(struct rev_info *, int, const char **);
||||||| 125dcea963
void diff_no_index(struct repository *, struct rev_info *, int, const char **);
=======
int diff_no_index(struct repository *, struct rev_info *,
		  int implicit_no_index, int, const char **);
>>>>>>> cdb5330a9b
\end{lstlisting}

Looks very straight forward:

\subsubsection*{dMU}
Removed {\bf struct repository *} as an argument to {\bf diff\_no\_index} call.

\subsubsection*{dML}
Added (or rather, {\it inserted}) {\bf int implicit\_no\_index} as an argument to {\bf diff\_no\_index} call.

\subsubsection*{Resolution}
Given that it is much simpler to {\bf delete} things than moving things around, we will work from {\bf LB}.

\begin{lstlisting}[style=c_style,
	firstnumber=447,
	caption={\bf Exercise 7} - Step 1 {\bf LB} in {\bf diff.h}]
=======
int diff_no_index(struct repository *, struct rev_info *,
		  int implicit_no_index, int, const char **);
>>>>>>> cdb5330a9b
\end{lstlisting}

Then we delete {\bf struct repository *} as an argument.

\begin{lstlisting}[style=c_style,
	firstnumber=447,
	caption={\bf Exercise 7} - Step 2 {\bf LB} in {\bf diff.h}]
=======
int diff_no_index(struct rev_info *,
		  int implicit_no_index, int, const char **);
>>>>>>> cdb5330a9b
\end{lstlisting}

\begin{lstlisting}[style=c_style,
	firstnumber=441,
	caption={\bf Exercise 7} - Resolution {\bf CB} in {\bf diff.h}]
int diff_result_code(struct diff_options *, int);

int diff_no_index(struct rev_info *, int implicit_no_index, int,
		  const char **);

int index_differs_from(struct repository *r, const char *def,
		       const struct diff_flags *flags,
		       int ita_invisible_in_index);
\end{lstlisting}

Notice how I repositioned the arguments to match a more {\it git-project standards} formatting.

This of course means that all existing calls to {\bf diff\_no\_index} will be broken. It's likely that there will
be some conflicting ones, but we already know that there might be some that might get away with it that we will
have to hunt down.\footnote{Given that this is a c project, it's not like those broken calls that didn't get a conflict
will make it alive. If we avoid them for any reason, when we compile the project they will burst in flames. Working on
python? Well, you might have to sit down to {\bf really} find them.}

\subsection*{CB in builtin/diff.c}

This one will make us sweat a little bit.

\begin{lstlisting}[style=c_style,
	basicstyle=\small,
	firstnumber=324,
	caption={\bf Exercise 7} - {\bf CB} in {\bf builtin/diff.c}]
<<<<<<< HEAD
	if (no_index && argc != i + 2) {
		if (no_index == DIFF_NO_INDEX_IMPLICIT) {
			/*
			 * There was no --no-index and there were not two
			 * paths. It is possible that the user intended
			 * to do an inside-repository operation.
			 */
			fprintf(stderr, "Not a git repository\n");
			fprintf(stderr,
				"To compare two paths outside a working tree:\n");
		}
		/* Give the usage message for non-repository usage and exit. */
		usagef("git diff %s <path> <path>",
		       no_index == DIFF_NO_INDEX_EXPLICIT ?
		       "--no-index" : "[--no-index]");

	}
||||||| 125dcea963
	if (no_index && argc != i + 2) {
		if (no_index == DIFF_NO_INDEX_IMPLICIT) {
			/*
			 * There was no --no-index and there were not two
			 * paths. It is possible that the user intended
			 * to do an inside-repository operation.
			 */
			fprintf(stderr, "Not a git repository\n");
			fprintf(stderr,
				"To compare two paths outside a working tree:\n");
		}
		/* Give the usage message for non-repository usage and exit. */
		usagef("git diff %s <path> <path>",
		       no_index == DIFF_NO_INDEX_EXPLICIT ?
		       "--no-index" : "[--no-index]");

	}
	if (no_index)
		/* If this is a no-index diff, just run it and exit there. */
		diff_no_index(the_repository, &rev, argc, argv);

	/* Otherwise, we are doing the usual "git" diff */
	rev.diffopt.skip_stat_unmatch = !!diff_auto_refresh_index;
=======
	if (no_index)
		/* If this is a no-index diff, just run it and exit there. */
		exit(diff_no_index(the_repository, &rev,
				   no_index == DIFF_NO_INDEX_IMPLICIT,
				   argc, argv));

	/* Otherwise, we are doing the usual "git" diff */
	rev.diffopt.skip_stat_unmatch = !!diff_auto_refresh_index;
>>>>>>> cdb5330a9b
\end{lstlisting}

This one looks like each one of the blocks has taken care of removing one part of the {\bf MB}. Let's go through
with the whole analysis process.

\subsubsection*{dMU}
The second conditional from {\bf MB} (lines 360-362) along with the code that followed have been removed.

\subsubsection*{dML}
The first conditional from {\bf MB} (lines 343-359) has been removed. The call to {\bf diff\_no\_index} got
a new parameter (from the expression {\bf no\_index == DIFF\_NO\_INDEX\_IMPLICIT }) and it is now enclosed inside one
{\bf exit} call (lines 360-362). At this point I want you to be very {\it cautious} about what will happen with conflict
resolution because this is the call that we had to work for {\bf API conflicts} in the previous {\bf CB}.


\subsubsection*{Resolution}

Always wanting to do a thorough analysis, let's check when/how the code was {\bf deleted} on each branch.

\paragraph{Code deleted toward HEAD}
First, let's try a reverse blame to see when the code was present last time:

\begin{lstlisting}[style=console_style,
	caption={\bf Exercise 7} - {\bf git blame reverse}]
$ git blame -s --reverse MERGE_HEAD..HEAD builtin/diff.c
.
.
.
^cdb5330a9b 321)        repo_init_revisions(the_repository, &rev, prefix);
^cdb5330a9b 322) 
^cdb5330a9b 323)        if (no_index)
^cdb5330a9b 324)                /* If this is a no-index diff, just run it and exit there. */
^cdb5330a9b 325)                exit(diff_no_index(the_repository, &rev,
^cdb5330a9b 326)                                   no_index == DIFF_NO_INDEX_IMPLICIT,
^cdb5330a9b 327)                                   argc, argv));
^cdb5330a9b 328) 
^cdb5330a9b 329)        /* Otherwise, we are doing the usual "git" diff */
^cdb5330a9b 330)        rev.diffopt.skip_stat_unmatch = !!diff_auto_refresh_index;
^cdb5330a9b 331) 
^cdb5330a9b 332)        /* Scale to real terminal size and respect statGraphWidth config */
\end{lstlisting}

Last time those lines were visible {\bf toward} {\bf HEAD} was on revision {\bf cdb5330a9b} and by checking the history
of the file we will be able to find where those lines were removed. There are ony a handful of revisions to check:

\begin{lstlisting}[style=console_style,
	caption={\bf Exercise 7} - revisions to check]
$ git log --oneline cdb5330a9b..HEAD -- builtin/diff.c
12e5bdd9c4 Merge branch 'jk/diff-no-index-initialize'
287ab28bfa diff: reuse diff setup for --no-index case
3a14fdec88 Merge branch 'sl/const'
33de80b1d5 various: tighten constness of some local variables
f8adbec9fe cache.h: flip NO_THE_INDEX_COMPATIBILITY_MACROS switch
1b0d968b34 read-cache.c: replace update_index_if_able with repo_&
\end{lstlisting}

And by checking them one by one we find our culprit in {\bf 287ab28bfae14}:

\begin{lstlisting}[style=console_style,
	caption={\bf Exercise 7} - revision where code was {\it moved}]
$ git show --pretty= 287ab28bfae14 -- builtin/diff.c
diff --git a/builtin/diff.c b/builtin/diff.c
index f0393bba23..777ca87156 100644
--- a/builtin/diff.c
+++ b/builtin/diff.c
@@ -337,21 +337,23 @@ int cmd_diff(int argc, const char **argv, const char *prefix)
                       "--no-index" : "[--no-index]");
 
        }
-       if (no_index)
-               /* If this is a no-index diff, just run it and exit there. */
-               diff_no_index(the_repository, &rev, argc, argv);
-
-       /* Otherwise, we are doing the usual "git" diff */
-       rev.diffopt.skip_stat_unmatch = !!diff_auto_refresh_index;
 
-       /* Scale to real terminal size and respect statGraphWidth config */
+       /* Set up defaults that will apply to both no-index and regular diffs. */
        rev.diffopt.stat_width = -1;
        rev.diffopt.stat_graph_width = -1;
-
-       /* Default to let external and textconv be used */
        rev.diffopt.flags.allow_external = 1;
        rev.diffopt.flags.allow_textconv = 1;
 
+       /* If this is a no-index diff, just run it and exit there. */
+       if (no_index)
+               diff_no_index(&rev, argc, argv);
+
+       /*
+        * Otherwise, we are doing the usual "git" diff; set up any
+        * further defaults that apply to regular diffs.
+        */
+       rev.diffopt.skip_stat_unmatch = !!diff_auto_refresh_index;
+
        /*
         * Default to intent-to-add entries invisible in the
         * index. This makes them show up as new files in diff-files
\end{lstlisting}

We can see that the code was placed a little bit further down the file. And the lines are there {\it currently}
in our conflicted file, outside of the {\bf CB}:

\begin{lstlisting}[style=c_style,
	basicstyle=\small,
	firstnumber=383,
	caption={\bf Exercise 7} - section of {\bf builtin/diff.c}]
	/* If this is a no-index diff, just run it and exit there. */
	if (no_index)
		diff_no_index(&rev, argc, argv);

	/*
	 * Otherwise, we are doing the usual "git" diff; set up any
	 * further defaults that apply to regular diffs.
	 */
	rev.diffopt.skip_stat_unmatch = !!diff_auto_refresh_index;
\end{lstlisting}

And, just to make sure, we can blame the lines to see if they showed up on this revision we are talking about:

\begin{lstlisting}[style=console_style,
	caption={\bf Exercise 7} - revision where code was {\it moved}]
$ git blame -s -L 383,391 -- builtin/diff.c
287ab28bfae 383)        /* If this is a no-index diff, just run it and exit there. */
287ab28bfae 384)        if (no_index)
287ab28bfae 385)                diff_no_index(&rev, argc, argv);
287ab28bfae 386) 
287ab28bfae 387)        /*
287ab28bfae 388)         * Otherwise, we are doing the usual "git" diff; set up any
287ab28bfae 389)         * further defaults that apply to regular diffs.
287ab28bfae 390)         */
287ab28bfae 391)        rev.diffopt.skip_stat_unmatch = !!diff_auto_refresh_index;
\end{lstlisting}

And this is precisely one of the spots where we have to consider the API change. Remember the function
we had to adjust? {\bf diff\_no\_index}. Exactly what we have in line {\bf 385}. You can see that in the patch where the
call is moved, it originally had {\bf the\_repository} as one argument and in the final position that paraneter is {\it gone},
so that part of our original {\bf API conflict} is done. Coming back the {\bf deleted code analysis}, it seems like we
can safely assume that all adjustments have been taken care of properly. Now let's check the deleted code on {\bf the other
branch}.

\paragraph{Code deleted toward the other branch}

\begin{lstlisting}[style=console_style,
	caption={\bf Exercise 7} - {\bf git blame reverse}]
$ git blame -s --reverse HEAD..MERGE_HEAD builtin/diff.c
.
.
.
^4284497396 322)        repo_init_revisions(the_repository, &rev, prefix);
^4284497396 323) 
^4284497396 324)        if (no_index && argc != i + 2) {
^4284497396 325)                if (no_index == DIFF_NO_INDEX_IMPLICIT) {
^4284497396 326)                        /*
^4284497396 327)                         * There was no --no-index and there were not two
^4284497396 328)                         * paths. It is possible that the user intended
^4284497396 329)                         * to do an inside-repository operation.
^4284497396 330)                         */
^4284497396 331)                        fprintf(stderr, "Not a git repository\n");
^4284497396 332)                        fprintf(stderr,
^4284497396 333)                                "To compare two paths outside a working tree:\n");
^4284497396 334)                }
^4284497396 335)                /* Give the usage message for non-repository usage and exit. */
^4284497396 336)                usagef("git diff %s <path> <path>",
^4284497396 337)                       no_index == DIFF_NO_INDEX_EXPLICIT ?
^4284497396 338)                       "--no-index" : "[--no-index]");
^4284497396 339) 
^4284497396 340)        }
\end{lstlisting}

From {\bf 4284497396}, let's see what revisions are following it:

\begin{lstlisting}[style=console_style,
	caption={\bf Exercise 7} - revisions to check]
$ git log 4284497396..MERGE_HEAD --oneline -- builtin/diff.c
16bb3d714d diff --no-index: use parse_options() instead of diff_opt_parse()
\end{lstlisting}

Not that many revisions, huh? Let's check:

\begin{lstlisting}[style=console_style,
	caption={\bf Exercise 7} - revision where code was {\it deleted} and call {\bf adjusted}]
$ git show 16bb3d714d --pretty= -- builtin/diff.c
diff --git a/builtin/diff.c b/builtin/diff.c
index f0393bba23..52dc3e136f 100644
--- a/builtin/diff.c
+++ b/builtin/diff.c
@@ -320,26 +320,11 @@ int cmd_diff(int argc, const char **argv, const char *prefix)
 
        repo_init_revisions(the_repository, &rev, prefix);
 
-       if (no_index && argc != i + 2) {
-               if (no_index == DIFF_NO_INDEX_IMPLICIT) {
-                       /*
-                        * There was no --no-index and there were not two
-                        * paths. It is possible that the user intended
-                        * to do an inside-repository operation.
-                        */
-                       fprintf(stderr, "Not a git repository\n");
-                       fprintf(stderr,
-                               "To compare two paths outside a working tree:\n");
-               }
-               /* Give the usage message for non-repository usage and exit. */
-               usagef("git diff %s <path> <path>",
-                      no_index == DIFF_NO_INDEX_EXPLICIT ?
-                      "--no-index" : "[--no-index]");
-
-       }
        if (no_index)
                /* If this is a no-index diff, just run it and exit there. */
-               diff_no_index(the_repository, &rev, argc, argv);
+               exit(diff_no_index(the_repository, &rev,
+                                  no_index == DIFF_NO_INDEX_IMPLICIT,
+                                  argc, argv));
 
        /* Otherwise, we are doing the usual "git" diff */
        rev.diffopt.skip_stat_unmatch = !!diff_auto_refresh_index;
\end{lstlisting}

And here we can see how the code was {\bf indeed} deleted, and we got an adjustment in the call to {\bf diff\_no\_index}.
If we are so kind to adjust the call that we saw is still in line 385, we should be fine. As given that the {\bf MB} was
completely {\bf deleted} by what both branches did, that points to this as the ultimate final resolution:


\begin{lstlisting}[style=c_style,
	firstnumber=318,
	caption={\bf Exercise 7} - {\bf CB} in {\bf builtin/diff.c}]
	init_diff_ui_defaults();
	git_config(git_diff_ui_config, NULL);
	precompose_argv(argc, argv);

	repo_init_revisions(the_repository, &rev, prefix);

	/* Set up defaults that will apply to both no-index and regular diffs. */
	rev.diffopt.stat_width = -1;
	rev.diffopt.stat_graph_width = -1;
	rev.diffopt.flags.allow_external = 1;
	rev.diffopt.flags.allow_textconv = 1;

	/* If this is a no-index diff, just run it and exit there. */
	if (no_index)
		exit(diff_no_index(&rev, no_index == DIFF_NO_INDEX_IMPLICIT,
				   argc, argv));

	/*
	 * Otherwise, we are doing the usual "git" diff; set up any
	 * further defaults that apply to regular diffs.
	 */
	rev.diffopt.skip_stat_unmatch = !!diff_auto_refresh_index;
\end{lstlisting}

Notice how the call to {\bf diff\_no\_index} is enclosed in a call to {\bf exit} and how we inserted
the new parameter with the value coming from {\bf the other branch}.


\subsection*{CB\#1 in diff-no-index.c}

There are 2 {\bf CB}s in {\bf diff-no-index.c}. The first one is directly related to the signature of the function that
we worked on {\bf diff.h}. Now we are dealing with the signature in the implementation.

\begin{lstlisting}[style=c_style,
	firstnumber=237,
	caption={\bf Exercise 7} - {\bf CB\#1} in {\bf diff-no-index.c}]
<<<<<<< HEAD
void diff_no_index(struct rev_info *revs,
		   int argc, const char **argv)
||||||| 125dcea963
void diff_no_index(struct repository *r,
		   struct rev_info *revs,
		   int argc, const char **argv)
=======
static const char * const diff_no_index_usage[] = {
	N_("git diff --no-index [<options>] <path> <path>"),
	NULL
};

int diff_no_index(struct repository *r,
		  struct rev_info *revs,
		  int implicit_no_index,
		  int argc, const char **argv)
>>>>>>> cdb5330a9b
\end{lstlisting}

\subsubsection*{dMU}
Removed {\bf struct repository *r} as an argument to {\bf diff\_no\_index}.

\subsubsection*{dML}
Added the definition of {\bf diff\_no\_index\_usage} and also {\it inserted} {\bf implicit\_no\_index} as an argument to
the {\bf diff\_no\_index} call.

\subsubsection*{Resolution}
It's more than obvious we need to work from {\bf LB}, right?

\begin{lstlisting}[style=c_style,
	firstnumber=244,
	caption={\bf Exercise 7} - Step 1 - {\bf LB\#1} in {\bf diff-no-index.c}]
=======
static const char * const diff_no_index_usage[] = {
	N_("git diff --no-index [<options>] <path> <path>"),
	NULL
};

int diff_no_index(struct repository *r,
		  struct rev_info *revs,
		  int implicit_no_index,
		  int argc, const char **argv)
>>>>>>> cdb5330a9b
\end{lstlisting}

We remove {\bf struct the repository *r} as an argument, per {\bf dMU}.

\begin{lstlisting}[style=c_style,
	firstnumber=244,
	caption={\bf Exercise 7} - Step 2 - {\bf LB\#1} in {\bf diff-no-index.c}]
=======
static const char * const diff_no_index_usage[] = {
	N_("git diff --no-index [<options>] <path> <path>"),
	NULL
};

int diff_no_index(struct rev_info *revs,
		  int implicit_no_index,
		  int argc, const char **argv)
>>>>>>> cdb5330a9b
\end{lstlisting}

Done with {\bf CB\#1}:

\begin{lstlisting}[style=c_style,
	firstnumber=234,
	caption={\bf Exercise 7} - Resolution of {\bf CB\#1} in {\bf diff-no-index.c}]
	}
}

static const char * const diff_no_index_usage[] = {
	N_("git diff --no-index [<options>] <path> <path>"),
	NULL
};

int diff_no_index(struct rev_info *revs,
		  int implicit_no_index,
		  int argc, const char **argv)
{
	int i, no_index;
	const char *paths[2];
\end{lstlisting}

Can you take care of solving the remaning conflicts in the project? Here's what has to be done, once you are done:

\subsection*{CB\#2 in diff-no-index.c}
\begin{lstlisting}[style=c_style,
	firstnumber=255,
	caption={\bf Exercise 7} - Resolution of {\bf CB\#2} in {\bf diff-no-index.c}]
	struct option *options;

	options = parse_options_concat(no_index_options,
				       revs->diffopt.parseopts);
	argc = parse_options(argc, argv, revs->prefix, options,
			     diff_no_index_usage, 0);
	if (argc != 2) {
		if (implicit_no_index)
			warning(_("Not a git repository. Use --no-index to "
				  "compare two paths outside a working tree"));
		usage_with_options(diff_no_index_usage, options);
	}
	FREE_AND_NULL(options);
	for (i = 0; i < 2; i++) {
\end{lstlisting}

No more need for the call to {\bf repo\_diff\_setup} because {\bf r} (the repository) is no longer a part of the
parameters of the function we are working on.

\subsection*{CB in diff.c}

\begin{lstlisting}[style=c_style,
	firstnumber=5530,
	caption={\bf Exercise 7} - Resolution of {\bf CB} in {\bf diff.c}]
	ac = parse_options(ac, av, prefix, options->parseopts, NULL,
			   PARSE_OPT_KEEP_DASHDASH |
			   PARSE_OPT_KEEP_UNKNOWN |
			   PARSE_OPT_NO_INTERNAL_HELP |
			   PARSE_OPT_ONE_SHOT |
			   PARSE_OPT_STOP_AT_NON_OPTION);

	return ac;
}
\end{lstlisting}

We keep {\bf LB} {\it as is} because the new definition to handle {\bf color-moved-ws} per revision {\bf 8ce2020ff07995} will
also take care of {\bf --no-color-moved-ws}.

\subsubsection{What's missing?}
Last step would be checkout for any calls to the conflicted function:

\begin{lstlisting}[style=console_style,
	caption={\bf Exercise 7} - calls to {\bf diff\_no\_index}]
$ git grep -n "diff_no_index("
builtin/diff.c:332:             diff_no_index(&rev, no_index == DIFF_NO_INDEX_IMPLICIT, 
diff-no-index.c:242:int diff_no_index(struct rev_info *revs,
diff.h:443:int diff_no_index(struct rev_info *, int implicit_no_index, int,
\end{lstlisting}

And it's all related to the code that we worked on while working on the conflict so we are all good.

Finally, if you compare with {\bf dcd6a8c09a}, you should get no meaningful differences.

