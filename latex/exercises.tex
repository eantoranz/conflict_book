4 % copyright 2020 Edmundo Carmona Antoranz
% Released under the terms of Creative Commons Attribution-ShareAlike 4.0 International Public License

\chapter{Ejercicios}

\section{Ejercicio 1}
\label{exercise_01}
\subsection*{Problema}
Tome la rama {\bf exercise1/branchA } del \hyperref[exercises_repo]{\bf repositorio de ejercicios}. Hay una lista
e verbos irregulares en inglés dentro de {\bf irregular.txt} en orden alfabético. Agregue estos dos verbos: {\bf drink} y {\bf know}.
Luego tome {\bf example1/branchB } y agregue estos dos verbos: {\bf lose } y {\bf keep }. Mezcle ambas ramas.

\subsection*{Conflicto}
Se debe ver un conflicto así:
\begin{lstlisting}[style=console_style, numbers=left, caption={\bf Ejercicio 1} - conflicto]
# List of irregular verbs
# simple form<tab>past tense<tab>past participle
catch	caught	caught
draw	drew	drawn
drink	drank	drinken
eat	ate	eaten
fight	fought	fought
fly	flew	flown
grow	grew	grown
hang	hung	hung
<<<<<<< HEAD
know	knew	known
=======
keep	kept	kept
>>>>>>> exercise1/branchB
let	let	let
lose	lost	lost
read	read	read
run	ran	run
sleep	slept	slept
\end{lstlisting}

Puede ser que las secciones del conflicto estén en orden inverso (primero {\bf keep} y luego {\bf know}) si estaban en la rama
{\bf branchB} e intentaron mezclar la rama {\bf branchA}. Fíjense que el conflicto sol se refiere a {\bf keep/know}. Los otros dos verbos,
{\bf drink} y {\bf lose}, se mezclaron correctamente, unque vienen de ramas diferentes.

\subsection*{Resolución}
Dado que los verbos se están agregando en orden alfabético, el archivo resultante se debe ver así:
\begin{lstlisting}[style=console_style, numbers=left, caption={\bf Ejercicio 1} - Resolución]
# List of irregular verbs
# simple form<tab>past tense<tab>past participle
catch	caught	caught
draw	drew	drawn
drink	drank	drinken
eat	ate	eaten
fight	fought	fought
fly	flew	flown
grow	grew	grown
hang	hung	hung
keep	kept	kept
know	knew	known
let	let	let
lose	lost	lost
read	read	read
run	ran	run
sleep	slept	slept
\end{lstlisting}
Dado que la lista es alfabética, se debe colocar {\bf keep} seguido de {\bf know}.


\section{Ejercicio 2}
\label{exercise_02}
Del \hyperref[exercises_repo]{\bf repositorio de ejercicios}, mezcle las ramas {\bf exercise2/branchA} y {\bf exercise2/branchB}.

\subsection*{Conflicto}
\begin{lstlisting}[style=python_style, caption={\bf Ejercicio 2} - conflicto]
#!/usr/bin/python

import sys

colors = {"black": "black mirror",
          "white":  "white noise",
          "blue": "blue sky"}

def getPhrase(color):
<<<<<<< HEAD
    phrase = colors[color.lower()]
=======
    if color not in colors:
        sys.stderr.write("Got no phrase for color %s\n" % color)
        sys.exit(1)
    phrase = colors[color]
>>>>>>> exercise2/branchB
    return phrase

print(getPhrase(sys.argv[1]))
\end{lstlisting}

\subsection*{Resolution}
La solución correcta debe ser algo así:

\begin{lstlisting}[style=python_style, caption={\bf Ejercicio 2} - resolución]
#!/usr/bin/python

import sys

colors = {"black": "black mirror",
          "white":  "white noise",
          "blue": "blue sky"}

def getPhrase(color):
    if color.lower() not in colors:
        sys.stderr.write("Got no phrase for color %s\n" % color)
        sys.exit(1)
    phrase = colors[color.lower()]
    return phrase

print(getPhrase(sys.argv[1]))
\end{lstlisting}

Si no colocaron la llamada a {\bf lower()} en la línea 10, está bien... por el momento. Explicaré la lógica
para agregar la llamada más adelante. Pudieron haber hecho otras cosas como guardar el color para que quede en minúsculas antes
de entrar al {\bf if} de la línea 10 pero veremos las cosas que se podría/debería hacer a medida que avanzamos en otros tópicos.
Puedo asegurarles algo, esto es solo un abrebocas de lo que es un conflicto.

\section{Ejercicio 3}
\label{exercise_03}
\subsubsection{Ejercicio 3}
Del \hyperref[git_repo]{repo de git}, hacer checkout de la revisión {\bf fe870600fe} y mezclar {\bf 1bdca81641}
\footnote{Estos son los padres de la revisión {\bf f8cb64e3d4}}. Resuelvan ambos conflictos (hay 2 archivos con conflctos,
un conflicto en cada archivo).

\subsection*{Conflicto en path.c}
\begin{lstlisting}[style=c_style, firstnumber=852, caption={\bf Ejercicio 3} - conflicto en {\bf path.c}]
	if (is_git_directory(".")) {
<<<<<<< HEAD
		set_git_dir(".", 0);
		check_repository_format();
||||||| 51ebf55b93
		set_git_dir(".");
		check_repository_format();
=======
		set_git_dir(".");
		check_repository_format(NULL);
>>>>>>> 1bdca81641
		return path;
	}
\end{lstlisting}
Creo que noes muy difícil entender lo que sucedió en cada rama. En {\bf HEAD} la llamada a {\bf set\_git\_dir()}
toma un segundo parámetro en la línea 854. En {\bf la otra rama}, la llamada a {\bf check\_repository\_format()} tiene un
nuevo parámetro {\bf NULL} en la línea 861. Esto apunta a tener esto como resolución del conflicto:

\begin{lstlisting}[style=c_style, firstnumber=852, caption={\bf Ejercicio 3} - resolución del conflicto en {\bf path.c}]
	if (is_git_directory(".")) {
		set_git_dir(".", 0);
		check_repository_format(NULL);
		return path;
	}
\end{lstlisting}

\subsection*{Conflicto en builtin/pack-objects.c}
\begin{lstlisting}[style=c_style, basicstyle=\small, firstnumber=880, caption={\bf Ejercicio 3} - conflicto en {\bf builtin/pack-objects.c}]
	len = encode_in_pack_object_header(header, sizeof(header),
					   OBJ_REF_DELTA, size);
	hashwrite(out, header, len);
<<<<<<< HEAD
	hashwrite(out, base_oid.hash, 20);
||||||| 51ebf55b93
	hashwrite(out, base_sha1, 20);
=======
	hashwrite(out, base_sha1, the_hash_algo->rawsz);
>>>>>>> 1bdca81641
	copy_pack_data(out, reuse_packfile, w_curs, cur, next - cur);
	return;
\end{lstlisting}

En este caso, en {\bf HEAD} el {\it segundo parámetro} a la llamada a {\bf hashwrite()} fue cambiado de {\bf base\_sha1} en
la línea 886 a {\bf base\_oid.hash} en la línea 884. En {\bf la otra rama}, el {\it tercer parámetro} a la misma llamada
cambió de {\bf 20} en la línea 886 a {\bf the\_hash\_algo-$>$rawsz} en la línea 888. Esto apunta a esto como resolución:

\begin{lstlisting}[style=c_style, firstnumber=880, basicstyle=\small, caption={\bf Ejercicio 3} - resolución del conflicto en {\bf builtin/pack-objects.c}]
	len = encode_in_pack_object_header(header, sizeof(header),
					   OBJ_REF_DELTA, size);
	hashwrite(out, header, len);
	hashwrite(out, base_oid.hash, the_hash_algo->rawsz);
	copy_pack_data(out, reuse_packfile, w_curs, cur, next - cur);
	return;
\end{lstlisting}

Si comparan con la revisión {\bf f8cb64e3d4}, ne deberían obtener diferencias significativas.

Como ejercicio {\bf adicional},  imaginen lo que tendrían que hacer para resolver esos conflictos si no usaran {\bf diff3}.

\section{Ejercicio 4 - un conflicto de git}
\label{exercise_04}
Del \hyperref[git_repo]{repo de git}, hagan checkout de la revisión {\bf d9d65e9f6a} y hagan {\it merge} de {\bf b57e8119e6}
\footnote{Estos son los padres de la revisión {\bf 01f8d78887}}.

\subsection*{CB en contrib/completion/git-completion.bash}
\begin{lstlisting}[style=c_style,
	firstnumber=2614,
	basicstyle=\tiny,
	caption={\bf Ejercicio 4} - CB en {\bf contrib/completion/git-completion.bash}]
<<<<<<< HEAD
	local subcommands="add status init deinit update summary foreach sync absorbgitdirs"
||||||| d62dad7a7d
	local subcommands="add status init deinit update summary foreach sync"
=======
	local subcommands="add status init deinit update set-branch summary foreach sync"
>>>>>>> b57e8119e6
\end{lstlisting}

\subsection*{Trabajando en el {\bf UB} (mantenemos los marcadores por claridad)}
\begin{lstlisting}[style=c_style,
	firstnumber=2614,
	basicstyle=\tiny,
	caption={\bf Ejercicio 4} - UB en {\bf contrib/completion/git-completion.bash}]
<<<<<<< HEAD
	local subcommands="add status init deinit update summary foreach sync absorbgitdirs"
||||||| d62dad7a7d
\end{lstlisting}

\subsection*{Analizar {\bf dML}}
\begin{lstlisting}[style=c_style,
	firstnumber=2616,
	basicstyle=\tiny,
	caption={\bf Ejercicio 4} - {\bf MB} y {\bf LB} en {\bf contrib/completion/git-completion.bash}]
||||||| d62dad7a7d
	local subcommands="add status init deinit update summary foreach sync"
=======
	local subcommands="add status init deinit update set-branch summary foreach sync"
>>>>>>> b57e8119e6
\end{lstlisting}

{\bf dML}: se agrega {\bf set-branch} entre {\bf update} y {\bf summary} como un nuevo {\bf subcomando}.

\subsection*{Aplicar {\bf dML} en el {\bf UB}}
\begin{lstlisting}[style=c_style,
	firstnumber=2614,
	basicstyle=\tiny,
	caption={\bf Ejercicio 4} - UB en {\bf contrib/completion/git-completion.bash}]
<<<<<<< HEAD
	local subcommands="add status init deinit update set-branch summary foreach sync absorbgitdirs"
||||||| d62dad7a7d
\end{lstlisting}

\subsection*{Resultado}
\begin{lstlisting}[style=c_style,
	firstnumber=2610,
	basicstyle=\tiny,
	caption={\bf Ejercicio 4} - Resolución de conflicto]
_git_submodule ()
{
	__git_has_doubledash && return

	local subcommands="add status init deinit update set-branch summary foreach sync absorbgitdirs"
	local subcommand="$(__git_find_on_cmdline "$subcommands")"
\end{lstlisting}


\section{Ejercicio 5 - otro conflicto en git}
\label{exercise_05}
Del \hyperref[git_repo]{repo de git}, hagan checkout de la revisión {\bf cf054f817a} y hagan {\it merge} de {\bf caf388caa1}.
\footnote{Estos son los padres de la revisión {\bf 9b6606f43d}}

Dejaré que trabajen en la solución por su cuenta. Hay 5 {\bf CBs} en 3 archivos. Cuando terminen, comparen
con {\bf 9b6606f43d} y no deberían tener diferencias significativas.

\section{Ejercicio 6}
\label{exercise_06}

Del \hyperref[git_repo]{repo de git}, hagan checkout de la revisión {\bf 0194c9ad72} y hagan un merge de {\bf aa46a0da30}.
Van a ver un {\bf CB} en {\bf ref-filter.c}:

\begin{lstlisting}[style=c_style,
	firstnumber=843,
	caption={\bf Ejercico 6} - {\bf CB} en {\bf ref-filter.c}]
<<<<<<< HEAD
/*
 * Given an object name, read the object data and size, and return a
 * "struct object".  If the object data we are returning is also borrowed
 * by the "struct object" representation, set *eaten as well---it is a
 * signal from parse_object_buffer to us not to free the buffer.
 */
static void *get_obj(const struct object_id *oid, struct object **obj, unsigned long *sz, int *eaten)
{
	enum object_type type;
	void *buf = read_object_file(oid, &type, sz);

	if (buf)
		*obj = parse_object_buffer(the_repository, oid, type, *sz,
					   buf, eaten);
	else
		*obj = NULL;
	return buf;
}

||||||| e3331758f1
/*
 * Given an object name, read the object data and size, and return a
 * "struct object".  If the object data we are returning is also borrowed
 * by the "struct object" representation, set *eaten as well---it is a
 * signal from parse_object_buffer to us not to free the buffer.
 */
static void *get_obj(const struct object_id *oid, struct object **obj, unsigned long *sz, int *eaten)
{
	enum object_type type;
	void *buf = read_object_file(oid, &type, sz);

	if (buf)
		*obj = parse_object_buffer(oid, type, *sz, buf, eaten);
	else
		*obj = NULL;
	return buf;
}

=======
>>>>>>> aa46a0da30
\end{lstlisting}

Es un conflicto por código borrado.

\subsection*{dMU}
Se modificó la llamada a {\bf parse\_object\_buffer()} para que el primer argumento sea {\bf the\_repository} en la línea 856.

\subsection*{dML}
Se borró toda la sección de código.

\subsection*{Resolución}
Deberíamos borrar todo el {\bf CB} y seguir adelante con nuestras vidas? Espero que hayan tenid un rejlejo de Pavlov diciendo
{\bf NOOOOO!!!!!}. Excelente! Todo ese trabajo duro está dando frutos. Debemos averiguar qué pasó con ese código. La forma
dura es averiguar en qué revision se borró el código para ver si fue movido. Primero hacemos el {\bf git blame --reverse}:

\begin{lstlisting}[style=console_style,
	caption={\bf Ejercicio 6} - {\bf git blame --reverse} en {\bf ref-filter.c}]
.
.
.
04f6ee1a58e  807)       if (buf)
04f6ee1a58e  808)               *obj = parse_object_buffer(oid, type, *sz, buf, eaten);
04f6ee1a58e  809)       else
04f6ee1a58e  810)               *obj = NULL;
\end{lstlisting}

La última revisión en la cual la línea estuvo {\bf presente} fue en {\bf 04f6ee1a58e}. Vemos qué sigue a esa revisión:
\begin{lstlisting}[style=console_style,
	caption={\bf Ejercicio 6} - {\bf git log --oneline} en {\bf ref-filter.c}]
$ git log --oneline --graph 04f6ee1a58e..aa46a0da30 -- ref-filter.c
* aa46a0da30 ref-filter: use oid_object_info() to get object
* e2255179f6 ref-filter: merge get_obj and get_object
\end{lstlisting}

Y al verificar esas dos revisions vemos queel código fue {\it aparentemente} {\bf borrado} en {\bf e2255179f6}:

\begin{lstlisting}[style=c_style,
	firstnumber=843,
	caption={\bf Ejercicio 6} - {\bf git show e2255179f6}]
$ git show e2255179f6 --pretty=
diff --git a/ref-filter.c b/ref-filter.c
index 8db7ca95b1..2b401a17c4 100644
--- a/ref-filter.c
+++ b/ref-filter.c
@@ -797,24 +797,6 @@ int verify_ref_format(struct ref_format *format)
        return 0;
 }
 
-/*
- * Given an object name, read the object data and size, and return a
- * "struct object".  If the object data we are returning is also borrowed
- * by the "struct object" representation, set *eaten as well---it is a
- * signal from parse_object_buffer to us not to free the buffer.
- */
-static void *get_obj(const struct object_id *oid, struct object **obj, unsigned long *sz, int *eaten)
-{
-       enum object_type type;
-       void *buf = read_object_file(oid, &type, sz);
-
-       if (buf)
-               *obj = parse_object_buffer(oid, type, *sz, buf, eaten);
-       else
-               *obj = NULL;
-       return buf;
-}
-
 static int grab_objectname(const char *name, const struct object_id *oid,
                           struct atom_value *v, struct used_atom *atom)
 {
@@ -1437,21 +1419,25 @@ static const char *get_refname(struct used_atom *atom, struct ref_array_item *re
 }
 
 static int get_object(struct ref_array_item *ref, const struct object_id *oid,
-                      int deref, struct object **obj, struct strbuf *err)
+                     int deref, struct object **obj, struct strbuf *err)
 {
        /* parse_object_buffer() will set eaten to 0 if free() will be needed */
        int eaten = 1;
        int ret = 0;
        unsigned long size;
-       void *buf = get_obj(oid, obj, &size, &eaten);
+       enum object_type type;
+       void *buf = read_object_file(oid, &type, &size);
        if (!buf)
                ret = strbuf_addf_ret(err, -1, _("missing object %s for %s"),
                                      oid_to_hex(oid), ref->refname);
-       else if (!*obj)
-               ret = strbuf_addf_ret(err, -1, _("parse_object_buffer failed on %s for %s"),
-                                     oid_to_hex(oid), ref->refname);
-       else
-               grab_values(ref->value, deref, *obj, buf, size);
+       else {
+               *obj = parse_object_buffer(oid, type, size, buf, &eaten);
+               if (!*obj)
+                       ret = strbuf_addf_ret(err, -1, _("parse_object_buffer failed on %s for %s"),
+                                             oid_to_hex(oid), ref->refname);
+               else
+                       grab_values(ref->value, deref, *obj, buf, size);
+       }
        if (!eaten)
                free(buf);
        return ret;
\end{lstlisting}

Pero en el segundo {\bf hunk} se puede ver que la llamada a {\bf parse\_object\_buffer} fue {\bf agregada} de nuevo. En otras
palabras, la llamada fue {\bf movida}, y esa llamada sigue en nuestro archivo, en la línea 1520:
\begin{lstlisting}[style=c_style,
	firstnumber=1519,
	caption={\bf Ejercicio 6} - sección de {\bf ref-filter.c}]
	if (oi->info.contentp) {
		*obj = parse_object_buffer(&oi->oid, oi->type, oi->size, oi->content, &eaten);
		if (!obj) {
			if (!eaten)
				free(oi->content);
			return strbuf_addf_ret(err, -1, _("parse_object_buffer failed on %s for %s"),
					       oid_to_hex(&oi->oid), ref->refname);
		}
		grab_values(ref->value, deref, *obj, oi->content, oi->size);
	}
\end{lstlisting}

Así que todo loque debemos hacer, como nos lo quiere el {\bf dMU}, es agregar {\bf the\_repository}
como primer argumento a la llamada e la línea 1520:
\begin{lstlisting}[style=c_style,
	firstnumber=1519,
	caption={\bf Ejercicio 6} - Sección ajustada de {\bf ref-filter.c}]
	if (oi->info.contentp) {
		*obj = parse_object_buffer(the_repository, &oi->oid, oi->type, oi->size, oi->content, &eaten);
		if (!obj) {
			if (!eaten)
				free(oi->content);
			return strbuf_addf_ret(err, -1, _("parse_object_buffer failed on %s for %s"),
					       oid_to_hex(&oi->oid), ref->refname);
		}
		grab_values(ref->value, deref, *obj, oi->content, oi->size);
	}
\end{lstlisting}

Se remuele el {\bf CB} original:
\begin{lstlisting}[style=c_style,
	firstnumber=840,
	caption={\bf Ejercicio 6} - {\bf CB} removido en {\bf ref-filter.c}]
	return 0;
}

static int grab_objectname(const char *name, const struct object_id *oid,
			   struct atom_value *v, struct used_atom *atom)
{
\end{lstlisting}

Y si verifican las diferencias con la revisión {\bf c83149ace6}, no deberían obtener ninguna.

\section{Ejercicio 7}
\label{exercise_07}

De el \hyperref[git_repo]{git repo}, hagen checkout de la revisión {\bf 4284497396} and hagan merge de {\bf cdb5330a9b}\footnote{estos son
los padres de la revisión {\bf dcd6a8c09a}}. Resuelvan todos los conflictos.

\subsection*{CB en diff.h}

Hay un {\bf conflicto de API} en {\bf diff.h}

\begin{lstlisting}[style=c_style,
	basicstyle=\small,
	firstnumber=443,
	caption={\bf Ejercicio 7} - {\bf CB} en {\bf diff.h}]
<<<<<<< HEAD
void diff_no_index(struct rev_info *, int, const char **);
||||||| 125dcea963
void diff_no_index(struct repository *, struct rev_info *, int, const char **);
=======
int diff_no_index(struct repository *, struct rev_info *,
		  int implicit_no_index, int, const char **);
>>>>>>> cdb5330a9b
\end{lstlisting}

Parece bastante sencillo

\subsubsection*{dMU}
Se removió el parámetro {\bf struct repository *} como argumento a la función {\bf diff\_no\_index}.

\subsubsection*{dML}
Se agregó (o mejor dicho, {\it se insertó}) {\bf int implicit\_no\_index} como argumento a la función {\bf diff\_no\_index}.

\subsubsection*{Resolución}
Dado que es más sencillo {\bf borrar} que estar moviendo cosas de un bloque al otro, trabajaremos desde el {\bf LB}.

\begin{lstlisting}[style=c_style,
	firstnumber=447,
	caption={\bf Ejercicio 7} - Paso 1 {\bf LB} en {\bf diff.h}]
=======
int diff_no_index(struct repository *, struct rev_info *,
		  int implicit_no_index, int, const char **);
>>>>>>> cdb5330a9b
\end{lstlisting}

Entonces borramos {\bf struct repository *} como argumento.

\begin{lstlisting}[style=c_style,
	firstnumber=447,
	caption={\bf Ejercicio 7} - Paso 2 {\bf LB} en {\bf diff.h}]
=======
int diff_no_index(struct rev_info *,
		  int implicit_no_index, int, const char **);
>>>>>>> cdb5330a9b
\end{lstlisting}

\begin{lstlisting}[style=c_style,
	firstnumber=441,
	caption={\bf Ejercicio 7} - Resolución de {\bf CB} en {\bf diff.h}]
int diff_result_code(struct diff_options *, int);

int diff_no_index(struct rev_info *, int implicit_no_index, int,
		  const char **);

int index_differs_from(struct repository *r, const char *def,
		       const struct diff_flags *flags,
		       int ita_invisible_in_index);
\end{lstlisting}

Noten como reorganicé los parámetros de la llamada para que tuviera un formato más acorde a los {\it standares del proyecto git}.

Esto, claro está, significa que todas las llamadas existentes a {\bf diff\_no\_index} estarán rotas. Es bastante probable que haya
algunas que presenten conflictos, pero ya sabemos que podría haber otras que escapan a conflictos y que tendremos que cazar
\footnote{Dado que este es un proyecto c, con solo compilar el proyecto podremos detectar las llamadas que están rotas. Están
trabajando con Python? Más vale que salgan a cazarlas}.

\subsection*{CB en builtin/diff.c}

Esto nos hará sudar un poco.

\begin{lstlisting}[style=c_style,
	basicstyle=\small,
	firstnumber=324,
	caption={\bf Ejercicio 7} - {\bf CB} en {\bf builtin/diff.c}]
<<<<<<< HEAD
	if (no_index && argc != i + 2) {
		if (no_index == DIFF_NO_INDEX_IMPLICIT) {
			/*
			 * There was no --no-index and there were not two
			 * paths. It is possible that the user intended
			 * to do an inside-repository operation.
			 */
			fprintf(stderr, "Not a git repository\n");
			fprintf(stderr,
				"To compare two paths outside a working tree:\n");
		}
		/* Give the usage message for non-repository usage and exit. */
		usagef("git diff %s <path> <path>",
		       no_index == DIFF_NO_INDEX_EXPLICIT ?
		       "--no-index" : "[--no-index]");

	}
||||||| 125dcea963
	if (no_index && argc != i + 2) {
		if (no_index == DIFF_NO_INDEX_IMPLICIT) {
			/*
			 * There was no --no-index and there were not two
			 * paths. It is possible that the user intended
			 * to do an inside-repository operation.
			 */
			fprintf(stderr, "Not a git repository\n");
			fprintf(stderr,
				"To compare two paths outside a working tree:\n");
		}
		/* Give the usage message for non-repository usage and exit. */
		usagef("git diff %s <path> <path>",
		       no_index == DIFF_NO_INDEX_EXPLICIT ?
		       "--no-index" : "[--no-index]");

	}
	if (no_index)
		/* If this is a no-index diff, just run it and exit there. */
		diff_no_index(the_repository, &rev, argc, argv);

	/* Otherwise, we are doing the usual "git" diff */
	rev.diffopt.skip_stat_unmatch = !!diff_auto_refresh_index;
=======
	if (no_index)
		/* If this is a no-index diff, just run it and exit there. */
		exit(diff_no_index(the_repository, &rev,
				   no_index == DIFF_NO_INDEX_IMPLICIT,
				   argc, argv));

	/* Otherwise, we are doing the usual "git" diff */
	rev.diffopt.skip_stat_unmatch = !!diff_auto_refresh_index;
>>>>>>> cdb5330a9b
\end{lstlisting}

En este, parece que cada una de las ramas se encargó de borrar una de las partes del {\bf MB}. Hagamos el análisis completo
de todos modos.

\subsubsection*{dMU}
Se borró el segundo condicional del {\bf MB} (líneas 360-362) junto con todo lo que le seguía.

\subsubsection*{dML}
El primer condicional del {\bf MB} (líneas 343-359) fue borrado. La llamada a {\bf diff\_no\_index} ahra tiene un
nuevo parámetro (a partir de la expresión {\bf no\_index == DIFF\_NO\_INDEX\_IMPLICIT }) y la llamada está incluída
dentro de una llamada a {\bf exit} (líneas 360-362). En este punto quiero que estén bien atentos acerca de la resolución del
conflicto porque esta esla llamada sobre la que hubo un {\bf conflicto de API} en el {\bf CB} anterior.

\subsubsection*{Resolución}

Siempre queriendo cuidar los detalles, verifiquemos cuando/como se borró el código en cada rama.

\paragraph{Código borrado hacia HEAD}
Primero hagamos el {\bf blame reverso} para ver cuando fue la última vez que esas líneas estuvieron presentes:

\begin{lstlisting}[style=console_style,
	caption={\bf Ejercicio 7} - {\bf git blame reverse}]
$ git blame -s --reverse MERGE_HEAD..HEAD builtin/diff.c
.
.
.
^cdb5330a9b 321)        repo_init_revisions(the_repository, &rev, prefix);
^cdb5330a9b 322) 
^cdb5330a9b 323)        if (no_index)
^cdb5330a9b 324)                /* If this is a no-index diff, just run it and exit there. */
^cdb5330a9b 325)                exit(diff_no_index(the_repository, &rev,
^cdb5330a9b 326)                                   no_index == DIFF_NO_INDEX_IMPLICIT,
^cdb5330a9b 327)                                   argc, argv));
^cdb5330a9b 328) 
^cdb5330a9b 329)        /* Otherwise, we are doing the usual "git" diff */
^cdb5330a9b 330)        rev.diffopt.skip_stat_unmatch = !!diff_auto_refresh_index;
^cdb5330a9b 331) 
^cdb5330a9b 332)        /* Scale to real terminal size and respect statGraphWidth config */
\end{lstlisting}

La útima vez que esas líneas estuvieron presentes {\it hacia} {\bf HEAD} fue en la revisión {\bf cdb5330a9b} y al verificar la historia
del achivo veremos que solo hay un puñado de versiones posteriores a ella:

\begin{lstlisting}[style=console_style,
	caption={\bf Ejercicio 7} - revisiones que verificar]
$ git log --oneline cdb5330a9b..HEAD -- builtin/diff.c
12e5bdd9c4 Merge branch 'jk/diff-no-index-initialize'
287ab28bfa diff: reuse diff setup for --no-index case
3a14fdec88 Merge branch 'sl/const'
33de80b1d5 various: tighten constness of some local variables
f8adbec9fe cache.h: flip NO_THE_INDEX_COMPATIBILITY_MACROS switch
1b0d968b34 read-cache.c: replace update_index_if_able with repo_&
\end{lstlisting}

y al verificarlas una a una encontramos que el culpable es la revisión {\bf 287ab28bfae14}:

\begin{lstlisting}[style=console_style,
	caption={\bf Ejercicio 7} - revisión donde se {\it movió} el código]
$ git show --pretty= 287ab28bfae14 -- builtin/diff.c
diff --git a/builtin/diff.c b/builtin/diff.c
index f0393bba23..777ca87156 100644
--- a/builtin/diff.c
+++ b/builtin/diff.c
@@ -337,21 +337,23 @@ int cmd_diff(int argc, const char **argv, const char *prefix)
                       "--no-index" : "[--no-index]");
 
        }
-       if (no_index)
-               /* If this is a no-index diff, just run it and exit there. */
-               diff_no_index(the_repository, &rev, argc, argv);
-
-       /* Otherwise, we are doing the usual "git" diff */
-       rev.diffopt.skip_stat_unmatch = !!diff_auto_refresh_index;
 
-       /* Scale to real terminal size and respect statGraphWidth config */
+       /* Set up defaults that will apply to both no-index and regular diffs. */
        rev.diffopt.stat_width = -1;
        rev.diffopt.stat_graph_width = -1;
-
-       /* Default to let external and textconv be used */
        rev.diffopt.flags.allow_external = 1;
        rev.diffopt.flags.allow_textconv = 1;
 
+       /* If this is a no-index diff, just run it and exit there. */
+       if (no_index)
+               diff_no_index(&rev, argc, argv);
+
+       /*
+        * Otherwise, we are doing the usual "git" diff; set up any
+        * further defaults that apply to regular diffs.
+        */
+       rev.diffopt.skip_stat_unmatch = !!diff_auto_refresh_index;
+
        /*
         * Default to intent-to-add entries invisible in the
         * index. This makes them show up as new files in diff-files
\end{lstlisting}

Podemos ver que el código fue colocado un poco más abajo en el achivo. Y las líneas están presentes aún en el archivo, luego del {\bf CB} que estamos
analizando::

\begin{lstlisting}[style=c_style,
	basicstyle=\small,
	firstnumber=383,
	caption={\bf Ejercicio 7} - sección de {\bf builtin/diff.c}]
	/* If this is a no-index diff, just run it and exit there. */
	if (no_index)
		diff_no_index(&rev, argc, argv);

	/*
	 * Otherwise, we are doing the usual "git" diff; set up any
	 * further defaults that apply to regular diffs.
	 */
	rev.diffopt.skip_stat_unmatch = !!diff_auto_refresh_index;
\end{lstlisting}

Y, solo para asegurarnos, podemos hacer un {\bf blame} de las líneas para ver si se corresponden con la revisión que estamos hablando:

\begin{lstlisting}[style=console_style,
	caption={\bf Ejercicio 7} - {\bf git blame}]
$ git blame -s -L 383,391 -- builtin/diff.c
287ab28bfae 383)        /* If this is a no-index diff, just run it and exit there. */
287ab28bfae 384)        if (no_index)
287ab28bfae 385)                diff_no_index(&rev, argc, argv);
287ab28bfae 386) 
287ab28bfae 387)        /*
287ab28bfae 388)         * Otherwise, we are doing the usual "git" diff; set up any
287ab28bfae 389)         * further defaults that apply to regular diffs.
287ab28bfae 390)         */
287ab28bfae 391)        rev.diffopt.skip_stat_unmatch = !!diff_auto_refresh_index;
\end{lstlisting}

Y este es precisamente uno de los sitios donde debemos considerer el cambio de API. Recuerdan la función que debimos ajustar?
{\bf diff\_no\_index}. Exactamente lo que tenemos en la línea {\bf 385}. podemos ver que en el parche donde se mueve la llamada, originalmente
tenía el parámetro {\bf the\_repository} pero en la posición final el parámetro ha sido removido, así qye esa parte de nuestro {\bf conflicto
de API} ya fue resuelto.

Veamos el código borrado de {\bf la otra rama}.

\paragraph{Código borrado hacia la otra rama}

\begin{lstlisting}[style=console_style,
	caption={\bf Ejercicio 7} - {\bf git blame reverse}]
$ git blame -s --reverse HEAD..MERGE_HEAD builtin/diff.c
.
.
.
^4284497396 322)        repo_init_revisions(the_repository, &rev, prefix);
^4284497396 323) 
^4284497396 324)        if (no_index && argc != i + 2) {
^4284497396 325)                if (no_index == DIFF_NO_INDEX_IMPLICIT) {
^4284497396 326)                        /*
^4284497396 327)                         * There was no --no-index and there were not two
^4284497396 328)                         * paths. It is possible that the user intended
^4284497396 329)                         * to do an inside-repository operation.
^4284497396 330)                         */
^4284497396 331)                        fprintf(stderr, "Not a git repository\n");
^4284497396 332)                        fprintf(stderr,
^4284497396 333)                                "To compare two paths outside a working tree:\n");
^4284497396 334)                }
^4284497396 335)                /* Give the usage message for non-repository usage and exit. */
^4284497396 336)                usagef("git diff %s <path> <path>",
^4284497396 337)                       no_index == DIFF_NO_INDEX_EXPLICIT ?
^4284497396 338)                       "--no-index" : "[--no-index]");
^4284497396 339) 
^4284497396 340)        }
\end{lstlisting}

A partir de la revisión {\bf 4284497396}, veamos qué le sigue:

\begin{lstlisting}[style=console_style,
	caption={\bf Ejercicio 7} - revisiones que verificar]
$ git log 4284497396..MERGE_HEAD --oneline -- builtin/diff.c
16bb3d714d diff --no-index: use parse_options() instead of diff_opt_parse()
\end{lstlisting}

No hay muchas revisiones, no? Veamos:

\begin{lstlisting}[style=console_style,
	caption={\bf Ejercicio 7} - revisión donde el código fue {\it borrado} y la llamada {\bf ajustada}]
$ git show 16bb3d714d --pretty= -- builtin/diff.c
diff --git a/builtin/diff.c b/builtin/diff.c
index f0393bba23..52dc3e136f 100644
--- a/builtin/diff.c
+++ b/builtin/diff.c
@@ -320,26 +320,11 @@ int cmd_diff(int argc, const char **argv, const char *prefix)
 
        repo_init_revisions(the_repository, &rev, prefix);
 
-       if (no_index && argc != i + 2) {
-               if (no_index == DIFF_NO_INDEX_IMPLICIT) {
-                       /*
-                        * There was no --no-index and there were not two
-                        * paths. It is possible that the user intended
-                        * to do an inside-repository operation.
-                        */
-                       fprintf(stderr, "Not a git repository\n");
-                       fprintf(stderr,
-                               "To compare two paths outside a working tree:\n");
-               }
-               /* Give the usage message for non-repository usage and exit. */
-               usagef("git diff %s <path> <path>",
-                      no_index == DIFF_NO_INDEX_EXPLICIT ?
-                      "--no-index" : "[--no-index]");
-
-       }
        if (no_index)
                /* If this is a no-index diff, just run it and exit there. */
-               diff_no_index(the_repository, &rev, argc, argv);
+               exit(diff_no_index(the_repository, &rev,
+                                  no_index == DIFF_NO_INDEX_IMPLICIT,
+                                  argc, argv));
 
        /* Otherwise, we are doing the usual "git" diff */
        rev.diffopt.skip_stat_unmatch = !!diff_auto_refresh_index;
\end{lstlisting}

Y aquí podemos ver que el código fue borrado, y tenemos un ajuste a la llamada a {\bf diff\_no\_index}. Si hacemos los ajustes a la llamada
que vimos que existe aún e la línea 385, estará todo muy bien. Y dado que el {\bf MB} fue borrado por completo por los cambios que
cada rama hizo, eso apunta a que esta sea la solución definitiva:


\begin{lstlisting}[style=c_style,
	firstnumber=318,
	caption={\bf Ejercicio 7} - {\bf CB} en {\bf builtin/diff.c}]
	init_diff_ui_defaults();
	git_config(git_diff_ui_config, NULL);
	precompose_argv(argc, argv);

	repo_init_revisions(the_repository, &rev, prefix);

	/* Set up defaults that will apply to both no-index and regular diffs. */
	rev.diffopt.stat_width = -1;
	rev.diffopt.stat_graph_width = -1;
	rev.diffopt.flags.allow_external = 1;
	rev.diffopt.flags.allow_textconv = 1;

	/* If this is a no-index diff, just run it and exit there. */
	if (no_index)
		exit(diff_no_index(&rev, no_index == DIFF_NO_INDEX_IMPLICIT,
				   argc, argv));

	/*
	 * Otherwise, we are doing the usual "git" diff; set up any
	 * further defaults that apply to regular diffs.
	 */
	rev.diffopt.skip_stat_unmatch = !!diff_auto_refresh_index;
\end{lstlisting}

Vean como la llamada a {\bf diff\_no\_index} está dentro de la llamada a {\bf exit} y cómo se insertó
el nuevo parámetro como estaba definido en {\bf la otra rama}.

\subsection*{CB\#1 en diff-no-index.c}

Hay 2 {\bf CB}s en {\bf diff-no-index.c}. El primero se relaciona directamente con la firma de la función sobre la que trabajamos en
{\bf diff.h}. Ahora estamos trabajando sobre la firma en la implementación de la función.

\begin{lstlisting}[style=c_style,
	firstnumber=237,
	caption={\bf Ejercicio 7} - {\bf CB\#1} en {\bf diff-no-index.c}]
<<<<<<< HEAD
void diff_no_index(struct rev_info *revs,
		   int argc, const char **argv)
||||||| 125dcea963
void diff_no_index(struct repository *r,
		   struct rev_info *revs,
		   int argc, const char **argv)
=======
static const char * const diff_no_index_usage[] = {
	N_("git diff --no-index [<options>] <path> <path>"),
	NULL
};

int diff_no_index(struct repository *r,
		  struct rev_info *revs,
		  int implicit_no_index,
		  int argc, const char **argv)
>>>>>>> cdb5330a9b
\end{lstlisting}

\subsubsection*{dMU}
Se quitó el parámetro {\bf struct repository *r} de la función {\bf diff\_no\_index}.

\subsubsection*{dML}
Se agregó el parámetro {\bf diff\_no\_index\_usage} y también se {\it insertó} {\bf implicit\_no\_index} como argumento
a la función {\bf diff\_no\_index}.

\subsubsection*{Resolución}
Es más que obvio que hay que trabajar desde el {\bf LB}, cierto?

\begin{lstlisting}[style=c_style,
	firstnumber=244,
	caption={\bf Ejercicio 7} - Paso 1 - {\bf LB\#1} en {\bf diff-no-index.c}]
=======
static const char * const diff_no_index_usage[] = {
	N_("git diff --no-index [<options>] <path> <path>"),
	NULL
};

int diff_no_index(struct repository *r,
		  struct rev_info *revs,
		  int implicit_no_index,
		  int argc, const char **argv)
>>>>>>> cdb5330a9b
\end{lstlisting}

Removemos {\bf struct the repository *r} como argumento, siguiente el {\bf dMU}.

\begin{lstlisting}[style=c_style,
	firstnumber=244,
	caption={\bf Ejercicio 7} - Paso 2 - {\bf LB\#1} en {\bf diff-no-index.c}]
=======
static const char * const diff_no_index_usage[] = {
	N_("git diff --no-index [<options>] <path> <path>"),
	NULL
};

int diff_no_index(struct rev_info *revs,
		  int implicit_no_index,
		  int argc, const char **argv)
>>>>>>> cdb5330a9b
\end{lstlisting}

Listo con el {\bf CB\#1}:

\begin{lstlisting}[style=c_style,
	firstnumber=234,
	caption={\bf Ejercicio 7} - Resolución del {\bf CB\#1} en {\bf diff-no-index.c}]
	}
}

static const char * const diff_no_index_usage[] = {
	N_("git diff --no-index [<options>] <path> <path>"),
	NULL
};

int diff_no_index(struct rev_info *revs,
		  int implicit_no_index,
		  int argc, const char **argv)
{
	int i, no_index;
	const char *paths[2];
\end{lstlisting}

Podrían encargarse de resolver los otros conflictos en el proyecto? Esto es lo que hay que hacer, una vez que los terminen, para verificar:

\subsection*{CB\#2 en diff-no-index.c}
\begin{lstlisting}[style=c_style,
	firstnumber=255,
	caption={\bf Ejercicio 7} - Resolución del {\bf CB\#2} en {\bf diff-no-index.c}]
	struct option *options;

	options = parse_options_concat(no_index_options,
				       revs->diffopt.parseopts);
	argc = parse_options(argc, argv, revs->prefix, options,
			     diff_no_index_usage, 0);
	if (argc != 2) {
		if (implicit_no_index)
			warning(_("Not a git repository. Use --no-index to "
				  "compare two paths outside a working tree"));
		usage_with_options(diff_no_index_usage, options);
	}
	FREE_AND_NULL(options);
	for (i = 0; i < 2; i++) {
\end{lstlisting}

No se necesita la llamada a {\bf repo\_diff\_setup} porque {\bf r} (el repositotio) ya no es parte de los parámetros
de la función sobre la que estamos trabajando.

\subsection*{CB en diff.c}

\begin{lstlisting}[style=c_style,
	firstnumber=5530,
	caption={\bf Ejercicio 7} - Resolucion del {\bf CB} en {\bf diff.c}]
	ac = parse_options(ac, av, prefix, options->parseopts, NULL,
			   PARSE_OPT_KEEP_DASHDASH |
			   PARSE_OPT_KEEP_UNKNOWN |
			   PARSE_OPT_NO_INTERNAL_HELP |
			   PARSE_OPT_ONE_SHOT |
			   PARSE_OPT_STOP_AT_NON_OPTION);

	return ac;
}
\end{lstlisting}

Nos quedamos con el {\bf LB} {\it tal cual} porque la nueva definición para manejar {\bf color-moved-ws} según la revisión {\bf 8ce2020ff07995}
se hará cargo de la opción {\bf --no-color-moved-ws}.

\subsubsection{Qué falta?}
El último paso sería verificar otras llamas a la función con el conflicto:

\begin{lstlisting}[style=console_style,
	caption={\bf Ejercicio 7} - llamadas a {\bf diff\_no\_index}]
$ git grep -n "diff_no_index("
builtin/diff.c:332:             diff_no_index(&rev, no_index == DIFF_NO_INDEX_IMPLICIT, 
diff-no-index.c:242:int diff_no_index(struct rev_info *revs,
diff.h:443:int diff_no_index(struct rev_info *, int implicit_no_index, int,
\end{lstlisting}

Y todas se relacionan con código que ya trabajamos en el que estamos así que todo esta bien.

Finalmente, si comparan con {\bf dcd6a8c09a}, no deberían obtener ninguna diferencia significativa.

