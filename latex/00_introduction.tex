% copyright 2020 Edmundo Carmona Antoranz
% Released under the terms of Creative Commons Attribution-ShareAlike 4.0 International Public License

\section{Introducción}

Los desarolladores gastan tanto tiempo escribiendo código. Agregando, modificando y borrando código... moviéndolo,
reformateándolo, reformándolo, mejorándolo\footnote{... la mayoría de las veces}. Dado que nos hemos movido a
{\bf Sistemas de Control de Versiones Distribuidos} (o {\it DVCS} por sus siglas en inglés), especialmente git,
estamos ejecutando operaciones que pueden generar conflictos con mucha más frecuencia así que, de hecho, la mayoría
de los desarrolladores gastamos algo de nuestro tiempo resolviendo conflictos, a veces dolorosamente, cada tanto.

Lo que normalmente sucede es que un proyecto es desarrollado por múltiples desarrolladores {\it de forma independiente}
y luego todo este trabajo separado se tiene que juntar. Cuando git trata de mezclar el desarrollo que se hizo en dos direcciones
diferentes, él trata lo mejor que puede de resolver como juntar todo de forma programada.... sin embargo, hay situaciones en las que
git tirará la toalla y levantará la mano para pedir ayuda de una persona para que mire el código y lo arregle. Cuando esto
sucede, se dice que hay un {\it conflicto}.

No hay nada malo en que haya conflictos. En los viejos tiempos, cuando los {\bf Sistema de Control de Versiones Centralizados} eran la
norma, se tenía estás ramas largas que se mezclaban muy poco por el tiempo que toma resolver todos los conflictos. Sin embargo, con el
advenimiento de los DVCSs, mezclar se hace de forma mucho más frecuente y hay muchas más oportunidades de generar conflictos con
operaciones que no se usaban con frecuencia anteriormente (si es que existían en los sistemas centralizados) como cherry-pick,
rebase o stash.

Luego de gastar algo de tiempo resolviendo conflictos, los desarrolladores tienden a hacer todo lo que sea posible por tratar de
evitarlos por el tiempo que toma y por lo complejo que puede ser resolverlos, especialmente cierto si el código que se está
trabajando no le es familiar a la persona que está resolviendo el conflicto pero, una vez que el conflicto está delante de
ti, no hay mucho que se pueda hacer para evitarlo. Se puede buscar una forma de mezclar las partes en conflicto... o se puede
reescribir la sección en conflicto desde 0 (y esperemos que haya un muy buen conjunto de pruebas para verificar que no se está rompiendo
nada).... o se puede rendir, cancelar la operación de mezcla y esperar un nuevo día con un cielo más azul y más claro para intentar
la operación de nuevo.

Dada la cantidad de tiempo que toma desarrollar las habilidades para ser un {\bf mezclador} eficiente y viendo el poco material
que hay disponible para explicar como resolver conflictos, decidí hacer lo propio y me senté a escribir esta guía
{\it básica} para ayudar a las personas a manejarlos.

Espero que cuando hayan terminado la lectura se sientas más tranquilos y cómodos a la hora de enfrentar conflictos.


