% copyright 2020 Edmundo Carmona Antoranz
% Released under the terms of Creative Commons Attribution-ShareAlike 4.0 International Public License

\section{Introducción}

Estás orgulloso de ser un desarrolador. Inviertes mucho tiempo trabajando en cada feature. Escribes
pruebas unitarias preciosas para cada una de las condiciones de los requerimientos (antes de escribir
la primera línea de código que {\bf no es de tests}, cierto? Que viva el TDD!)... y ves las pruebas fallar....
apenas puedes aguantar las ganas de corregirlas. Te sientas a escribir el código del requermiento (ahora sí)
y ves como las pruebas otra vez están pasando sin problemas. Empujas los cambios y creas un PR.... una secuencia
de cambios que se explican solos... lo más próximo que el código pueda estar de ser comparado con poesía. El PR
es mezclado en la rama principal y la vida es bella. Felicitaciones de todos los conocidos. Vuelves a seguir
programando otro requerimiento... luego de tomarte una deliciosa taza de café recien colado, por supuesto.

Pasan 3 semanas. Te halan a una reunión.... necesito especificar que {\bf sin aviso}? O sea, de qué otra forma te podrían
meter en una reunión con todos los pesos pesados del equipo de desarrollo y del equipo de soporte de producción?
Luego de algunos minutos de tratar de obtener algún contexto, finalmente te das cuenta de que todas esas voces estresadas
están refiriénsode al bello... {\it poema} que escribiste antes. Está roto.... EN PRODUCCIÓN. SE NECESITA UN ARREGLO...
{\bf YA!!!}

Luego de lograr que tu pulso vuelva parcialmente a la normalidad, te sientas a analizar el código de lo que está en
producción y, pero qué??? El código de tu PR no está completo!!! Le falta una parte.... en realidad, le falta {\bf bastante}.
Pero cómo??? Luego de analizar un poco más te das cuenta de que hubo un {\bf merge} cuando se iba a cortar la rama
de producción y el código del PR fue.... bueno, {\bf destruido} mientras se resolvía un conflicto. Así que estás
en la línea de fuego... y ni si quiera es tu culpa.

Esta es una situación bastante extrema que podría surgir de un conflicto no resuelto de la forma correcta pero es más que
probable que hayas estado involucrado en una situación así.... o que conozcas alguien al que le haya pasado (mientras más
grande el proyecto, es más probable que se de esta situación).

Resolver conflictos es una {\bf materia engañosa}. Los desarolladores gastan tanto tiempo escribiendo código.
Agregando, modificando y borrando código... moviéndolo, reformateándolo, reformándolo, mejorándolo
\footnote{... la mayoría de las veces}. Dado que nos hemos movido a {\bf Sistemas de Control de Versiones Distribuidos}
(o {\it DVCS} por sus siglas en inglés), especialmente git, estamos ejecutando operaciones que pueden generar conflictos
con mucha más frecuencia así que, de hecho, la mayoría de los desarrolladores gastamos algo de nuestro tiempo resolviendo
conflictos, a veces dolorosamente, cada tanto.

Lo que normalmente sucede es que un proyecto es desarrollado por múltiples desarrolladores {\it de forma independiente}
y luego todo este trabajo separado se tiene que juntar. Cuando git trata de mezclar el desarrollo que se hizo en dos direcciones
diferentes, él trata lo mejor que puede de resolver como juntar todo de forma programada.... sin embargo, hay situaciones en las que
git tirará la toalla y levantará la mano para pedir ayuda de una persona para que mire el código y lo arregle. Cuando esto
sucede, se dice que hay un {\it conflicto}.

No hay nada malo en que haya conflictos. En los viejos tiempos, cuando los {\bf Sistema de Control de Versiones Centralizados} eran la
norma, se tenía estás ramas largas que se mezclaban muy poco por el tiempo que toma resolver todos los conflictos. Sin embargo, con el
advenimiento de los DVCSs, mezclar se hace de forma mucho más frecuente y hay muchas más oportunidades de generar conflictos con
operaciones que no se usaban con frecuencia anteriormente (si es que existían en los sistemas centralizados) como cherry-pick,
rebase o stash.

Luego de gastar algo de tiempo resolviendo conflictos, los desarrolladores tienden a hacer todo lo que sea posible por tratar de
evitarlos por el tiempo que toma y por lo complejo que puede ser resolverlos, especialmente cierto si el código que se está
trabajando no le es familiar a la persona que está resolviendo el conflicto pero, una vez que el conflicto está delante de
ti, no hay mucho que se pueda hacer para evitarlo. Se puede buscar una forma de mezclar las partes en conflicto... o se puede
reescribir la sección en conflicto desde 0 (y esperemos que haya un muy buen conjunto de pruebas para verificar que no se está rompiendo
nada).... o se puede rendir, cancelar la operación de mezcla y esperar un nuevo día con un cielo más azul y más claro para intentar
la operación de nuevo.

Dada la cantidad de tiempo que toma desarrollar las habilidades para ser un {\bf mezclador} eficiente y viendo el poco material
que hay disponible para explicar como resolver conflictos, decidí hacer lo propio y me senté a escribir esta guía
{\it básica} para ayudar a las personas a manejarlos.

Espero que cuando hayan terminado la lectura se sientas más tranquilos y cómodos a la hora de enfrentar conflictos.


