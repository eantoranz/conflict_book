% copyright 2020 Edmundo Carmona Antoranz
% Released under the terms of Creative Commons Attribution-ShareAlike 4.0 International Public License

\section{Introduction}

Developers spend so much time writing code. Actually adding, modifying and deleting code.... moving it around, reformatting it,
refactoring, improving.... etc. As we have moved to {\bf Distributed Version Control Systems} (or {\it DVCS}, for short), git
specially, we are running way more often operations that make conflicts show up so, in fact, most of us developers spend some
time dealing with conflicts, even if painfully, every once in a while.

What normally happens is that a project is developed by multiple developers {\it independently} and then all this separate work
has to be put together. When git tries to merge the work that was developed in two separate directions it tries its best to figure
out how to put it together programmatically.... however, there are some situations where git will give up and raise its hands asking
for a person to actually take a look at the code to help git sort it out. When this happens, it is called {\bf a conflict}.

There's nothing wrong with having conflicts. In the old days when Centralized VCSs were the norm you had these long running branches
that were merged every now and then because of how time-consuming the task can be. However with the advent of Distributed VCSs
merging is much more frequent and you have many more opportunities to generate conflicts with operations that were not used often
(if not completely missing from centralized VCSs) like cherry-picking, rebasing or stashing.

After a while of dealing with conflicts, developers try their best to avoid them because of how long and how complex it can be to
take care of them, specially true if the code involved is not familiar to the person trying to merge the code, but once it is in
front of you there's not much you can do to avoid it. You can either find a way to merge together the conflicting pieces of code...
or you rewrite the whole section of code from scratch (and let's hope you have a great set of tests to make sure you are not breaking
anything)... or you can give up and cancel the merge operation and wait for a day with a bluer clearer sky to try again.

Given the amount of time if takes to master the skills to be a proficient {\bf merger} and seeing how little structured material there
is out there to explain how to deal with conflicts, I have decided to do the proper and write a {\it basic} guide to help people deal
with them.

I hope that by the time you finish reading this guide you find yourself more comfortable when facing them.

