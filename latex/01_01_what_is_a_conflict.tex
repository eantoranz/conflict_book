% copyright 2020 Edmundo Carmona Antoranz
% Released under the terms of Creative Commons Attribution-ShareAlike 4.0 International Public License

\section{What is a Conflict?}

A conflict arises when there are two different directions in which the same piece of code was modified. As long as you keep them separate (as in {\it separate branches}) there will be no problem, but when you try to {\bf merge} them, git will stop and ask you for your help so that you can figure out what needs to be done to finish the merge operation.

\subsection{A non-conflict}

Let me introduce you to one of the files that we will be using (with a few variations along the way) as the {\it basis} for some of our examples moving forward.

\subsubsection{Example 0}

\begin{lstlisting}[style=python_style]
#!/usr/bin/python

import sys

colors = {"black": "black mirror",
          "white": "white noise",
          "blue": "blue sky"}

def getPhrase(color):
    phrase = colors[color]
    return phrase

print(getPhrase(sys.argv[1]))
\end{lstlisting}

It's a very simple python script.

Suppose that two developers start working from this file having to add a different color each. {\bf Developer A} produces this:

\begin{lstlisting}[style=python_style]
#!/usr/bin/python

import sys

colors = {"black": "black mirror",
          "white": "white noise",
          "red": "red tide",
          "blue": "blue sky"}

def getPhrase(color):
    phrase = colors[color]
    return phrase

print(getPhrase(sys.argv[1]))
\end{lstlisting}

We can see that color red was added on line 7.

And {\bf Developer B} produced this:
\begin{lstlisting}[style=python_style]
#!/usr/bin/python

import sys

colors = {"black": "black mirror",
          "green": "green peas",
          "white": "white noise",
          "blue": "blue sky"}

def getPhrase(color):
    phrase = colors[color]
    return phrase

print(getPhrase(sys.argv[1]))
\end{lstlisting}

History of the branches looks like this:
\begin{lstlisting}[style=branch_history_style]
* d3c8087 (example0/branchB) Adding green peas
| * 4f281f2 (example0/branchA) Adding red tide
|/  
* f44c861 Get a phrase from a color
\end{lstlisting}

Given that the lines were added at {\it different} locations (separated by the line that defines color white phrase), then git has
no problem merging them:

\begin{lstlisting}[style=console_style]
$ git merge example0/branchB --no-edit
Auto-merging example.py
Merge made by the 'recursive' strategy.
 example.py | 1 +
 1 file changed, 1 insertion(+)
\end{lstlisting}

Resulting history looks like this: \footnote{If you attempt to do the merge on your own and the revision ID is not the same,
that's not a problem. That is to be expected. Same thing applies for all the following examples if you attempt to do them on your own.}
\begin{lstlisting}[style=branch_history_style]
*   74a049b Merge branch 'example0/branchB' into example0/branchA
|\  
| * d3c8087 (example0/branchB) Adding green peas
* | 4f281f2 Adding red tide
|/  
* f44c861 Get a phrase from a color
\end{lstlisting}

The resulting file looks like this:
\begin{lstlisting}[style=python_style]
#!/usr/bin/python

import sys

colors = {"black": "black mirror",
          "green": "green peas",
          "white": "white noise",
          "red": "red tide",
          "blue": "blue sky"}

def getPhrase(color):
    phrase = colors[color]
    return phrase

print(getPhrase(sys.argv[1]))
\end{lstlisting}

And you can see how the resulting file has both colors green and red:
\begin{lstlisting}[style=code_section_style, firstnumber=5]
colors = {"black": "black mirror", <-- Preexisting line
          "green": "green peas",   <-- Line from Developer B
          "white": "white noise",  <-- Preexisting line
          "red": "red tide",       <-- Line from Developer A
          "blue": "blue sky"}      <-- Preexisting line
\end{lstlisting}

This is also visible if you check the annotations of the file:\footnote{Annotations can be generated with
{\bf git annotate} or {\bf git blame}}

\begin{lstlisting}[style=console_style, basicstyle=\tiny] % FIXME get output that is smaller, the date could be removed)
^f44c861 (Developer A 2020-03-14 09:49:10 -0600  5) colors = {"black": "black mirror",
d3c80878 (Developer B 2020-03-14 09:56:34 -0600  6)           "green": "green peas",
^f44c861 (Developer A 2020-03-14 09:49:10 -0600  7)           "white": "white noise",
4f281f22 (Developer A 2020-03-14 09:54:08 -0600  8)           "red": "red tide",
^f44c861 (Developer A 2020-03-14 09:49:10 -0600  9)           "blue": "blue sky"}
\end{lstlisting}

There we have a successful merge. Git is able to merge the code because they touch different sections of the file.
Git can see that the two pieces of code are separated by the line defining the white color phrase and so the merge
goes fine.

