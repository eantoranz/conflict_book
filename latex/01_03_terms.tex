% copyright 2020 Edmundo Carmona Antoranz
% Released under the terms of Creative Commons Attribution-ShareAlike 4.0 International Public License

\section{Términos y Acrónimos}

Ahora que hemos visto como se ven los conflictos (con {\bf el ancestro común}), quiero presentarles algunos
términos acerca de las secciones de un conflicto que evitarán algunas ambugüedades futuras.

\subsection{Acerca de las secciones de un conflicto}

\subsubsection{Bloque de Conflicto o CB}
Un {\bf Bloque de Conflicto} o {\bf CB}\footnote{por sus siglas en inglés} comienza con la {\bf Marca de Inicio de Conflicto} y termina con la {\bf Marca de Cierre de Conflicto}.

\subsubsection{Bloque Superior o UB}
El Bloque Superior o {\bf UB} es la sección superior de un {\bf CB}\footnote{por sus siglas en inglés}. Esta sección
{\bf siempre} tendrá el contenido como se presenta en {\bf HEAD}, es decir, la revisión donde estamos trabajando.

\subsubsection{Bloque Medio o MB}
El {\bf Bloque Medio} o {\bf MB}\footnote{por sus siglas en inglés} es la sección media de un {\bf CB}. Esta sección aparecerá
si están usando \hyperref[diff3]{diff3} y, {\bf\it durante una operación de merge}, contendrá lo que normalmente llamamos
{\bf el ancestro común}.

\subsubsection{Bloque Inferior o LB}
El {\bf Bloque Inferior} o {\bf LB}\footnote{por sus siglas en inglés} es la sección inferior de un {\bf CB}. Esta sección
contendrá, {\bf\it durante una operación de merge}, lo que hay en {\bf la otra rama}.

\subsubsection{Términos aplicados en el Ejemplo 3}

Del \hyperref[example_03]{Ejemplo 3}:
\begin{lstlisting}[style=c_style, firstnumber=671, caption=Conflicto del {\bf ejemplo 3}]
	struct bitmap *objects = bitmap_git->result;

<<<<<<< HEAD
	ewah_iterator_init(&it, type_filter);
||||||| d0654dc308
	if (bitmap_git->reuse_objects == bitmap_git->pack->num_objects)
		return;

	ewah_iterator_init(&it, type_filter);
=======
	if (bitmap_git->reuse_objects == bitmap_git->pack->num_objects)
		return;

	init_type_iterator(&it, bitmap_git, object_type);
>>>>>>> 20a5fd881a

	for (i = 0; i < objects->word_alloc &&
\end{lstlisting}

\subsubsection*{CB}
\begin{lstlisting}[style=c_style, firstnumber=673, caption={\bf CB} del {\bf ejemplo 3}]
<<<<<<< HEAD
	ewah_iterator_init(&it, type_filter);
||||||| d0654dc308
	if (bitmap_git->reuse_objects == bitmap_git->pack->num_objects)
		return;

	ewah_iterator_init(&it, type_filter);
=======
	if (bitmap_git->reuse_objects == bitmap_git->pack->num_objects)
		return;

	init_type_iterator(&it, bitmap_git, object_type);
>>>>>>> 20a5fd881a
\end{lstlisting}

\paragraph{UB}
\begin{lstlisting}[style=c_style, firstnumber=674, caption={\bf UB} del {\bf ejemplo 3}]
	ewah_iterator_init(&it, type_filter);
\end{lstlisting}

\paragraph{MB}
\begin{lstlisting}[style=c_style, firstnumber=676, caption=One {\bf MB} del {\bf ejemplo 3}]
	if (bitmap_git->reuse_objects == bitmap_git->pack->num_objects)
		return;

	ewah_iterator_init(&it, type_filter);
\end{lstlisting}

\paragraph{LB}
\begin{lstlisting}[style=c_style, firstnumber=681, caption={\bf LB} del {\bf ejemplo 3}]
	if (bitmap_git->reuse_objects == bitmap_git->pack->num_objects)
		return;

	init_type_iterator(&it, bitmap_git, object_type);
\end{lstlisting}

\subsubsection{Términos aplicados a un conflicto del Ejercicio 3}
Conflict from {\bf path.c} of \hyperref[exercise_03]{Ejercicio 3}:

\begin{lstlisting}[style=c_style, firstnumber=852, caption=Conflicto en {\bf path.c} del {\bf Ejercicio 3}]
	if (is_git_directory(".")) {
<<<<<<< HEAD
		set_git_dir(".", 0);
		check_repository_format();
||||||| 51ebf55b93
		set_git_dir(".");
		check_repository_format();
=======
		set_git_dir(".");
		check_repository_format(NULL);
>>>>>>> 1bdca81641
		return path;
	}
\end{lstlisting}


\paragraph{CB}
\begin{lstlisting}[style=c_style, firstnumber=853, caption={\bf CB} del {\bf Ejercicio 3}]
<<<<<<< HEAD
		set_git_dir(".", 0);
		check_repository_format();
||||||| 51ebf55b93
		set_git_dir(".");
		check_repository_format();
=======
		set_git_dir(".");
		check_repository_format(NULL);
>>>>>>> 1bdca81641
\end{lstlisting}


\paragraph{UB}
\begin{lstlisting}[style=c_style, firstnumber=854, caption={\bf UB} del {\bf Ejercicio 3}]
		set_git_dir(".", 0);
		check_repository_format();
\end{lstlisting}

\subsubsection*{MB}
\begin{lstlisting}[style=c_style, firstnumber=857, caption={\bf MB} del {\bf Ejercicio 3}]
		set_git_dir(".");
		check_repository_format();
\end{lstlisting}

\paragraph{LB}
\begin{lstlisting}[style=c_style, firstnumber=860, caption={\bf LB} del {\bf Ejercicio 3}]
		set_git_dir(".");
		check_repository_format(NULL);
\end{lstlisting}

\subsection{Sobre las diferencias entre ellos}
\subsubsection{dMU}
Los cambios {\bf aparentes} que son aplicados sobre el {\bf MB} para convertirse en el {\bf UB}, sin considerar
revisiones intermedias.

\subsubsection{dML}
Los cambios {\bf aparentes} que son aplicados sobre el {\bf MB} para convertirse en el {\bf LB}, sin considerrar
revisiones intermedias.

\subsubsection{Términos aplicados en el CB del Ejemplo 3}
\paragraph{dMU}
Arrancando desde el {\bf MB}:
\begin{lstlisting}[style=c_style, firstnumber=676, caption={\bf MB} del {\bf Ejemplo 3}]
	if (bitmap_git->reuse_objects == bitmap_git->pack->num_objects)
		return;

	ewah_iterator_init(&it, type_filter);
\end{lstlisting}
Terminando en el {\bf UB}:
\begin{lstlisting}[style=c_style, firstnumber=674, caption={\bf UB} del {\bf Ejemplo 3}]
	ewah_iterator_init(&it, type_filter);
\end{lstlisting}

{\bf dMU}: Remover el condicional que arranca en la línea 676 del {\bf MB}.

\paragraph{dML}
Arrancando en el  {\bf MB}:
\begin{lstlisting}[style=c_style, firstnumber=676, caption={\bf MB} del {\bf Ejemplo 3}]
	if (bitmap_git->reuse_objects == bitmap_git->pack->num_objects)
		return;

	ewah_iterator_init(&it, type_filter);
\end{lstlisting}
Terminando en el {\bf LB}:
\begin{lstlisting}[style=c_style, firstnumber=681, caption={\bf LB} from {\bf example 3}]
	if (bitmap_git->reuse_objects == bitmap_git->pack->num_objects)
		return;

	init_type_iterator(&it, bitmap_git, object_type);
\end{lstlisting}
{\bf dML}: cambiar la llamada a {\bf ewah\_iterator\_init()} en la línea 679 por una llamada
a {\bf init\_type\_iterator()} en la línea 684.

\subsubsection{Términos aplicados a un CB del Ejercicio 3}
\paragraph{dMU}
Arrancando desde el {\bf MB}:
\begin{lstlisting}[style=c_style, firstnumber=857, caption={\bf MB} del {\bf Ejercicio 3}]
		set_git_dir(".");
		check_repository_format();
\end{lstlisting}
Terminando en el  {\bf UB}:
\begin{lstlisting}[style=c_style, firstnumber=854, caption={\bf UB} del {\bf Ejercicio 3}]
		set_git_dir(".", 0);
		check_repository_format();
\end{lstlisting}
{\bf dMU}: Agregar un segundo parámetro a la llamada a {\bf set\_git\_dir()}. El segundo parámetro es {\bf 0}.

\paragraph{dML}
Arrancando desde el {\bf MB}:
\begin{lstlisting}[style=c_style, firstnumber=857, caption={\bf MB} del {\bf Ejercicio 3}]
		set_git_dir(".");
		check_repository_format();
\end{lstlisting}
Terminando en el {\bf LB}:
\begin{lstlisting}[style=c_style, firstnumber=860, caption={\bf LB} del {\bf Ejercicio 3}]
		set_git_dir(".");
		check_repository_format(NULL);
\end{lstlisting}
{\bf dML}: Agregar {\bf NULL} como un parámetro a la llamada a {\bf check\_repository\_format()}.

