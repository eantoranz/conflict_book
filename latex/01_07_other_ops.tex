% copyright 2020 Edmundo Carmona Antoranz
% Released under the terms of Creative Commons Attribution-ShareAlike 4.0 International Public License

\section{cherry-pick, rebase, revert}
\label{other_ops}

Ya han visto como son los {\bf merge conflicts}, los que aparecen cuando se hace una operación de mezcla.
Hay otras 3 operaciones que pueden generar conflictos:

\begin{itemize}
	\item {\bf git cherry-pick}
	\item {\bf git rebase }
	\item {\bf git revert}
\end{itemize}

Aunque la mecánica básica es la misma, cada sección del {\bf CB} tiene un {\bf significado} diferente
\footnote{En relación a de donde viene el código que está en cada sección del {\bf CB}}

\subsection{Example 12 - cherry-pick}
\label{example_12}

En git 2.27, se introdujo este cambio:
\begin{lstlisting}[style=console_style,
	caption={\bf Ejemplo 12} - Cambio de {\bf 3be7efcafceeae34}]
$ git show --pretty= 3be7efcafceeae34
diff --git a/commit-graph.c b/commit-graph.c
index f013a84e29..e4f1a5b2f1 100644
--- a/commit-graph.c
+++ b/commit-graph.c
@@ -23,6 +23,7 @@
 #define GRAPH_CHUNKID_DATA 0x43444154 /* "CDAT" */
 #define GRAPH_CHUNKID_EXTRAEDGES 0x45444745 /* "EDGE" */
 #define GRAPH_CHUNKID_BASE 0x42415345 /* "BASE" */
+#define MAX_NUM_CHUNKS 5
 
 #define GRAPH_DATA_WIDTH (the_hash_algo->rawsz + 16)
 
@@ -1350,8 +1351,8 @@ static int write_commit_graph_file(struct write_commit_graph_context *ctx)
        int fd;
        struct hashfile *f;
        struct lock_file lk = LOCK_INIT;
-       uint32_t chunk_ids[6];
-       uint64_t chunk_offsets[6];
+       uint32_t chunk_ids[MAX_NUM_CHUNKS + 1];
+       uint64_t chunk_offsets[MAX_NUM_CHUNKS + 1];
        const unsigned hashsz = the_hash_algo->rawsz;
        struct strbuf progress_title = STRBUF_INIT;
        int num_chunks = 3;
\end{lstlisting}

Podemos ver la forma en la que se cambia la forma en la que se define el tamaño de {\bf chunk\_ids} y {\bf chunk\_offsets} a usar
una macro.

Como un experimento, queremos hacer un {\it backport} de este cambio sobre {\bf v2.22.4}.

\begin{lstlisting}[style=console_style,
	caption={\bf ejemplo 12} - haciendo {\it backport} de {\bf 3be7efcafceeae34}]
$ git checkout v2.22.4
HEAD is now at c9808fa014 Git 2.22.4
$ git cherry-pick 3be7efcafceeae34
Auto-merging commit-graph.c
CONFLICT (content): Merge conflict in commit-graph.c
error: could not apply 3be7efcafc... commit-graph: define and use MAX_NUM_CHUNKS
hint: after resolving the conflicts, mark the corrected paths
hint: with 'git add <paths>' or 'git rm <paths>'
hint: and commit the result with 'git commit'
$ git status
HEAD detached at v2.22.4
You are currently cherry-picking commit 3be7efcafc.
  (fix conflicts and run "git cherry-pick --continue")
  (use "git cherry-pick --skip" to skip this patch)
  (use "git cherry-pick --abort" to cancel the cherry-pick operation)

Unmerged paths:
  (use "git add <file>..." to mark resolution)
        both modified:   commit-graph.c

no changes added to commit (use "git add" and/or "git commit -a")
\end{lstlisting}

Como se ve este conflicto? Hay 2 {\bf CB}s en {\bf commit-graph.c}. El primero de ellos se ve así:

\subsubsection{CB\#1}
\begin{lstlisting}[style=c_style,
	firstnumber=25,
	caption={\bf Ejemplo 12} - {\bf CB\#1} en {\bf commit-graph.c}]
<<<<<<< HEAD
||||||| parent of 3be7efcafc... commit-graph: define and use MAX_NUM_CHUNKS
#define GRAPH_CHUNKID_BASE 0x42415345 /* "BASE" */
=======
#define GRAPH_CHUNKID_BASE 0x42415345 /* "BASE" */
#define MAX_NUM_CHUNKS 5
>>>>>>> 3be7efcafc... commit-graph: define and use MAX_NUM_CHUNKS
\end{lstlisting}

Se ve muy parecido a los conflictos que hemos visto hasta ahora, cierto? Exactamente como en un {\bf merge}, hay 3 secciones
en el {\bf CB} y seguiremos usando los mismos términos/acrónimos que hemos usado. El primero es el {\bf UB}, en otras palabras,
el código como está en {\bf HEAD}. En este caso, la etiqueta {\bf v2.22.4}, dado que ahí es donde estamos trabajando. A diferencia
de un {\bf merge}, donde se ve el código de {\bf el ancestro común} en el {\bf MB} y de {\bf la otra rama} en el {\bf LB},
durante una operación de {\bf cherry-pick}, los otros dos bloques muestran otra cosa, como indican las marcas de conflicto.
El {\bf MB} nos muestra el contenido de {\bf el papá} de la revisión a la que le estamos haciendo {\bf cherry-pick}
\footnote{La revision {\bf 9fadedd637}, en caso de que tengan curiosidad}. El {\bf LB} nos muestra el código como está en la revisión
{\bf 3be7efcafc}, que es la que estamos tratando de hacerle {\bf cherry-pick}. Aunque el {\bf MB} y el {\bf LB} se refieren
a conceptos diferentes a los de un {\bf merge}, la forma de hacer análisis es {\bf la misma} que hemos utilizado.

\subsubsection{dMU\#1}
La macro {\bf GRAPH\_CHUNKID\_BASE} (presente en la línea 27) fue borrado.

\subsubsection{dML\#1}
La macro {\bf MAX\_NUM\_CHUNKS} fue agregada en la línea 30.

\subsubsection{Resolución\#1}
No veo mucha diferencia entre trabajar desde el {\bf UB} o el {\bf LB}.... simplemente es un poco más
sencillo {\bf borrar} que tener que copiar cambios de un sitio al otro por lo tanto trabajaremos desde
el {\bf LB}:

\begin{lstlisting}[style=c_style,
	firstnumber=28,
	caption={\bf Ejemplo 12} - {\bf LB\#1} en {\bf commit-graph.c}]
=======
#define GRAPH_CHUNKID_BASE 0x42415345 /* "BASE" */
#define MAX_NUM_CHUNKS 5
>>>>>>> 3be7efcafc... commit-graph: define and use MAX_NUM_CHUNKS
\end{lstlisting}

Luego, como lo pide el {\bf dMU}, borramos {\bf GRAPH\_CHUNKID\_BASE}.

\begin{lstlisting}[style=c_style,
	firstnumber=28,
	caption={\bf Ejemplo 12} - {\bf LB\#1} en {\bf commit-graph.c}]
=======
#define MAX_NUM_CHUNKS 5
>>>>>>> 3be7efcafc... commit-graph: define and use MAX_NUM_CHUNKS
\end{lstlisting}

Se remueven las otras secciones del conflicto y los marcadores y terminamos.

\begin{lstlisting}[style=c_style,
	firstnumber=24,
	caption={\bf Ejemplo 12} - Solución del {\bf CB\#1} en {\bf commit-graph.c}]
#define GRAPH_CHUNKID_EXTRAEDGES 0x45444745 /* "EDGE" */
#define MAX_NUM_CHUNKS 5

#define GRAPH_DATA_WIDTH (the_hash_algo->rawsz + 16)
\end{lstlisting}

Luego vamos al {\bf CB\#2} en {\bf commit-graph.c}:
\footnote{Números de línea luego de solucionar el {\bf CB\#1}}
\subsubsection{CB\#2}
\begin{lstlisting}[style=c_style,
	firstnumber=1031,
	caption={\bf Ejemplo 12} - {\bf CB\#2} en {\bf commit-graph.c}]
<<<<<<< HEAD
	uint32_t chunk_ids[5];
	uint64_t chunk_offsets[5];
||||||| parent of 3be7efcafc... commit-graph: define and use MAX_NUM_CHUNKS
	uint32_t chunk_ids[6];
	uint64_t chunk_offsets[6];
=======
	uint32_t chunk_ids[MAX_NUM_CHUNKS + 1];
	uint64_t chunk_offsets[MAX_NUM_CHUNKS + 1];
>>>>>>> 3be7efcafc... commit-graph: define and use MAX_NUM_CHUNKS
\end{lstlisting}

\subsubsection{dMU\#2}
El tamaño de {\bf chunk\_ids} y {\bf chunk\_offsets} fue cambiado de 6 a 5.

\subsubsection{dML\#2}
Descartamos el valor numérico y los reemplazamos por {\bf MAX\_NUM\_CHUNKS + 1}, usando la macro dedinida en el {\bf CB} anterior.

\subsubsection{Resolución\#2}
En este caso especial, podemos quedarnos con lo que tenemos en el {\bf LB} ``tal cual`` dado que el cambio de 6 a 5
no tiene incidencia en el requerimiento: Seguimos usando la macro así que:
\begin{lstlisting}[style=c_style,
	firstnumber=1030,
	caption={\bf Ejemplo 12} - Solución del {\bf CB\#2} en {\bf commit-graph.c}]
	struct lock_file lk = LOCK_INIT;
	uint32_t chunk_ids[MAX_NUM_CHUNKS + 1];
	uint64_t chunk_offsets[MAX_NUM_CHUNKS + 1];
	const unsigned hashsz = the_hash_algo->rawsz;
\end{lstlisting}

Y listo!

... O bueno, eso creen! Recuerdan la {\it intención} de la revisión a la que le estamos haciendo {\bf cherry-pick}?
Se refería a la forma en la que se define el tamaño de los arreglos para que usen una macro. Eso es probablemente para poder
fácilmente cambiar su tamaño luego de ser necesario con solo ajustar el valor de la macro. Bien! Pero no la estamos aplicando
en una revisión que tenía {\bf 6} como tamaño del arreglo. El tamaño del arreglo en {\bf HEAD} es {\bf 5} y ese tamaño debe
permanecer igual. Lo que significa que debemos cambiar el tamaño del macro a {\bf 4}, para que el tamaño de los arreglos sea el
correcto. Así que debemos regresar al {\bf CB\#1} y ajustar el valor de la macro:

\begin{lstlisting}[style=c_style,
	firstnumber=24,
	caption={\bf Ejemplo 12} - Ajuste de la solución del {\bf CB\#1} en {\bf commit-graph.c}]
#define GRAPH_CHUNKID_EXTRAEDGES 0x45444745 /* "EDGE" */
#define MAX_NUM_CHUNKS 4

#define GRAPH_DATA_WIDTH (the_hash_algo->rawsz + 16)
\end{lstlisting}

Y {\bf ahora} sí terminamos! Esa es la forma en la que el {\bf dMU} entra en juego en la resolución del conflicto, aunque no
fue en el {\bf CB} que estábamos trabajando.\footnote{Les dije que siempre se trata de \hyperref[intent]{intención}?}

{\bf Pregunta}: estaría bien si dejamos la macro con el valor de {\bf 5} y cambiamos los arreglos en el {\bf CB\#2} para que
usen {\bf MAX\_NUMS\_CHUNKS}?
De esta forma:

\begin{lstlisting}[style=c_style,
	firstnumber=1030,
	caption={\bf Ejemplo 12} - Solución alternativa al {\bf CB\#2} en {\bf commit-graph.c}]
	struct lock_file lk = LOCK_INIT;
	uint32_t chunk_ids[MAX_NUM_CHUNKS];
	uint64_t chunk_offsets[MAX_NUM_CHUNKS];
	const unsigned hashsz = the_hash_algo->rawsz;
\end{lstlisting}

Desde el punto de vista del código, {\bf podría} estarlo {\bf Y estaría dispuesto a decir que no es correcto, de todos modos... pero
sigan leyendo}. El código compila y se comporta de la forma esperada, cierto? Pero están rompiendo la {\bf intención} de la revisión
a la que le están haciendo {\bf cherry-pick}. Solo para que se imaginen un posible escenario como consecuencia de esto, que sucedería
si, más adelante, necesitaran hacer un {\bf cherry-pick} de otra revisión que use el valor de esta misma macro? Ahora que el valor ha sido
{\it roto}, tendrán que introducir más ajustes sobre el código por sobre la intención de lo que estén haciéndole {\bf cherry-pick}
en ese momento.... y puede ser que ni si quiera tengan conflictos cuando hagan el {\bf cherry-pick}. Se podrán acordar de este cambio
para que logren detectar estas situaciones para arreglar el código? Yo se que sí podría.... pero solo por los siguientes
{\bf 30 segundos} luego de modificar el código, pero no creo que lo pueda recordar más allá de eso.... y se qué es extremadamente
difícil que todos logremos recordar todos estos detalles. Conclusión: Apéguense a la {\bf intención original} del parche.
Considérense advertidos.

\subsection{Ejemplo 13 - revert}

Cuando se quiere revertir cambios que fueron introducidos por una revisión, la operación que se aplica es {\bf revert}
\footnote{No estamos hablando de quitar una revisión e la historia de la rama, lo cual implica hacer un {\bf rewrite} de
la historia (como i la revisión nunca se hubiera aplicado). {\bf Revertir} una revisión mantenrá la historia justo como
está actualmente y se creará una revisión adicional sobre la rama actual para revertir los cambios introducidos por la
revisión que estamos revirtiendo}. Consideremos un pequeño ejemplo de git:

Desde el \hyperref[git_repo]{repo de git}, hagan checkout de la etiqueta {\bf v2.26.2}. Vamos a revertir la revisión {\bf 22a69fda197f}.
Primero veamos lo que hizo la revisión originalmente sobre {\bf builtin/rebase.c}:

\begin{lstlisting}[style=console_style,
	basicstyle=\tiny,
	caption={\bf Ejemplo 13} - Revisión original]
$ git show --pretty="" 22a69fda197f -- builtin/rebase.c
diff --git a/builtin/rebase.c b/builtin/rebase.c
index 8081741f8a..faa4e0d406 100644
--- a/builtin/rebase.c
+++ b/builtin/rebase.c
@@ -453,8 +453,9 @@ int cmd_rebase__interactive(int argc, const char **argv, const char *prefix)
                OPT_NEGBIT(0, "ff", &opts.flags, N_("allow fast-forward"),
                           REBASE_FORCE),
                OPT_BOOL(0, "keep-empty", &opts.keep_empty, N_("keep empty commits")),
-               OPT_BOOL(0, "allow-empty-message", &opts.allow_empty_message,
-                        N_("allow commits with empty messages")),
+               OPT_BOOL_F(0, "allow-empty-message", &opts.allow_empty_message,
+                          N_("allow commits with empty messages"),
+                          PARSE_OPT_HIDDEN),
                OPT_BOOL(0, "rebase-merges", &opts.rebase_merges, N_("rebase merge commits")),
                OPT_BOOL(0, "rebase-cousins", &opts.rebase_cousins,
                         N_("keep original branch points of cousins")),
@@ -1495,9 +1496,10 @@ int cmd_rebase(int argc, const char **argv, const char *prefix)
                OPT_STRING_LIST('x', "exec", &exec, N_("exec"),
                                N_("add exec lines after each commit of the "
                                   "editable list")),
-               OPT_BOOL(0, "allow-empty-message",
-                        &options.allow_empty_message,
-                        N_("allow rebasing commits with empty messages")),
+               OPT_BOOL_F(0, "allow-empty-message",
+                          &options.allow_empty_message,
+                          N_("allow rebasing commits with empty messages"),
+                          PARSE_OPT_HIDDEN),
                {OPTION_STRING, 'r', "rebase-merges", &rebase_merges,
                        N_("mode"),
                        N_("try to rebase merges instead of skipping them"),
\end{lstlisting}

La revisión cambió las llamadas originales a {\bf OPT\_BOOL} por {\bf OPT\_BOOL\_F} para definir {\bf allow-empty-message}, y
se hizo unos ajustes sobre los parámetros de las llamadas. Eso quiere decir que debemos terminar con llamadas a {\bf OPT\_BOOL}
si queremos {\bf revertir}, cierto? Intentémoslo:

\begin{lstlisting}[style=console_style,
	basicstyle=\small,
	caption={\bf Ejemplo 13} - Revirtiendo]
$ git revert --no-commit 22a69fda197f
Auto-merging builtin/rebase.c
CONFLICT (content): Merge conflict in builtin/rebase.c
Auto-merging Documentation/git-rebase.txt
error: could not revert 22a69fda19... git-rebase.txt: update description of --allow-empty-message
hint: after resolving the conflicts, mark the corrected paths
hint: with 'git add <paths>' or 'git rm <paths>'
$ git status
HEAD detached at v2.26.2
You are currently reverting commit 22a69fda19.
  (fix conflicts and run "git revert --continue")
  (use "git revert --skip" to skip this patch)
  (use "git revert --abort" to cancel the revert operation)

Changes to be committed:
  (use "git restore --staged <file>..." to unstage)
        modified:   Documentation/git-rebase.txt

Unmerged paths:
  (use "git restore --staged <file>..." to unstage)
  (use "git add <file>..." to mark resolution)
        both modified:   builtin/rebase.c
\end{lstlisting}

Hay un {\bf CB} en {\bf builtin/rebase.c}:
\begin{lstlisting}[style=c_style,
	basicstyle=\tiny,
	firstnumber=473,
	caption={\bf Ejemplo 13} - {\bf CB} en {\bf builtin/rebase.c}]
<<<<<<< HEAD
		{ OPTION_CALLBACK, 'k', "keep-empty", &options, NULL,
			N_("(DEPRECATED) keep empty commits"),
			PARSE_OPT_NOARG | PARSE_OPT_HIDDEN,
			parse_opt_keep_empty },
		OPT_BOOL_F(0, "allow-empty-message", &opts.allow_empty_message,
			   N_("allow commits with empty messages"),
			   PARSE_OPT_HIDDEN),
||||||| 22a69fda19... git-rebase.txt: update description of --allow-empty-message
		OPT_BOOL(0, "keep-empty", &opts.keep_empty, N_("keep empty commits")),
		OPT_BOOL_F(0, "allow-empty-message", &opts.allow_empty_message,
			   N_("allow commits with empty messages"),
			   PARSE_OPT_HIDDEN),
=======
		OPT_BOOL(0, "keep-empty", &opts.keep_empty, N_("keep empty commits")),
		OPT_BOOL(0, "allow-empty-message", &opts.allow_empty_message,
			 N_("allow commits with empty messages")),
>>>>>>> parent of 22a69fda19... git-rebase.txt: update description of --allow-empty-message
\end{lstlisting}

Esto no es muy diferente de lo que vimos anteriormente en {\bf cherry-pick}, cierto? {\bf UB} es como está el
código en {\bf v2.26.2}, dado que esa es la revisión donde estábamos al comenzar. La parte interesante es en los otros
dos bloques. Se pueder ver por los marcadores que es {\bf MB} es como se ve el código en la revisión que queremos revertir
(en otras palabras, el código {\bf luego} de aplicar la revisión originalmente) y en {\bf UB} está como se ve en el padre
de la revisión que queremos revertir, es decir el orden {\bf opuesto} de cómo se presentam estos dos bloques a cuando se
hace un {\bf cherry-pick}. En {\bf cherry-pick}, el {\bf MB} es el paá de la revisión a la que se le hace {\bf cherry-pick}
y en {\bf LB} está el código luego de aplicar la revisión. Eso significa que, justo como en {\bf merge} y {\bf cherry-pick},
nosotros debemos seguir la misma metodología que hemos aplicado hasta ahora analizando el {\bf dMU} y el {\bf dML}.

\subsubsection{dMU}
La forma en la que se define {\bf keep-empty} se cambia de una llamada a {\bf OPT\_BOOL()} en la línea 482 a una definición
{\bf a mano} en las líneas 474-477.

\subsubsection{dML}
La forma en la que se define {\bf allow-empty-message} se cambia de una llamada a {\bf OPT\_BOOL\_F()} en las líneas 483-485 a
una llamada a {\bf OPT\_BOOL()} en las líneas 488-489, lo cual también remueve {\bf PARSE\_OPT\_HIDDEN} como parámetro de la
llamada.

\subsubsection{Resolución}
Nos quedaremos trabajando en el {\bf UB}, solo porque el cambio de la definición de {\bf keep-empty} es más grande que 
el cambio de la definición de {\bf allow-empty-message}. Luego aplicaremos el cambio de la llamada de la macro {\bf OPT\_BOOL\_F()}
a {\bf OPT\_BOOL()} (líneas 478-480):

\begin{lstlisting}[style=c_style,
	basicstyle=\small,
	firstnumber=473,
	caption={\bf Ejemplo 13} - Paso 1 {\bf UB} en {\bf builtin/rebase.c}]
<<<<<<< HEAD
		{ OPTION_CALLBACK, 'k', "keep-empty", &options, NULL,
			N_("(DEPRECATED) keep empty commits"),
			PARSE_OPT_NOARG | PARSE_OPT_HIDDEN,
			parse_opt_keep_empty },
		OPT_BOOL_F(0, "allow-empty-message", &opts.allow_empty_message,
			   N_("allow commits with empty messages"),
			   PARSE_OPT_HIDDEN),
||||||| 22a69fda19... git-rebase.txt: update description of --allow-empty-message
\end{lstlisting}

Removemos {\bf PARSE\_OPT\_HIDDEN} como parámetro (presente en la línea 480):
\begin{lstlisting}[style=c_style,
	basicstyle=\small,
	firstnumber=473,
	caption={\bf Ejemplo 13} - Paso 2 {\bf UB} en {\bf builtin/rebase.c}]
<<<<<<< HEAD
		{ OPTION_CALLBACK, 'k', "keep-empty", &options, NULL,
			N_("(DEPRECATED) keep empty commits"),
			PARSE_OPT_NOARG | PARSE_OPT_HIDDEN,
			parse_opt_keep_empty },
		OPT_BOOL(0, "allow-empty-message", &opts.allow_empty_message,
			   N_("allow commits with empty messages")),
||||||| 22a69fda19... git-rebase.txt: update description of --allow-empty-message
\end{lstlisting}

Desplazamos la segunda línea de los argmentos para que esté alineada con la posición de la
llamada (línea 479):
\begin{lstlisting}[style=c_style,
	basicstyle=\small,
	firstnumber=473,
	caption={\bf Ejemplo 13} - Paso 3 {\bf UB} en {\bf builtin/rebase.c}]
<<<<<<< HEAD
		{ OPTION_CALLBACK, 'k', "keep-empty", &options, NULL,
			N_("(DEPRECATED) keep empty commits"),
			PARSE_OPT_NOARG | PARSE_OPT_HIDDEN,
			parse_opt_keep_empty },
		OPT_BOOL(0, "allow-empty-message", &opts.allow_empty_message,
			 N_("allow commits with empty messages")),
||||||| 22a69fda19... git-rebase.txt: update description of --allow-empty-message
\end{lstlisting}

Y ya terminamos con el {\bf revert}:

\begin{lstlisting}[style=c_style,
	basicstyle=\tiny,
	firstnumber=470,
	caption={\bf Ejemplo 13} - Solución del {\bf CB} en {\bf builtin/rebase.c}]
	struct option options[] = {
		OPT_NEGBIT(0, "ff", &opts.flags, N_("allow fast-forward"),
			   REBASE_FORCE),
		{ OPTION_CALLBACK, 'k', "keep-empty", &options, NULL,
			N_("(DEPRECATED) keep empty commits"),
			PARSE_OPT_NOARG | PARSE_OPT_HIDDEN,
			parse_opt_keep_empty },
		OPT_BOOL(0, "allow-empty-message", &opts.allow_empty_message,
			 N_("allow commits with empty messages"),
		OPT_BOOL(0, "rebase-merges", &opts.rebase_merges, N_("rebase merge commits")),
\end{lstlisting}

Y podemos ver como terminamos con la llamada a {\bf OPT\_BOOL()} para definir {\bf allow-empty-message}, justo lo que
queríamos para lograr la reversión.

Sólo para que vean que las mismas reglas aplican para hacer el {\bf revert}, intenten hacer la resolución del conflicto pero
trabajando desde el {\bf LB} y aplicando los cambios del {\bf dMU}. Deben terminar con exactamente el mismo código.

\subsection{Ejemplo 14 - rebase}

Juguemos un poco. Del \hyperref[git_repo]{repo de git}, hagan checkout de la revisión {\bf ce9baf234f}. Veamos una sección
del código de {\bf builtin/push.c} en esta revisión:

\begin{lstlisting}[style=c_style,
	basicstyle=\tiny,
	firstnumber=548,
	caption={\bf Ejemplo 14} - Sección de {\bf builtin/push.c} en {\bf ce9baf234f}]
		OPT_BIT('f', "force", &flags, N_("force updates"), TRANSPORT_PUSH_FORCE),
		{ OPTION_CALLBACK,
		  0, CAS_OPT_NAME, &cas, N_("<refname>:<expect>"),
		  N_("require old value of ref to be at this value"),
		  PARSE_OPT_OPTARG | PARSE_OPT_LITERAL_ARGHELP, parseopt_push_cas_option },
		OPT_CALLBACK(0, "recurse-submodules", &recurse_submodules, "(check|on-demand|no)",
			     N_("control recursive pushing of submodules"), option_parse_recurse_submodules),
		OPT_BOOL_F( 0 , "thin", &thin, N_("use thin pack"), PARSE_OPT_NOCOMPLETE),
\end{lstlisting}

Miren cuidadosamente la forma en la que {\bf $<$refname$>$:$<$expect$>$} y {\bf recurse-submodules} están definidas.
{\bf $<$refname$>$:$<$expect$>$} se define {\bf manualmente} en las líneas 549-552 y {\bf recurse-submodules} se define
con una llamada a {\bf OPT\_CALLBACK} en las líneas 553-554.

Pidámosle a git que haga un {\bf rebase} sobre la revisión {\bf 6652716200}\footnote{{\bf ce9baf234f} y {\bf 6652716200}
son los padres de la revisión {\bf b75dc16ae3}. Los comandos para hacer esto son: {\bf git checkout ce9baf234f; git rebase 6652716200}}.
Deben ver un {\bf CB} relativo a la ección de {\bf builtin/push.c} que les mostré previamente:

\begin{lstlisting}[style=c_style,
	basicstyle=\tiny,
	firstnumber=550,
	caption={\bf Ejemplo 14} - {\bf CB} en {\bf builtin/push.c}]
		OPT_BIT('f', "force", &flags, N_("force updates"), TRANSPORT_PUSH_FORCE),
<<<<<<< HEAD
		OPT_CALLBACK_F(0, CAS_OPT_NAME, &cas, N_("<refname>:<expect>"),
			       N_("require old value of ref to be at this value"),
			       PARSE_OPT_OPTARG | PARSE_OPT_LITERAL_ARGHELP, parseopt_push_cas_option),
		{ OPTION_CALLBACK, 0, "recurse-submodules", &recurse_submodules, "(check|on-demand|no)",
			N_("control recursive pushing of submodules"),
			PARSE_OPT_OPTARG, option_parse_recurse_submodules },
||||||| parent of ce9baf234f... push: unset PARSE_OPT_OPTARG for --recurse-submodules
		{ OPTION_CALLBACK,
		  0, CAS_OPT_NAME, &cas, N_("<refname>:<expect>"),
		  N_("require old value of ref to be at this value"),
		  PARSE_OPT_OPTARG | PARSE_OPT_LITERAL_ARGHELP, parseopt_push_cas_option },
		{ OPTION_CALLBACK, 0, "recurse-submodules", &recurse_submodules, "(check|on-demand|no)",
			N_("control recursive pushing of submodules"),
			PARSE_OPT_OPTARG, option_parse_recurse_submodules },
=======
		{ OPTION_CALLBACK,
		  0, CAS_OPT_NAME, &cas, N_("<refname>:<expect>"),
		  N_("require old value of ref to be at this value"),
		  PARSE_OPT_OPTARG | PARSE_OPT_LITERAL_ARGHELP, parseopt_push_cas_option },
		OPT_CALLBACK(0, "recurse-submodules", &recurse_submodules, "(check|on-demand|no)",
			     N_("control recursive pushing of submodules"), option_parse_recurse_submodules),
>>>>>>> ce9baf234f... push: unset PARSE_OPT_OPTARG for --recurse-submodules
		OPT_BOOL_F( 0 , "thin", &thin, N_("use thin pack"), PARSE_OPT_NOCOMPLETE),
\end{lstlisting}

Recuerdan la revisión en la que estábamos cuando corrimos el {\bf rebase}? Era la {\bf ce9baf234f}, cierto? Miren el {\bf UB}.
Se ve como lo que está en {\bf ce9baf234f}? No? Ok. Y el {\bf LB}? Ah, ahí está! Y entonces qué hay en el {\bf UB}?
Recuerdan que les dije que en {\bf UB} siempre está lo que hay en {\bf HEAD}? Pues bien, eso pasa también en este caso.
El truco es darse cuenta de que cuando se hace un {\bf rebase}, primero git tiene que hacer {\bf checkout} de la rama sobre
la que se va a a hacer el {\bf rebase} y entonces se aplican las revisiones a las que se les va a hacer rebase.

Así que, en nuestro ejemplo, en {\bf UB} se vería el código como está en {\bf 6652716200} y en {\bf LB} se vería
el códogo como lo que está en {\bf ce9baf234f}. En {\bf MB} se vería el papá de la revisión que se está tratando de aplicar
en ese momento. En todo caso, siempre se puede ver los detalles de lo que hay en cada sección del {\bf CB} a través de las marcas
de conflicto.

Ahora que hemos aclarado todo el asunto, es el momento de continuar con nuestro análisis como lo hemos hecho hasta ahora, sin cambios:

\subsubsection{dMU}
La forma en la que se define{\bf $<$refname$>$:$<$expect$>$} es cambiada de {\bf a mano} en las líneas 559-562 por una llamada a
{\bf OPT\_CALLBACK\_F()} en las líneas 552-554.

\subsubsection{dML}
La forma en la que se define {\bf recurse-submodules} se cambia de {\it a mano} en las líneas 563-565 a una llamada a
{\bf OPT\_CALLBACK()} en las líneas 571-572.

\subsubsection{Resolución}
No veo mucha diferencia entre trabajar desde el {\bf UB} o desde el {\bf LB}. Comencemos a trabajar desde el {\bf UB}:

\begin{lstlisting}[style=c_style,
	basicstyle=\tiny,
	firstnumber=550,
	caption={\bf Ejemplo 14} - Paso 1 {\bf UB} en {\bf builtin/push.c}]
<<<<<<< HEAD
		OPT_CALLBACK_F(0, CAS_OPT_NAME, &cas, N_("<refname>:<expect>"),
			       N_("require old value of ref to be at this value"),
			       PARSE_OPT_OPTARG | PARSE_OPT_LITERAL_ARGHELP, parseopt_push_cas_option),
		{ OPTION_CALLBACK, 0, "recurse-submodules", &recurse_submodules, "(check|on-demand|no)",
			N_("control recursive pushing of submodules"),
			PARSE_OPT_OPTARG, option_parse_recurse_submodules },
||||||| parent of ce9baf234f... push: unset PARSE_OPT_OPTARG for --recurse-submodules
\end{lstlisting}

Modificamos la forma en la que se define {\bf recurse-submodules} según el {\bf dML}:
\begin{lstlisting}[style=c_style,
	basicstyle=\tiny,
	firstnumber=550,
	caption={\bf Ejemplo 14} - Paso 2 {\bf UB} en {\bf builtin/push.c}]
<<<<<<< HEAD
		OPT_CALLBACK_F(0, CAS_OPT_NAME, &cas, N_("<refname>:<expect>"),
			       N_("require old value of ref to be at this value"),
			       PARSE_OPT_OPTARG | PARSE_OPT_LITERAL_ARGHELP, parseopt_push_cas_option),
		OPT_CALLBACK(0, "recurse-submodules", &recurse_submodules, "(check|on-demand|no)",
			     N_("control recursive pushing of submodules"), option_parse_recurse_submodules),
||||||| parent of ce9baf234f... push: unset PARSE_OPT_OPTARG for --recurse-submodules
\end{lstlisting}

Y listo!
\begin{lstlisting}[style=c_style,
	basicstyle=\tiny,
	firstnumber=550,
	caption={\bf Ejemplo 14} - Solución del {\bf CB} en {\bf builtin/push.c}]
		OPT_BIT('f', "force", &flags, N_("force updates"), TRANSPORT_PUSH_FORCE),
		OPT_CALLBACK_F(0, CAS_OPT_NAME, &cas, N_("<refname>:<expect>"),
			       N_("require old value of ref to be at this value"),
			       PARSE_OPT_OPTARG | PARSE_OPT_LITERAL_ARGHELP, parseopt_push_cas_option),
		OPT_CALLBACK(0, "recurse-submodules", &recurse_submodules, "(check|on-demand|no)",
			     N_("control recursive pushing of submodules"), option_parse_recurse_submodules),
		OPT_BOOL_F( 0 , "thin", &thin, N_("use thin pack"), PARSE_OPT_NOCOMPLETE),
\end{lstlisting}

Agregamos el archivo al índice, no hay nada más pendiente y podemos continuar con el rebase.
\footnote{{\bf git add builtin/push.c; git rebase --continue}}.


% TODO Talk about that even though methodology is the same for all cases, when doing historical analysis, things change
% deleted code? Well... no need to do blame-reverse and so on
