% copyright 2020 Edmundo Carmona Antoranz
% Released under the terms of Creative Commons Attribution-ShareAlike 4.0 International Public License

\section{cherry-pick, rebase, revert}
\label{other_ops}

You have already seen what {\bf merge conflicts} (the ones that come about when we are merging branches) look like.
There are 3 other operations that can produce conflicts:

\begin{itemize}
	\item {\bf git cherry-pick}
	\item {\bf git rebase }
	\item {\bf git revert}
\end{itemize}

Even though the basics are the same, each one of the sections of the {\bf CB} has a different {\bf meaning}
\footnote{In terms of where the code is coming {\bf from}} from what they were during a merge.

\subsection{Example 12 - cherry-pick}
\label{example_12}

In git 2.27, this change was introduced:
\begin{lstlisting}[style=console_style,
	caption={\bf example 12} - change from {\bf 3be7efcafceeae34}]
$ git show --pretty= 3be7efcafceeae34
diff --git a/commit-graph.c b/commit-graph.c
index f013a84e29..e4f1a5b2f1 100644
--- a/commit-graph.c
+++ b/commit-graph.c
@@ -23,6 +23,7 @@
 #define GRAPH_CHUNKID_DATA 0x43444154 /* "CDAT" */
 #define GRAPH_CHUNKID_EXTRAEDGES 0x45444745 /* "EDGE" */
 #define GRAPH_CHUNKID_BASE 0x42415345 /* "BASE" */
+#define MAX_NUM_CHUNKS 5
 
 #define GRAPH_DATA_WIDTH (the_hash_algo->rawsz + 16)
 
@@ -1350,8 +1351,8 @@ static int write_commit_graph_file(struct write_commit_graph_context *ctx)
        int fd;
        struct hashfile *f;
        struct lock_file lk = LOCK_INIT;
-       uint32_t chunk_ids[6];
-       uint64_t chunk_offsets[6];
+       uint32_t chunk_ids[MAX_NUM_CHUNKS + 1];
+       uint64_t chunk_offsets[MAX_NUM_CHUNKS + 1];
        const unsigned hashsz = the_hash_algo->rawsz;
        struct strbuf progress_title = STRBUF_INIT;
        int num_chunks = 3;
\end{lstlisting}

We can see the way that {\bf chunk\_ids} and {\bf chunk\_offsets}'s sizes are changed to be defined by a macro.

As an experiment, we want to {\it backport} this change on top of v2.22.4:

\begin{lstlisting}[style=console_style,
	caption={\bf example 12} - backporting {\bf 3be7efcafceeae34}]
$ git checkout v2.22.4
HEAD is now at c9808fa014 Git 2.22.4
$ git cherry-pick 3be7efcafceeae34
Auto-merging commit-graph.c
CONFLICT (content): Merge conflict in commit-graph.c
error: could not apply 3be7efcafc... commit-graph: define and use MAX_NUM_CHUNKS
hint: after resolving the conflicts, mark the corrected paths
hint: with 'git add <paths>' or 'git rm <paths>'
hint: and commit the result with 'git commit'
$ git status
HEAD detached at v2.22.4
You are currently cherry-picking commit 3be7efcafc.
  (fix conflicts and run "git cherry-pick --continue")
  (use "git cherry-pick --skip" to skip this patch)
  (use "git cherry-pick --abort" to cancel the cherry-pick operation)

Unmerged paths:
  (use "git add <file>..." to mark resolution)
        both modified:   commit-graph.c

no changes added to commit (use "git add" and/or "git commit -a")
\end{lstlisting}

What does the conflict look like? There are two {\bf CB}s in {\bf commit-graph.c}. The first {\bf CB} looks like this:

\subsubsection{CB\#1}
\begin{lstlisting}[style=c_style,
	firstnumber=25,
	caption={\bf example 12} - {\bf CB\#1} in {\bf commit-graph.c}]
<<<<<<< HEAD
||||||| parent of 3be7efcafc... commit-graph: define and use MAX_NUM_CHUNKS
#define GRAPH_CHUNKID_BASE 0x42415345 /* "BASE" */
=======
#define GRAPH_CHUNKID_BASE 0x42415345 /* "BASE" */
#define MAX_NUM_CHUNKS 5
>>>>>>> 3be7efcafc... commit-graph: define and use MAX_NUM_CHUNKS
\end{lstlisting}

It looks a lot like the conflicts we have seen so far, right? Just like in a merge, there are 3 sections in the {\bf CB} and
I will keep on using the same terms/achronyms we have used so far. The first one is {\bf UB}, in other words, code as it is on
{\bf HEAD}, in our case, on tag {\bf v2.22.4}, as that is where we are currently working. Unlike normal merges where
we get to see {\bf the common ancestor}'s code in {\bf MB} and {\bf the other branch}'s code in {\bf LB}, during a
{\bf cherry-pick} operation the other two blocks display something different, as described on the conflict markers.
{\bf MB} shows us the content of {\it the parent} of the revision we are trying to cherry-pick\footnote{revision {\bf 9fadedd637},
in case you are wondering}. {\bf LB} shows us code as it is on the revision {\bf 3be7efcafc}, what we are trying to cherry-pick.
Even though the {\bf MB} and {\bf LB} relate to different concepts from a merge, the way the analysis has to be carried out is
{\bf the same}.

\subsubsection{dMU\#1}
The macro {\bf GRAPH\_CHUNKID\_BASE} (present in line 27) was removed.

\subsubsection{dML\#1}
The macro {\bf MAX\_NUM\_CHUNKS} was added (line 30).

\subsubsection{Resolution\#1}
I don't think there's much difference working from {\bf UB} or {\bf LB}.... it's just a tiny bit simpler to
delete things than copying from another place so I will start working from {\bf LB}:

\begin{lstlisting}[style=c_style,
	firstnumber=28,
	caption={\bf example 12} - {\bf LB\#1} in {\bf commit-graph.c}]
=======
#define GRAPH_CHUNKID_BASE 0x42415345 /* "BASE" */
#define MAX_NUM_CHUNKS 5
>>>>>>> 3be7efcafc... commit-graph: define and use MAX_NUM_CHUNKS
\end{lstlisting}

Then, as {\bf dMU} is requesting, we remove {\bf GRAPH\_CHUNKID\_BASE}.

\begin{lstlisting}[style=c_style,
	firstnumber=28,
	caption={\bf example 12} - {\bf LB\#1} in {\bf commit-graph.c}]
=======
#define MAX_NUM_CHUNKS 5
>>>>>>> 3be7efcafc... commit-graph: define and use MAX_NUM_CHUNKS
\end{lstlisting}

Remove the other sections of the conflict and the markers and we are done.

\begin{lstlisting}[style=c_style,
	firstnumber=24,
	caption={\bf example 12} - Solution of {\bf CB\#1} in {\bf commit-graph.c}]
#define GRAPH_CHUNKID_EXTRAEDGES 0x45444745 /* "EDGE" */
#define MAX_NUM_CHUNKS 5

#define GRAPH_DATA_WIDTH (the_hash_algo->rawsz + 16)
\end{lstlisting}

And then we go into {\bf CB\#2} in {\bf commit-graph.c}:
\footnote{Line numbering after solving {\bf CB\#1}}
\subsubsection{CB\#2}
\begin{lstlisting}[style=c_style,
	firstnumber=1031,
	caption={\bf example 12} - {\bf CB\#2} in {\bf commit-graph.c}]
<<<<<<< HEAD
	uint32_t chunk_ids[5];
	uint64_t chunk_offsets[5];
||||||| parent of 3be7efcafc... commit-graph: define and use MAX_NUM_CHUNKS
	uint32_t chunk_ids[6];
	uint64_t chunk_offsets[6];
=======
	uint32_t chunk_ids[MAX_NUM_CHUNKS + 1];
	uint64_t chunk_offsets[MAX_NUM_CHUNKS + 1];
>>>>>>> 3be7efcafc... commit-graph: define and use MAX_NUM_CHUNKS
\end{lstlisting}

\subsubsection{dMU\#2}
The size of {\bf chunk\_ids} and {\bf chunk\_offsets} was changed from 6 to 5.

\subsubsection{dML\#2}
Got rid of the numeric value and replaced it for {\bf MAX\_NUM\_CHUNKS + 1}, using the macro we defined on the previous {\bf CB}.

\subsubsection{Resolution\#2}
In this special case we can keep code from {\bf LB} as is because the change of {\bf 6} for {\bf 5} has no incidence on the requirement:
We still need to use the macro. So:
\begin{lstlisting}[style=c_style,
	firstnumber=1030,
	caption={\bf example 12} - Solution of {\bf CB\#2} in {\bf commit-graph.c}]
	struct lock_file lk = LOCK_INIT;
	uint32_t chunk_ids[MAX_NUM_CHUNKS + 1];
	uint64_t chunk_offsets[MAX_NUM_CHUNKS + 1];
	const unsigned hashsz = the_hash_algo->rawsz;
\end{lstlisting}

And we are done!
... Or so you think! Do you remember the {\it intent} of the revision we are trying to cherry-pick? It was related to
changing the way we define the size of the arrays to use a macro. This is probably so that if we needed to change the size of the
arrays from 6 to 20 in the future, we would only adjust the value of {\bf MAX\_NUM\_CHUNKS} from {\bf 5} to {\bf 19} and it would hit
both arrays at the same time. Nice! But we are {\bf not} applying it on a revision where the size of the arrays is {\bf 6}. We have
arrays of size {\bf 5} and it has to remain the same.\footnote{The revision you are cherry-picking didn't change the size of the arrays.
In the original revision being cherry-picked, size was {\bf 6} before and it was {\bf 6} after the revision.} Which means that we need to
set up the macro to {\bf 4}, so that the arrays end up with the correct original value when using the macro. Then we need to go back to
the {\bf previous CB} and we adjust the macro value:

\begin{lstlisting}[style=c_style,
	firstnumber=24,
	caption={\bf example 12} - Adjustment of Solution of {\bf CB\#1} in {\bf commit-graph.c}]
#define GRAPH_CHUNKID_EXTRAEDGES 0x45444745 /* "EDGE" */
#define MAX_NUM_CHUNKS 4

#define GRAPH_DATA_WIDTH (the_hash_algo->rawsz + 16)
\end{lstlisting}

And {\bf now} we are done!\footnote{Did I tell you already that it's always about \hyperref[intent]{intent}?}

{\bf Question}: Would it be OK to leave the macro with 5 and change the arrays in {\bf CB\#2} to use {\bf MAX\_NUMS\_CHUNKS} instead?
Like this:

\begin{lstlisting}[style=c_style,
	firstnumber=1030,
	caption={\bf example 12} - Alternative solution to {\bf CB\#2} in {\bf commit-graph.c}]
	struct lock_file lk = LOCK_INIT;
	uint32_t chunk_ids[MAX_NUM_CHUNKS];
	uint64_t chunk_offsets[MAX_NUM_CHUNKS];
	const unsigned hashsz = the_hash_algo->rawsz;
\end{lstlisting}

From the plain coding point if view, it {\bf might} be correct
\footnote{\bf And I'm willing to say it's not correct but anyway, continue reading.}.
Code compiles and behaves the same, right? But you are breaking the {\bf intent} of the revision you are cherry-picking. Just so that you
can glance the consequences of this decision, what would happen if, later on, you needed to cherry-pick another revision that uses this
same macro? Now that you have {\bf broken} the values that you are using for it, you will have to do more adjustments on the code than
the original intent of the other revisions you were cherry-picking then.... and you might not even get conflicts when you cherry-pick
them. Will you be able to keep this change in mind so that you are able to catch those situations later on and correct the code?
I know I would.... for the first 30 seconds after I modified the code, but I don't think I would be able to keep it for much longer.
And I'm sure it's extremely unlikey anyone else will keep this in mind. All in all: You better stick to {\bf the original intent}.
You have been forewarned.

\subsection{Example 13 - revert}
\label{example_13}

When you want to {\it take back} the changes introduced by a revision, the operation you have to apply is {\bf revert}
\footnote{We are not talking about removing a revision from history, which involves {\it rewriting} history (as if the revision
was never committed in the first place). {\bf Reverting} a revision will keep history just the way it is and it will create yet
another revision on top of your current branch that will take back the changes introduced by the revision being reverted.}. Let's
consider a little case from git.

From \hyperref[git_repo]{git repo}, checkout tag {\bf v2.26.2}. We will revert revision {\bf 22a69fda197f}. Let me show you what the
revision that I want to revert {\it did} originally to {\bf builtin/rebase.c}:

\begin{lstlisting}[style=console_style,
	basicstyle=\tiny,
	caption={\bf example 13} - Original revision]
$ git show --pretty="" 22a69fda197f -- builtin/rebase.c
diff --git a/builtin/rebase.c b/builtin/rebase.c
index 8081741f8a..faa4e0d406 100644
--- a/builtin/rebase.c
+++ b/builtin/rebase.c
@@ -453,8 +453,9 @@ int cmd_rebase__interactive(int argc, const char **argv, const char *prefix)
                OPT_NEGBIT(0, "ff", &opts.flags, N_("allow fast-forward"),
                           REBASE_FORCE),
                OPT_BOOL(0, "keep-empty", &opts.keep_empty, N_("keep empty commits")),
-               OPT_BOOL(0, "allow-empty-message", &opts.allow_empty_message,
-                        N_("allow commits with empty messages")),
+               OPT_BOOL_F(0, "allow-empty-message", &opts.allow_empty_message,
+                          N_("allow commits with empty messages"),
+                          PARSE_OPT_HIDDEN),
                OPT_BOOL(0, "rebase-merges", &opts.rebase_merges, N_("rebase merge commits")),
                OPT_BOOL(0, "rebase-cousins", &opts.rebase_cousins,
                         N_("keep original branch points of cousins")),
@@ -1495,9 +1496,10 @@ int cmd_rebase(int argc, const char **argv, const char *prefix)
                OPT_STRING_LIST('x', "exec", &exec, N_("exec"),
                                N_("add exec lines after each commit of the "
                                   "editable list")),
-               OPT_BOOL(0, "allow-empty-message",
-                        &options.allow_empty_message,
-                        N_("allow rebasing commits with empty messages")),
+               OPT_BOOL_F(0, "allow-empty-message",
+                          &options.allow_empty_message,
+                          N_("allow rebasing commits with empty messages"),
+                          PARSE_OPT_HIDDEN),
                {OPTION_STRING, 'r', "rebase-merges", &rebase_merges,
                        N_("mode"),
                        N_("try to rebase merges instead of skipping them"),
\end{lstlisting}

The revision we want to {\bf revert} changed the original calls to {\bf OPT\_BOOL()} for defining {\bf allow-empty-message}
to {\bf OPT\_BOOL\_F()} and adjusted the parameters. That means that if we want to {\bf revert} this revision, we need to end
up with the original {\bf OPT\_BOOL()} call to define {\bf allow-empty-message} in place after the revert is finished, right?
Let's try:

\begin{lstlisting}[style=console_style,
	basicstyle=\tiny,
	caption={\bf example 13} - Reverting]
$ git revert --no-commit 22a69fda197f
Auto-merging builtin/rebase.c
CONFLICT (content): Merge conflict in builtin/rebase.c
Auto-merging Documentation/git-rebase.txt
error: could not revert 22a69fda19... git-rebase.txt: update description of --allow-empty-message
hint: after resolving the conflicts, mark the corrected paths
hint: with 'git add <paths>' or 'git rm <paths>'
$ git status
HEAD detached at v2.26.2
You are currently reverting commit 22a69fda19.
  (fix conflicts and run "git revert --continue")
  (use "git revert --skip" to skip this patch)
  (use "git revert --abort" to cancel the revert operation)

Changes to be committed:
  (use "git restore --staged <file>..." to unstage)
        modified:   Documentation/git-rebase.txt

Unmerged paths:
  (use "git restore --staged <file>..." to unstage)
  (use "git add <file>..." to mark resolution)
        both modified:   builtin/rebase.c
\end{lstlisting}

There is a {\bf CB} in {\bf builtin/rebase.c}:
\begin{lstlisting}[style=c_style,
	basicstyle=\tiny,
	firstnumber=473,
	caption={\bf example 13} - {\bf CB} in {\bf builtin/rebase.c}]
<<<<<<< HEAD
		{ OPTION_CALLBACK, 'k', "keep-empty", &options, NULL,
			N_("(DEPRECATED) keep empty commits"),
			PARSE_OPT_NOARG | PARSE_OPT_HIDDEN,
			parse_opt_keep_empty },
		OPT_BOOL_F(0, "allow-empty-message", &opts.allow_empty_message,
			   N_("allow commits with empty messages"),
			   PARSE_OPT_HIDDEN),
||||||| 22a69fda19... git-rebase.txt: update description of --allow-empty-message
		OPT_BOOL(0, "keep-empty", &opts.keep_empty, N_("keep empty commits")),
		OPT_BOOL_F(0, "allow-empty-message", &opts.allow_empty_message,
			   N_("allow commits with empty messages"),
			   PARSE_OPT_HIDDEN),
=======
		OPT_BOOL(0, "keep-empty", &opts.keep_empty, N_("keep empty commits")),
		OPT_BOOL(0, "allow-empty-message", &opts.allow_empty_message,
			 N_("allow commits with empty messages")),
>>>>>>> parent of 22a69fda19... git-rebase.txt: update description of --allow-empty-message
\end{lstlisting}

This is not too different from what we saw in {\bf cherry-pick}, right? {\bf UB} is the way the code is on {\bf v2.26.2}, as that is
the revision we started working from. The interesting part is in the other two blocks. You can see from the markers that {\bf MB}
is the way code looked on the revision we are trying to revert (in other words, the code the way it looked {\it after} the revision was
applied) and {\bf LB} is the way code looks on its {\it parent} revision (in other words, the way the code looked before)
\footnote{In other words, the exact {\bf opposite} of {\bf cherry-pick}. In cherry-pick, {\bf MB} was the parent of the revision
being cherry-picked and {\bf LB} was code after the revision being cherry-picked was committed. During revert, {\bf MB}
shows code as it is {\bf after} the revision being reverted was committed and {\bf LB} shows code of the parent of the revision
being reverted.}. This means that we, just like in the merges and cherry-picks we have already solved, need to go through {\bf dMU} and
{\bf dML} without any changes in methodology so it's {\it business as usual}.

\subsubsection{dMU}
The way {\bf keep-empty} is defined is changed from a call to {\bf OPT\_BOOL()} in line 482 to a hand-made definition in
lines 474-477.

\subsubsection{dML}
The way {\bf allow-empty-message} is defined is changed from a call to {\bf OPT\_BOOL\_F()} in lines 483-485 to
{\bf OPT\_BOOL()} in lines 488-489, which also removed {\bf PARSE\_OPT\_HIDDEN} as a parameter to the new call.

\subsubsection{Resolution}
I will stay working on {\bf UB}, just because the way {\bf keep-empty} was changed from the {\bf MB} is bigger than the
change that was done for {\bf allow-empty-message} in {\bf dML}. Then I will change the macro call from {\bf OPT\_BOOL\_F()}
to {\bf OPT\_BOOL()} (lines 478-480):

\begin{lstlisting}[style=c_style,
	basicstyle=\tiny,
	firstnumber=473,
	caption={\bf example 13} - Step 1 {\bf UB} in {\bf builtin/rebase.c}]
<<<<<<< HEAD
		{ OPTION_CALLBACK, 'k', "keep-empty", &options, NULL,
			N_("(DEPRECATED) keep empty commits"),
			PARSE_OPT_NOARG | PARSE_OPT_HIDDEN,
			parse_opt_keep_empty },
		OPT_BOOL_F(0, "allow-empty-message", &opts.allow_empty_message,
			   N_("allow commits with empty messages"),
			   PARSE_OPT_HIDDEN),
||||||| 22a69fda19... git-rebase.txt: update description of --allow-empty-message
\end{lstlisting}

Remove {\bf PARSE\_OPT\_HIDDEN} as a parameter (present on line 480):
\begin{lstlisting}[style=c_style,
	basicstyle=\tiny,
	firstnumber=473,
	caption={\bf example 13} - Step 2 {\bf UB} in {\bf builtin/rebase.c}]
<<<<<<< HEAD
		{ OPTION_CALLBACK, 'k', "keep-empty", &options, NULL,
			N_("(DEPRECATED) keep empty commits"),
			PARSE_OPT_NOARG | PARSE_OPT_HIDDEN,
			parse_opt_keep_empty },
		OPT_BOOL(0, "allow-empty-message", &opts.allow_empty_message,
			   N_("allow commits with empty messages")),
||||||| 22a69fda19... git-rebase.txt: update description of --allow-empty-message
\end{lstlisting}

Reposition the second line for the argument on the second line of the macro call to align with
the position of the call (line 479):
\begin{lstlisting}[style=c_style,
	basicstyle=\tiny,
	firstnumber=473,
	caption={\bf example 13} - Step 3 {\bf UB} in {\bf builtin/rebase.c}]
<<<<<<< HEAD
		{ OPTION_CALLBACK, 'k', "keep-empty", &options, NULL,
			N_("(DEPRECATED) keep empty commits"),
			PARSE_OPT_NOARG | PARSE_OPT_HIDDEN,
			parse_opt_keep_empty },
		OPT_BOOL(0, "allow-empty-message", &opts.allow_empty_message,
			 N_("allow commits with empty messages")),
||||||| 22a69fda19... git-rebase.txt: update description of --allow-empty-message
\end{lstlisting}

And now we are done with the reversion:

\begin{lstlisting}[style=c_style,
	basicstyle=\tiny,
	firstnumber=470,
	caption={\bf example 13} - Solution of {\bf CB} in {\bf builtin/rebase.c}]
	struct option options[] = {
		OPT_NEGBIT(0, "ff", &opts.flags, N_("allow fast-forward"),
			   REBASE_FORCE),
		{ OPTION_CALLBACK, 'k', "keep-empty", &options, NULL,
			N_("(DEPRECATED) keep empty commits"),
			PARSE_OPT_NOARG | PARSE_OPT_HIDDEN,
			parse_opt_keep_empty },
		OPT_BOOL(0, "allow-empty-message", &opts.allow_empty_message,
			 N_("allow commits with empty messages"),
		OPT_BOOL(0, "rebase-merges", &opts.rebase_merges, N_("rebase merge commits")),
\end{lstlisting}

And we can see how we ended up with a call to {\bf OPT\_BOOL()} to define {\bf allow-empty-message}, just what we wanted
to get with the {\it reversion}.

Just so that you can see that the same rules apply when analyzing a reversion, try solving the conflict working from {\bf LB}
and bring over the changes introduced in {\bf dMU}. You should end up with the same code.

\subsection{Example 14 - rebase}
\label{example_14}

Let's play a little bit. From \hyperref[git_repo]{git repo}, checkout revision {\bf ce9baf234f}. Let's take a look at a
section of {\bf builtin/push.c} on this revision:

\begin{lstlisting}[style=c_style,
	basicstyle=\tiny,
	firstnumber=548,
	caption={\bf example 14} - Section of {\bf builtin/push.c} in {\bf ce9baf234f}]
		OPT_BIT('f', "force", &flags, N_("force updates"), TRANSPORT_PUSH_FORCE),
		{ OPTION_CALLBACK,
		  0, CAS_OPT_NAME, &cas, N_("<refname>:<expect>"),
		  N_("require old value of ref to be at this value"),
		  PARSE_OPT_OPTARG | PARSE_OPT_LITERAL_ARGHELP, parseopt_push_cas_option },
		OPT_CALLBACK(0, "recurse-submodules", &recurse_submodules, "(check|on-demand|no)",
			     N_("control recursive pushing of submodules"), option_parse_recurse_submodules),
		OPT_BOOL_F( 0 , "thin", &thin, N_("use thin pack"), PARSE_OPT_NOCOMPLETE),
\end{lstlisting}

Take a close look at the way {\bf $<$refname$>$:$<$expect$>$} and {\bf recurse-submodules} are defined. {\bf $<$refname$>$:$<$expect$>$}
is defined using a {\bf struct} by hand in lines 549-552 and {\bf recurse-submodules} is defined with a call to {\bf OPT\_CALLBACK()}
in lines 553-554.

Let's ask git to rebase on top of {\bf 6652716200}\footnote{{\bf ce9baf234f} and {\bf 6652716200} are the parents of revision
{\bf b75dc16ae3}. The commands you can use to do this are: {\bf git checkout ce9baf234f; git rebase 6652716200}}. You should
see a {\bf CB} related to the section of {\bf builtin/push.c} that I showed you before:

\begin{lstlisting}[style=c_style,
	basicstyle=\tiny,
	firstnumber=550,
	caption={\bf example 14} - {\bf CB} in {\bf builtin/push.c}]
		OPT_BIT('f', "force", &flags, N_("force updates"), TRANSPORT_PUSH_FORCE),
<<<<<<< HEAD
		OPT_CALLBACK_F(0, CAS_OPT_NAME, &cas, N_("<refname>:<expect>"),
			       N_("require old value of ref to be at this value"),
			       PARSE_OPT_OPTARG | PARSE_OPT_LITERAL_ARGHELP, parseopt_push_cas_option),
		{ OPTION_CALLBACK, 0, "recurse-submodules", &recurse_submodules, "(check|on-demand|no)",
			N_("control recursive pushing of submodules"),
			PARSE_OPT_OPTARG, option_parse_recurse_submodules },
||||||| parent of ce9baf234f... push: unset PARSE_OPT_OPTARG for --recurse-submodules
		{ OPTION_CALLBACK,
		  0, CAS_OPT_NAME, &cas, N_("<refname>:<expect>"),
		  N_("require old value of ref to be at this value"),
		  PARSE_OPT_OPTARG | PARSE_OPT_LITERAL_ARGHELP, parseopt_push_cas_option },
		{ OPTION_CALLBACK, 0, "recurse-submodules", &recurse_submodules, "(check|on-demand|no)",
			N_("control recursive pushing of submodules"),
			PARSE_OPT_OPTARG, option_parse_recurse_submodules },
=======
		{ OPTION_CALLBACK,
		  0, CAS_OPT_NAME, &cas, N_("<refname>:<expect>"),
		  N_("require old value of ref to be at this value"),
		  PARSE_OPT_OPTARG | PARSE_OPT_LITERAL_ARGHELP, parseopt_push_cas_option },
		OPT_CALLBACK(0, "recurse-submodules", &recurse_submodules, "(check|on-demand|no)",
			     N_("control recursive pushing of submodules"), option_parse_recurse_submodules),
>>>>>>> ce9baf234f... push: unset PARSE_OPT_OPTARG for --recurse-submodules
		OPT_BOOL_F( 0 , "thin", &thin, N_("use thin pack"), PARSE_OPT_NOCOMPLETE),
\end{lstlisting}

You remember the revision you were on when you ran {\bf rebase}? It was on {\bf ce9baf234f}, right? Look at {\bf UB}. Does
it look the way it was in {\bf ce9baf234f}? No? Ok. How about {\bf LB}? Oh, it's there! And then, what is on {\bf UB}? You
remember that I said that {\bf UB} {\bf always} contains code as it is in {\bf HEAD}? Well, that is correct, even in this case.
The trick that you have to realize is that when you run a {\bf rebase} git first checks out the base branch for the rebase and
then starts to apply the revisions that you ask to rebase {\bf on top} of the base branch.

So, in our current example, in {\bf UB} you will see the code as it is in {\bf 6652716200} and in {\bf LB} you will see the
code from {\bf ce9baf234f}. In {\bf MB} you will see the parent of the revision that is being rebased at the time, so in this
case it will hold the parent of {\bf ce9baf234f}. You can see the details about the revision used in {\bf MB} and {\bf LB}
in the conflict markers, just in case.

Now that we have cleared that up, we continue with the analysis as we have done so far, no changes:

\subsubsection{dMU}
The way that {\bf $<$refname$>$:$<$expect$>$} is being defined is from {\it by-hand} in lines 559-562 to a call to {\bf OPT\_CALLBACK\_F()}
in lines 552-554.

\subsubsection{dML}
The way {\bf recurse-submodules} is being defined is changed from {\it by hand} in lines 563-565 to a call to {\bf OPT\_CALLBACK()}
in lines 571-572.

\subsubsection{Resolution}
I don't see much difference between working from {\bf UB } and {\bf LB}. Let's start working from {\bf UB}:

\begin{lstlisting}[style=c_style,
	basicstyle=\tiny,
	firstnumber=550,
	caption={\bf example 14} - Step 1 {\bf UB} in {\bf builtin/push.c}]
<<<<<<< HEAD
		OPT_CALLBACK_F(0, CAS_OPT_NAME, &cas, N_("<refname>:<expect>"),
			       N_("require old value of ref to be at this value"),
			       PARSE_OPT_OPTARG | PARSE_OPT_LITERAL_ARGHELP, parseopt_push_cas_option),
		{ OPTION_CALLBACK, 0, "recurse-submodules", &recurse_submodules, "(check|on-demand|no)",
			N_("control recursive pushing of submodules"),
			PARSE_OPT_OPTARG, option_parse_recurse_submodules },
||||||| parent of ce9baf234f... push: unset PARSE_OPT_OPTARG for --recurse-submodules
\end{lstlisting}

Then we modify the way that {\bf recurse-submodules} is defined as in {\bf dML}:
\begin{lstlisting}[style=c_style,
	basicstyle=\tiny,
	firstnumber=550,
	caption={\bf example 14} - Step 2 {\bf UB} in {\bf builtin/push.c}]
<<<<<<< HEAD
		OPT_CALLBACK_F(0, CAS_OPT_NAME, &cas, N_("<refname>:<expect>"),
			       N_("require old value of ref to be at this value"),
			       PARSE_OPT_OPTARG | PARSE_OPT_LITERAL_ARGHELP, parseopt_push_cas_option),
		OPT_CALLBACK(0, "recurse-submodules", &recurse_submodules, "(check|on-demand|no)",
			     N_("control recursive pushing of submodules"), option_parse_recurse_submodules),
||||||| parent of ce9baf234f... push: unset PARSE_OPT_OPTARG for --recurse-submodules
\end{lstlisting}

And we are done:
\begin{lstlisting}[style=c_style,
	basicstyle=\tiny,
	firstnumber=550,
	caption={\bf example 14} - Solution of {\bf CB} in {\bf builtin/push.c}]
		OPT_BIT('f', "force", &flags, N_("force updates"), TRANSPORT_PUSH_FORCE),
		OPT_CALLBACK_F(0, CAS_OPT_NAME, &cas, N_("<refname>:<expect>"),
			       N_("require old value of ref to be at this value"),
			       PARSE_OPT_OPTARG | PARSE_OPT_LITERAL_ARGHELP, parseopt_push_cas_option),
		OPT_CALLBACK(0, "recurse-submodules", &recurse_submodules, "(check|on-demand|no)",
			     N_("control recursive pushing of submodules"), option_parse_recurse_submodules),
		OPT_BOOL_F( 0 , "thin", &thin, N_("use thin pack"), PARSE_OPT_NOCOMPLETE),
\end{lstlisting}

We add the file to index, nothing else is pending and so we can continue with the rebase
\footnote{{\bf git add builtin/push.c; git rebase --continue}}.

\subsection{Tips}
Regarless of the operation being done (merge, cherry-pick, revert, rebase), always use the same technique to solve
conflicts: {\bf dMU}/{\bf dML}, {\bf Resolution}.


% TODO Talk about that even though methodology is the same for all cases, when doing historical analysis, things change
% deleted code? Well... no need to do blame-reverse and so on
