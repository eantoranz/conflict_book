% copyright 2020 Edmundo Carmona Antoranz
% Released under the terms of Creative Commons Attribution-ShareAlike 4.0 International Public License

\section{El código borrado puede ser un reto}
\label{deleted_code}

Cuando se encuentra casos de código borrado, puede ser tentador asumir que solo debemos borrar código y que no hay más
nada que pensar al respecto. Desafortunadamente, ese no siempre es el caso. En muchas ocasiones el código no fue realmente
borrado sino que se {\it movió}.

\subsection{Ejemplo 6}
\label{example_06}

\begin{lstlisting}[style=python_style,
	basicstyle=\small,
	caption={\bf Ejemplo 6}]
#!/usr/bin/python

import sys

colors = {"black": "black mirror",
          "white": "white noise",
          "blue": "blue sky"}

def getPhrase(origColor):
    color = origColor.lower()
    if color not in colors:
        sys.stderr.write("There is no phrase defined for color %s\n" % origColor)
        sys.exit(1)
    phrase = colors[color]
<<<<<<< HEAD
||||||| 3c7f401
    phrase = "%s: %s" % (origColor, phrase)
=======
    phrase = "Color: %s\nPhrase: %s" % (origColor, phrase)
>>>>>>> example7/branchB
    return phrase

print(getPhrase(sys.argv[1]))
\end{lstlisting}

\subsubsection{dMU}
La línea donde se modifica la frase para incluir el color original (originalmente en la línea 17) fue removida.

\subsubsection{dML}
El formato de la frase fue modificado (desde la línea 17 a la 19).

\subsubsection{Resolución}
Bajo condiciones normales, si la {\it intención} de una rama era borrar una línea, entonces {\bf  probablemente} no
hay punto en una modificación que se hizo de esa línea en otra rama, cierto? Así que:

\begin{lstlisting}[style=python_style,
	basicstyle=\small,
	caption={\bf Ejemplo 6} - resolución]
#!/usr/bin/python

import sys

colors = {"black": "black mirror",
          "white": "white noise",
          "blue": "blue sky"}

def getPhrase(origColor):
    color = origColor.lower()
    if color not in colors:
        sys.stderr.write("There is no phrase defined for color %s\n" % origColor)
        sys.exit(1)
    phrase = colors[color]
    return phrase

print(getPhrase(sys.argv[1]))
\end{lstlisting}

En este caso es bastante obvio, pero con bastante frecuencia se necesita un poquito más de uso de ciclos de máquina
cerebral para poder hacer una resolución {\bf informada}. Tomemos este otro ejemplo:

\subsection{Ejemplo 7}
\label{example_07}

\begin{lstlisting}[style=python_style,
	basicstyle=\small,
	caption={\bf Ejemplo 7}]
#!/usr/bin/python

import sys

colors = {"black": "black mirror",
          "white": "white noise",
          "blue": "blue sky"}

def getPhrase(color):
<<<<<<< HEAD
||||||| f75877d
    if color not in colors:
        sys.stderr.write("Got no phrase for color %s\n" % color)
        sys.exit(1)
=======
    if color not in colors:
        sys.stderr.write("Phrase for color %s is not defined\n" % color)
        sys.exit(1)
>>>>>>> example8/branchB
    phrase = getFormattedPhrase(color)
    return phrase

def getFormattedPhrase(aColor):
    return "%s: %s" % (aColor, colors[aColor])

print(getPhrase(sys.argv[1]))
\end{lstlisting}

\subsubsection{dMU}
El condicional para verificar que el color esté definido (líneas 12-14) fue borrado.

\subsubsection{dML}
El mensaje de error cuando el color no está definido fue modificado (de la línea 13 a la 17).

\subsubsection{Resolución}
Exactamente como en el ejemplo anterior, borramos las líneas en conflicto:

\begin{lstlisting}[style=python_style,
	basicstyle=\small,
	caption={\bf Ejemplo 7} - resolución]
#!/usr/bin/python

import sys

colors = {"black": "black mirror",
          "white":  "white noise",
          "blue": "blue sky"}

def getPhrase(color):
    phrase = getFormattedPhrase(color)
    return phrase

def getFormattedPhrase(aColor):
    return "%s: %s" % (aColor, colors[aColor])

print(getPhrase(sys.argv[1]))
\end{lstlisting}
Vamos {\bf muy bien!} Pero no nececitamos mucha potencia cerebral pare resolverlo, cierto? Donde está el truco?
Tomemos este otro ejemplo:

\subsection{Ejemplo 8}
\label{example_08}

\begin{lstlisting}[style=python_style,
	basicstyle=\small,
	caption={\bf Ejemplo 8}]
#!/usr/bin/python

import sys

colors = {"black": "black mirror",
          "white": "white noise",
          "blue": "blue sky"}

def getPhrase(color):
<<<<<<< HEAD
||||||| 305fd0a
    if color not in colors:
        sys.stderr.write("Got no phrase for color %s\n" % color)
        sys.exit(1)
=======
    if color not in colors:
        sys.stderr.write("Phrase for color %s is not defined\n" % color)
        sys.exit(1)
>>>>>>> example9/branchB
    phrase = getFormattedPhrase(color)
    return phrase

def getFormattedPhrase(aColor):
    if aColor not in colors:
        sys.stderr.write("Got no phrase for color %s\n" % aColor)
        sys.exit(1)
    return "%s: %s" % (aColor, colors[aColor])

print(getPhrase(sys.argv[1]))
\end{lstlisting}

Este se parece mucho al \hyperref[example_07]{ejemplo 7}.

\subsubsection{dMU}
La verificación de que el color esté definido fue borrada.

\subsubsection{dML}
El mensaje cuando el color no está definido fue modificado.

\subsubsection{Resolución}
Deberíamos eliminar las líneas, cierto? Justo como hicimos en el \hyperref{example_07}{ejemplo 7}. Bueno...
{\bf no exactamente}. Si miramos la definición de {\bf getFormattedPhrase()}, en las líneas 23-27, veremos que
tenemos exactamente el mismo mensaje de error del {\bf MB}... excepto por el nombre de la variable del color.
Miren el {\bf MB}:

\begin{lstlisting}[style=python_style,
	basicstyle=\small,
	firstnumber=11,
	caption={\bf Ejemplo 8} - {\bf MB}]
||||||| 305fd0a
    if color not in colors:
        sys.stderr.write("Got no phrase for color %s\n" % color)
        sys.exit(1)
=======
\end{lstlisting}

Miren la definición de {\bf getFormattedPhrase()}, fuera del {\bf CB}:
\begin{lstlisting}[style=python_style,
	basicstyle=\small,
	firstnumber=23,
	caption={\bf Ejemplo 8} - {\bf getFormattedPhrase()}]
def getFormattedPhrase(aColor):
    if aColor not in colors:
        sys.stderr.write("Got no phrase for color %s\n" % aColor)
        sys.exit(1)
    return "%s: %s" % (aColor, colors[aColor])
\end{lstlisting}

Y eso debería {\it sugerirnos}  que el bloque {\bf if} no fue realmente {\bf borrado}. En realidad fue {\bf movido}
como parte de un esfuerzo de {\it refactoring}\footnote{{\bf Refactoring} merece ser tratado de forma separada y estaremos
hablando de ello más adelante}. En este caso lo que debemos hacer es ajustar el código en {\bf getFormattedPhrase()} como
se indica en el {\bf dML}. Así que la resolución correcta del conflicto se vería así:

\begin{lstlisting}[style=python_style,
	basicstyle=\small,
	caption={\bf Ejemplo 8} - resolución]
#!/usr/bin/python

import sys

colors = {"black": "black mirror",
          "white": "white noise",
          "blue": "blue sky"}

def getPhrase(color):
    phrase = getFormattedPhrase(color)
    return phrase

def getFormattedPhrase(aColor):
    if aColor not in colors:
        sys.stderr.write("Phrase for color %s is not defined\n" % aColor)
        sys.exit(1)
    return "%s: %s" % (aColor, colors[aColor])

print(getPhrase(sys.argv[1]))
\end{lstlisting}
{\bf Mucho cuidado con el nombre de la variable!!!}

No piensen que un conflicto está restringido a los {\bf CB}s. Otras partes del código podrían requerir ajustes.
Nada está fuera de los límites para resolver conflictos. También consideren que, en este caso, el archivo era muy pequeño
así que era fácil ver qué había sucedido con el código. Pero consideren cual sería la situación si el archivo tuviera algunos
cientos de líneas y no fuera tan sencillo ver lo que pasó. O incluso un poco más difícil, si la sección fue movida a un
{\bf módulo/archivo diferente}. Entonces sería bastante más difícil ver qué sucedió.... justo como:

\subsection{Ejemplo 9}
\label{example_09}

\begin{lstlisting}[style=python_style,
	basicstyle=\small,
	caption={\bf Ejemplo 9}]
#!/usr/bin/python

from module import getFormattedPhrase
import sys

def getPhrase(color):
<<<<<<< HEAD
||||||| f78479a
    if color not in colors:
        sys.stderr.write("Got no phrase for color %s\n" % color)
        sys.exit(1)
=======
    if color not in colors:
        sys.stderr.write("Color %s has no phrase\n" % color)
        sys.exit(1)
>>>>>>> example10/branchB
    phrase = getFormattedPhrase(color)
    return phrase

print(getPhrase(sys.argv[1]))
\end{lstlisting}

Se parece al \hyperref[example_07]{ejemplo 7} donde solo debimos borrar la sección de código y ser felices. Pero si
miraran dentro de {\bf module.py} verían la misma sección de código que fue movida en el \hyperref[example_08]{ejemplo 8}:

\begin{lstlisting}[style=python_style,
	basicstyle=\small,
	caption={\bf Ejemplo 9} - {\bf module.py}]
import sys

colors = {"black": "black mirror",
          "white": "white noise",
          "blue": "blue sky"}

def getFormattedPhrase(aColor):
    if aColor not in colors:
        sys.stderr.write("Got no phrase for color %s\n" % aColor)
        sys.exit(1)
    return "%s: %s" % (aColor, colors[aColor])
\end{lstlisting}
En este caso resolver el conflicto requeriría ajustar el mensaje de error en {\bf module.py} siguiendo la nueva
frase del {\bf LB} de nuestro {\bf CB} en {\bf ejemplo.py}, ajustando el nombre de la variable, justo como hicimos
en el \hyperref[example_08]{Ejemplo 8}:

\begin{lstlisting}[style=python_style,
	basicstyle=\small,
	caption={\bf Ejemplo 9} - {\bf module.py} final]
import sys

colors = {"black": "black mirror",
          "white": "white noise",
          "blue": "blue sky"}

def getFormattedPhrase(aColor):
    if aColor not in colors:
        sys.stderr.write("Color %s has no phrase\n" % aColor)
        sys.exit(1)
    return "%s: %s" % (aColor, colors[aColor])
\end{lstlisting}

Y removemos el{\bf CB} de {\bf example.py}:

\begin{lstlisting}[style=python_style,
	basicstyle=\small,
	caption={\bf Ejemplo 9} - {\bf example.py} final]
#!/usr/bin/python

from module import getFormattedPhrase
import sys

def getPhrase(color):
    phrase = getFormattedPhrase(color)
    return phrase

print(getPhrase(sys.argv[1]))
\end{lstlisting}

% TODO add a comment that it is possible to do the analysis without guessing. It will be in recipes
\subsection{Cómo podemos evitar... adivinar?}

Primero, no asuman que es un proceso de un solo paso. Cazar {\it cuando} sucedió que se borró o movió código
implica subirse las mangas y ensuciarse de historia. Así que, sin más preambulo, intentemos con un ejemplo real:

\subsection{Ejemplo 10 - un ejemplo de git sobre código borrado}
\label{example_10}

Del \href{git_repo}{repo de git}, hagan un checkout de la revisión {\bf 0da63da794} y mezclen la revisión {\bf f1928f04b2}
\footnote{Estos son los padres de la revisión 56ceb64eb0}. Hay un {\bf CB} en {\bf builtin/grep.c}.

\begin{lstlisting}[style=c_style,
	firstnumber=1123,
	caption={\bf Ejemplo 10} - {\bf CB} en {\bf builtin/grep.c}]
<<<<<<< HEAD
	if (recurse_submodules && untracked)
		die(_("--untracked not supported with --recurse-submodules"));

||||||| d0654dc308
	if (recurse_submodules && (!use_index || untracked))
		die(_("option not supported with --recurse-submodules"));

=======
>>>>>>> f1928f04b2
	if (!show_in_pager && !opt.status_only)
		setup_pager();
\end{lstlisting}

y se puede ver fácilmente que se trata de {\bf código borrado}.

\subsubsection{dMU}
Se modificó el condidional y se modificó el mensaje que se escribe en la llamada a {\bf die()} en la línea 1125.

\subsubsection{dML}
La sección completa fue borrada.

\subsubsection{Resolution}
Deberíamos remover la sección completa {\it así como así}? Tratemos de averiguar {\bf cuando} el bloque fue borrado en
{\bf la otra rama}.

\begin{lstlisting}[style=console_style,
	basicstyle=\small,
	caption={\bf Ejemplo 10} - {\bf git blame --reverse}]
$ git blame -s --reverse d0654dc308..f1928f04b2 -- builtin/grep.c
.
.
.
d7992421e1a 1118)       if (recurse_submodules && (!use_index || untracked))
d7992421e1a 1119)               die(_("option not supported with --recurse-submodules"));
d7992421e1a 1120) 
f1928f04b25 1121)       if (!show_in_pager && !opt.status_only)
f1928f04b25 1122)               setup_pager();
\end{lstlisting}
{\bf git blame --reverse} nos permite ver cual fue la última revisión en la que una línea de código estuvo {\bf presente} en la
historia. La dos revisiones que utilicé fueron {\bf el ancestro común} (provisto en el {\bf CB}), donde sabemos que la línea
estaba {\it presente}, y la rama donde se borró el código que en este caso es {\bf la otra rama}.

Y podemos ver que la última revisión donde las dos líneas a las que les estamos siguiendo la pista están presentes es en 
la {\bf d7992421e1a}, en la dirección hacia {\bf la otra rama}. El siguiente paso sería ver en cual de las revisiones que
le siguen se removieron las líneas:

\begin{lstlisting}[style=console_style,
	caption={\bf Ejemplo 10} - {\bf git log --oneline}]
$ git log --graph --oneline d7992421e1a..f1928f04b2 -- builtin/grep.c
* f1928f04b2 grep: use no. of cores as the default no. of threads
* 70a9fef240 grep: move driver pre-load out of critical section
* 1184a95ea2 grep: re-enable threads in non-worktree case
* 6c307626f1 grep: protect packed_git [re-]initialization
* c441ea4edc grep: allow submodule functions to run in parallel
\end{lstlisting}

Y al mirar este puñado de revisiones\footnote{Con {\bf git show} sobre cada una} nos damos cuenta de que las líneas
fueron {\bf movidas}, no {\bf no borradas}, en {\bf c441ea4edc}:

\begin{lstlisting}[style=console_style,
	basicstyle=\small,
	caption={\bf ejemplo 10} - {\bf git show c441ea4edc}]
$ git show c441ea4edc
.
.
.
diff --git a/builtin/grep.c b/builtin/grep.c
index d3ed05c1da..ac3d86c2e5 100644
--- a/builtin/grep.c
+++ b/builtin/grep.c
.
.
.
@@ -1052,6 +1050,9 @@ int cmd_grep(int argc, const char **argv, const char *prefix)
        pathspec.recursive = 1;
        pathspec.recurse_submodules = !!recurse_submodules;
 
+       if (recurse_submodules && (!use_index || untracked))
+               die(_("option not supported with --recurse-submodules"));
+
        if (list.nr || cached || show_in_pager) {
                if (num_threads > 1)
                        warning(_("invalid option combination, ignoring --threads"));
.
.
.
@@ -1105,9 +1114,6 @@ int cmd_grep(int argc, const char **argv, const char *prefix)
                }
        }
 
-       if (recurse_submodules && (!use_index || untracked))
-               die(_("option not supported with --recurse-submodules"));
-
        if (!show_in_pager && !opt.status_only)
                setup_pager();
 
\end{lstlisting}

Esas líneas todavía están en el {\bf arbol de trabajo}? Sí están, aunque un poco desplazadas:
\begin{lstlisting}[style=c_style,
	firstnumber=1054,
	caption={\bf ejemplo 10} - otra sección de {\bf builtin/grep.c}]
	pathspec.recurse_submodules = !!recurse_submodules;

	if (recurse_submodules && (!use_index || untracked))
		die(_("option not supported with --recurse-submodules"));

	if (show_in_pager) {
\end{lstlisting}

Traigamos los cambios como fueron introducidos en el {\bf UB} y estaremos bien:
\begin{lstlisting}[style=c_style,
	firstnumber=1054,
	caption={\bf example 10} - adjust section of {\bf builtin/grep.c}]
	pathspec.recurse_submodules = !!recurse_submodules;

	if (recurse_submodules && untracked)
		die(_("--untracked not supported with --recurse-submodules"));

	if (show_in_pager) {
\end{lstlisting}


Borramos el {\bf CB} por completo:
\begin{lstlisting}[style=c_style,
	firstnumber=1120,
	caption={\bf example 10} - remover el {\bf CB} en {\bf builtin/grep.c}]
		}
	}

	if (!show_in_pager && !opt.status_only)
		setup_pager();
\end{lstlisting}

Y si comparamos con la revisión {\bf 56ceb64eb0}, no debería haber diferencias significativas.

\subsection{No se impresionen por conflictos grandes}
El código borrado puede producir conflictos muy grades que, al mirar más detenidamente, en realidad no lo son tanto.
Cuando se borra código en una rama y se modifica por lo menos una de esas líneas borradas en otra rama, git no puede
hacer mucho más que mostrar el código original y como se ve el código en las dos puntas que se mezclan, usando {\bf diff3}
como siempre debemos hacer, cierto? Si el bloque era originalmente de 5 líneas al comienzo y termina de 6 al final y en la
otra rama se borran las 5 líneas originales, eso sería 11 líneas más marcadores.... no suena muy grande. Pero si el código
original era de 300 líneas y el código final también es de 300 líneas con una línea modificada y en la otra rama se borran las
300 líneas originales, ahora estamos hablando de 300 * 2 + marcadores. Y todo ello por una línea de código que fue modificada.
Los ejemplos que intentamos previamente sobre código borrado no han sido grandes... así que utilicemos u ejemplo con {\it algunas}
líneas más.

\subsection{Ejemplo 11}

Del \hyperref[git_repo]{repo de git}, hagan checkout de la revisión {\bf d88949d86e} y mezclen {\bf d9f62dfa0d}
\footnote{Estos son los padres de la revisión {\bf 769af0fd9e}}. Miren el {\bf CB} en {\bf ref-filter.c}. El {\bf CB} comienza en la
línea 1680 y termina en la línea 1959 así que no voy a escribir el contenido del {\bf CB} en el artículo debido al tamaño.
Pero igualmente vamos a hacer el análisis y la resolución.

\subsubsection{dMU}
Toda ls sección fue borrada.

\subsubsection{dML}
Les voy a dar 5 minutos para que ustedes lo descubran por ustedes mismos. Luego les diré cual es la diferencia así que,
déjen de leer, miren el códogo por y minutos y vuelvan! {\bf Sin hacer trampas!}

[Pasan 5 minutos]

Entonces... lo pudieron ver? Se cambia {\bf !oidcpm()} por {\bf oideq()}, cierto? Fue difícil verlo? Seguro que lo fue....
en realidad no puedo asegurarlo ya que yo {\bf no} lo hice visualmente. Dejé que git me ayudara:

\begin{lstlisting}[style=c_style,
	firstnumber=1054,
	caption={\bf Ejemplo 11} - {\bf git diff}]
$ git diff HEAD...MERGE_HEAD -- ref-filter.c
diff --git a/ref-filter.c b/ref-filter.c
index 0bccfceff2..ccca317ce1 100644
--- a/ref-filter.c
+++ b/ref-filter.c
@@ -1710,7 +1710,7 @@ struct contains_stack {
 static int in_commit_list(const struct commit_list *want, struct commit *c)
 {
        for (; want; want = want->next)
-               if (!oidcmp(&want->item->object.oid, &c->object.oid))
+               if (oideq(&want->item->object.oid, &c->object.oid))
                        return 1;
        return 0;
 }
\end{lstlisting}
git es una herramienta de analisis fenomenal. Traten de aprender como extraer lo más posible:

De vuelta al conflicto... vieron? Todas esas 279 líneas de conflicto horrible se deben a este pequeño cambio (y el borrado
en el {\bf dMU}, por supuesto). Ahora que sabemos de qué se trataba, pueden resolver este conflicto por sí solos, cierto?
Inténtelo. Aquí está la lista completa de los cambios para que verifiquen al final:

Luego de averiguar que el código en cuestión fue {\bf movido}:
\begin{lstlisting}[style=c_style,
	firstnumber=426,
	caption={\bf Ejemplo 11} - {\bf commit-reach.c}]
static int in_commit_list(const struct commit_list *want, struct commit *c)
{
	for (; want; want = want->next)
		if (oideq(&want->item->object.oid, &c->object.oid))
			return 1;
	return 0;
}
\end{lstlisting}

Removiendo el {\bf CB} de {\bf ref-filter.c}:
\begin{lstlisting}[style=c_style,
	firstnumber=1679,
	caption={\bf Ejemplo 11} - {\bf ref-filter.c}]
/*
 * Return 1 if the refname matches one of the patterns, otherwise 0.
 * A pattern can be a literal prefix (e.g. a refname "refs/heads/master"
 * matches a pattern "refs/heads/mas") or a wildcard (e.g. the same ref
 * matches "refs/heads/mas*", too).
 */
static int match_pattern(const struct ref_filter *filter, const char *refname)
{
\end{lstlisting}

También hay un conflicto en este archivo:
\begin{lstlisting}[style=c_style,
	firstnumber=1763,
	caption={\bf Ejemplo 11} - {\bf builtin/log.c}]
	if (ignore_if_in_upstream) {
		/* Don't say anything if head and upstream are the same. */
		if (rev.pending.nr == 2) {
			struct object_array_entry *o = rev.pending.objects;
			if (oideq(&o[0].item->oid, &o[1].item->oid))
				goto done;
		}
		get_patch_ids(&rev, &ids);
	}
\end{lstlisting}


\subsection{Moraleja}
Piensen en los conflictos relacionados a código borrado como si se tratara de un iceberg. Solo podemos ver la
punta del conflicto, pero podría tratarse de una montaña invertida más grande que el Olimpo 
\footnote{... y me refiero al que está en \href{https://es.wikipedia.org/wiki/Monte_Olimpo_(Marte)}{Marte}, no al que está
en \href{https://es.wikipedia.org/wiki/Olimpo}{Grecia}}

\subsection{Ejercicios}
\subsubsection{Ejercicio 6}
Del \hyperref[git_repo]{repo de git}, hagan checkout de la revisión {\bf 0194c9ad72} y mezclen la revisión {\bf aa46a0da30}.
La solución está \hyperref[exercise_06]{aquí}.
