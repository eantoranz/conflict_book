% copyright 2020 Edmundo Carmona Antoranz
% Released under the terms of Creative Commons Attribution-ShareAlike 4.0 International Public License

\section{Deleted code can be a challenge}
\label{deleted_code}

When you find cases where code has been deleted, it might be tempting to assume that the code was deleted and there's nothing
else to think about. Unfortunately, that is not always the case. On many occasions the code was not actually deleted but
it was {\it moved} instead.

\subsection{Example 6}
\label{example_06}

\begin{lstlisting}[style=python_style,
	basicstyle=\small,
	caption={\bf example 6}]
#!/usr/bin/python

import sys

colors = {"black": "black mirror",
          "white": "white noise",
          "blue": "blue sky"}

def getPhrase(origColor):
    color = origColor.lower()
    if color not in colors:
        sys.stderr.write("There is no phrase defined for color %s\n" % origColor)
        sys.exit(1)
    phrase = colors[color]
<<<<<<< HEAD
||||||| 3c7f401
    phrase = "%s: %s" % (origColor, phrase)
=======
    phrase = "Color: %s\nPhrase: %s" % (origColor, phrase)
>>>>>>> example7/branchB
    return phrase

print(getPhrase(sys.argv[1]))
\end{lstlisting}

\subsubsection{dMU}
The line where the phrase is modified to include the original color in the output (line 17 originally) was removed.

\subsubsection{dML}
Formatting for the phrase was modified (from line 17 to line 19). Under normal conditions, the way a line has been modified on one
branch does not matter {\it if} on the other branch the line has been deleted.

\subsubsection{Resolution}
If the {\it intent} of one branch was to delete it, then there is {\bf probably} no point in a modification made on it, right? So:

\begin{lstlisting}[style=python_style,
	basicstyle=\small,
	caption={\bf example 6} - resolution]
#!/usr/bin/python

import sys

colors = {"black": "black mirror",
          "white": "white noise",
          "blue": "blue sky"}

def getPhrase(origColor):
    color = origColor.lower()
    if color not in colors:
        sys.stderr.write("There is no phrase defined for color %s\n" % origColor)
        sys.exit(1)
    phrase = colors[color]
    return phrase

print(getPhrase(sys.argv[1]))
\end{lstlisting}

In this case it's a clear shot but more often than not analysis actually requires more brain power to get a correct {\it informed}
resolution. Take this sample:

\subsection{Example 7}
\label{example_07}

\begin{lstlisting}[style=python_style,
	basicstyle=\small,
	caption={\bf example 7}]
#!/usr/bin/python

import sys

colors = {"black": "black mirror",
          "white": "white noise",
          "blue": "blue sky"}

def getPhrase(color):
<<<<<<< HEAD
||||||| f75877d
    if color not in colors:
        sys.stderr.write("Got no phrase for color %s\n" % color)
        sys.exit(1)
=======
    if color not in colors:
        sys.stderr.write("Phrase for color %s is not defined\n" % color)
        sys.exit(1)
>>>>>>> example8/branchB
    phrase = getFormattedPhrase(color)
    return phrase

def getFormattedPhrase(aColor):
    return "%s: %s" % (aColor, colors[aColor])

print(getPhrase(sys.argv[1]))
\end{lstlisting}

\subsubsection{dMU}
The conditional to check if the color is defined (defined in lines 12-14) was removed.

\subsubsection{dML}
The error message when the color is not defined was modified (from line 13 to line 17).

\subsubsection{Resolution}
Just like we did in our previous example, we get rid of the conflicting lines:

\begin{lstlisting}[style=python_style,
	basicstyle=\small,
	caption={\bf example 7} - resolution]
#!/usr/bin/python

import sys

colors = {"black": "black mirror",
          "white":  "white noise",
          "blue": "blue sky"}

def getPhrase(color):
    phrase = getFormattedPhrase(color)
    return phrase

def getFormattedPhrase(aColor):
    return "%s: %s" % (aColor, colors[aColor])

print(getPhrase(sys.argv[1]))
\end{lstlisting}
We are doing {\bf great!} But there was not much brain power we needed to spend, right? Where's the trick? Take this next example:

\subsection{Example 8}
\label{example_08}

\begin{lstlisting}[style=python_style,
	basicstyle=\small,
	caption={\bf example 8}]
#!/usr/bin/python

import sys

colors = {"black": "black mirror",
          "white": "white noise",
          "blue": "blue sky"}

def getPhrase(color):
<<<<<<< HEAD
||||||| 305fd0a
    if color not in colors:
        sys.stderr.write("Got no phrase for color %s\n" % color)
        sys.exit(1)
=======
    if color not in colors:
        sys.stderr.write("Phrase for color %s is not defined\n" % color)
        sys.exit(1)
>>>>>>> example9/branchB
    phrase = getFormattedPhrase(color)
    return phrase

def getFormattedPhrase(aColor):
    if aColor not in colors:
        sys.stderr.write("Got no phrase for color %s\n" % aColor)
        sys.exit(1)
    return "%s: %s" % (aColor, colors[aColor])

print(getPhrase(sys.argv[1]))
\end{lstlisting}

This looks very much like \hyperref[example_07]{example 7}.

\subsubsection{dMU}
The check for the color to be defined was removed.

\subsubsection{dML}
The error message when color is not defined was modified we modified.

\subsubsection{Resolution}
We should get rid of the code then, shouldn't we? Just like we did in \hyperref{example_07}{example 7}. Well...
{\bf not exactly}. If you look at the definition of {\bf getFormattedPhrase()}, on lines 23-27 you have the exact same error
message that was on {\bf MB}.... except for the name of the variable for color. Look at {\bf MB}:

\begin{lstlisting}[style=python_style,
	basicstyle=\small,
	firstnumber=11,
	caption={\bf example 8} - {\bf MB}]
||||||| 305fd0a
    if color not in colors:
        sys.stderr.write("Got no phrase for color %s\n" % color)
        sys.exit(1)
=======
\end{lstlisting}

Look at current definition of {\bf getFormattedPhrase()}, outside of the {\bf CB}:
\begin{lstlisting}[style=python_style,
	basicstyle=\small,
	firstnumber=23,
	caption={\bf example 8} - {\bf getFormattedPhrase()}]
def getFormattedPhrase(aColor):
    if aColor not in colors:
        sys.stderr.write("Got no phrase for color %s\n" % aColor)
        sys.exit(1)
    return "%s: %s" % (aColor, colors[aColor])
\end{lstlisting}

And this {\it hints} us that the {\bf if} block is not really {\bf gone}. Actually it was {\bf moved} as part of a {\bf refactoring}
effort \footnote{{\bf Refactoring} deserves to be treated separately so I will be talking about it later}. In this case what we need
to do is adjust the wording of the code in {\bf getFormattedPhrase()} as it was modified in {\bf dML}. So the correct conflict resolution
should look like this:

\begin{lstlisting}[style=python_style,
	basicstyle=\small,
	caption={\bf example 8} - resolution]
#!/usr/bin/python

import sys

colors = {"black": "black mirror",
          "white": "white noise",
          "blue": "blue sky"}

def getPhrase(color):
    phrase = getFormattedPhrase(color)
    return phrase

def getFormattedPhrase(aColor):
    if aColor not in colors:
        sys.stderr.write("Phrase for color %s is not defined\n" % aColor)
        sys.exit(1)
    return "%s: %s" % (aColor, colors[aColor])

print(getPhrase(sys.argv[1]))
\end{lstlisting}
{\bf Careful with the name of the variable!!!}

Don't think that a conflict is restricted to the conflicting section of code. Other parts of the code might need adjustments.
Nothing is off-limits to solve conflicts. Also consider that in this case it was a pretty small file where it is easy to spot
what happened with the code. But consider what the situation would be like if the file were a few hundred lines long and you are
not able to easily see that the lines were moved to a different place. Or, even more difficult, if the function were defined in
{\it a separate module/file}. Then you might have a much harder time figuring out what happened and what to do..... just like:

\subsection{Example 9}
\label{example_09}

\begin{lstlisting}[style=python_style,
	basicstyle=\small,
	caption={\bf example 9}]
#!/usr/bin/python

from module import getFormattedPhrase
import sys

def getPhrase(color):
<<<<<<< HEAD
||||||| f78479a
    if color not in colors:
        sys.stderr.write("Got no phrase for color %s\n" % color)
        sys.exit(1)
=======
    if color not in colors:
        sys.stderr.write("Color %s has no phrase\n" % color)
        sys.exit(1)
>>>>>>> example10/branchB
    phrase = getFormattedPhrase(color)
    return phrase

print(getPhrase(sys.argv[1]))
\end{lstlisting}

This looks like \hyperref[example_07]{example 7} where we only needed to delete the code and be happy with it.
But if you looked inside {\bf module.py}, you would see the same block of code that we saw moving on
\hyperref[example_08]{example 8}:

\begin{lstlisting}[style=python_style,
	basicstyle=\small,
	caption={\bf example 9} - {\bf module.py}]
import sys

colors = {"black": "black mirror",
          "white": "white noise",
          "blue": "blue sky"}

def getFormattedPhrase(aColor):
    if aColor not in colors:
        sys.stderr.write("Got no phrase for color %s\n" % aColor)
        sys.exit(1)
    return "%s: %s" % (aColor, colors[aColor])
\end{lstlisting}
In this case resolving the conflict would require adjusting the error message in {\bf module.py} following the new
phrasing of {\bf LB} from our {\bf CB} from {\bf example.py}, adjusting variable name, just like we did in
\hyperref[example_08]{example 8}:

\begin{lstlisting}[style=python_style,
	basicstyle=\small,
	caption={\bf example 9} - {\bf module.py} final]
import sys

colors = {"black": "black mirror",
          "white": "white noise",
          "blue": "blue sky"}

def getFormattedPhrase(aColor):
    if aColor not in colors:
        sys.stderr.write("Color %s has no phrase\n" % aColor)
        sys.exit(1)
    return "%s: %s" % (aColor, colors[aColor])
\end{lstlisting}

And removing the {\bf CB} from {\bf example.py}:

\begin{lstlisting}[style=python_style,
	basicstyle=\small,
	caption={\bf example 9} - {\bf example.py} final]
#!/usr/bin/python

from module import getFormattedPhrase
import sys

def getPhrase(color):
    phrase = getFormattedPhrase(color)
    return phrase

print(getPhrase(sys.argv[1]))
\end{lstlisting}

% TODO add a comment that it is possible to do the analysis without guessing. It will be in recipes
\subsection{How can we avoid... guessing?}

First, do not assume that this is a 1-step process. Hunting down for {\it when} a deletion happened to make
sure if code was moved or not normally involves pulling up your sleeves and getting dirty in history. So, without
further ado, let's try with a real project:

\subsection{Example 10 - a git example for deleted code}
\label{example_10}

From \href{git_repo}{git repo}, let's checkout revision {\bf 0da63da794} and merge revision {\bf f1928f04b2}
\footnote{These are the parents of revision 56ceb64eb0}. There will be a {\bf CB} in {\bf builtin/grep.c}.

\begin{lstlisting}[style=c_style,
	firstnumber=1123,
	caption={\bf example 10} - {\bf cb} in {\bf builtin/grep.c}]
<<<<<<< HEAD
	if (recurse_submodules && untracked)
		die(_("--untracked not supported with --recurse-submodules"));

||||||| d0654dc308
	if (recurse_submodules && (!use_index || untracked))
		die(_("option not supported with --recurse-submodules"));

=======
>>>>>>> f1928f04b2
	if (!show_in_pager && !opt.status_only)
		setup_pager();
\end{lstlisting}

And you can see that it involves {\bf deleted code}.

\subsubsection{dMU}
Modified the conditional and modified the wording of the message printed on {\bf die()} on line 1125.

\subsubsection{dML}
The whole block wad deleted.

\subsubsection{Resolution}
Should we remove the block {\it just like that}? Let's try to fund out {\bf when} the block was deleted on {\bf the other branch}.

\begin{lstlisting}[style=console_style,
	basicstyle=\small,
	caption={\bf example 10} - {\bf git blame --reverse}]
$ git blame -s --reverse d0654dc308..f1928f04b2 -- builtin/grep.c
.
.
.
d7992421e1a 1118)       if (recurse_submodules && (!use_index || untracked))
d7992421e1a 1119)               die(_("option not supported with --recurse-submodules"));
d7992421e1a 1120) 
f1928f04b25 1121)       if (!show_in_pager && !opt.status_only)
f1928f04b25 1122)               setup_pager();
\end{lstlisting}
{\bf git blame --reverse} allows us to see when was the last revision when code was {\bf present} in the history. The two
revisions that I used were the {\bf common ancestor} (provided in the {\bf CB}), where we know that the line was still
{\it present},  and the revision we are trying to merge aka {\bf the other branch}.

And we can see that the last revision when the two lines we are tracking were {\bf present} is in {\bf d7992421e1a},
in the direction of {\bf the other branch}. Next step would see what revision followed it to know in what revision it
was {\bf actually deleted}:

\begin{lstlisting}[style=console_style,
	caption={\bf example 10} - {\bf git log --oneline}]
$ git log --graph --oneline d7992421e1a..f1928f04b2 -- builtin/grep.c
* f1928f04b2 grep: use no. of cores as the default no. of threads
* 70a9fef240 grep: move driver pre-load out of critical section
* 1184a95ea2 grep: re-enable threads in non-worktree case
* 6c307626f1 grep: protect packed_git [re-]initialization
* c441ea4edc grep: allow submodule functions to run in parallel
\end{lstlisting}

And by checking this handful of revisions, we realize that the lines were {\bf moved}, not {\it really} deleted,
in {\bf c441ea4edc}:

\begin{lstlisting}[style=console_style,
	basicstyle=\small,
	caption={\bf example 10} - {\bf git show c441ea4edc}]
$ git show c441ea4edc
.
.
.
diff --git a/builtin/grep.c b/builtin/grep.c
index d3ed05c1da..ac3d86c2e5 100644
--- a/builtin/grep.c
+++ b/builtin/grep.c
.
.
.
@@ -1052,6 +1050,9 @@ int cmd_grep(int argc, const char **argv, const char *prefix)
        pathspec.recursive = 1;
        pathspec.recurse_submodules = !!recurse_submodules;
 
+       if (recurse_submodules && (!use_index || untracked))
+               die(_("option not supported with --recurse-submodules"));
+
        if (list.nr || cached || show_in_pager) {
                if (num_threads > 1)
                        warning(_("invalid option combination, ignoring --threads"));
.
.
.
@@ -1105,9 +1114,6 @@ int cmd_grep(int argc, const char **argv, const char *prefix)
                }
        }
 
-       if (recurse_submodules && (!use_index || untracked))
-               die(_("option not supported with --recurse-submodules"));
-
        if (!show_in_pager && !opt.status_only)
                setup_pager();
 
\end{lstlisting}

Are the lines there on the conflicted file currently on the working tree? They are, though a little displaced:
\begin{lstlisting}[style=c_style,
	firstnumber=1054,
	caption={\bf example 10} - another section of {\bf builtin/grep.c}]
	pathspec.recurse_submodules = !!recurse_submodules;

	if (recurse_submodules && (!use_index || untracked))
		die(_("option not supported with --recurse-submodules"));

	if (show_in_pager) {
\end{lstlisting}

Let's bring over the changes as they were introduced in {\bf UB} and we should be fine:
\begin{lstlisting}[style=c_style,
	firstnumber=1054,
	caption={\bf example 10} - adjust section of {\bf builtin/grep.c}]
	pathspec.recurse_submodules = !!recurse_submodules;

	if (recurse_submodules && (!use_index || untracked))
		die(_("--untracked not supported with --recurse-submodules"));

	if (show_in_pager) {
\end{lstlisting}

Get rid of the {\bf CB} altogether:
\begin{lstlisting}[style=c_style,
	firstnumber=1120,
	caption={\bf example 10} - remove {\bf CB} in {\bf builtin/grep.c}]
		}
	}

	if (!show_in_pager && !opt.status_only)
		setup_pager();
\end{lstlisting}

And if we compare with revision {\bf 56ceb64eb0}, there should be no differences.

\subsection{Bottom line}
Think of conflicts related to deleted code as an iceberg. You only get to see the tip of the conflict, but it can be an
inverted mountain of ice beneath the surface bigger than Olympus.
\footnote{... and I mean the one in \href{https://en.wikipedia.org/wiki/Olympus_Mons}{Mars}, not the one in
\href{https://en.wikipedia.org/wiki/Mount_Olympus}{Greece}}

\subsection{Exercises}
\subsubsection{Exercise 6}
From \hyperref[git_repo]{git repo}, checkout revision {\bf 0194c9ad72} and merge revision {\bf aa46a0da30}. Solution is
\hyperref[exercise_06]{here}.

