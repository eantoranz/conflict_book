% copyright 2020 Edmundo Carmona Antoranz
% Released under the terms of Creative Commons Attribution-ShareAlike 4.0 International Public License

\section{Convenciones}

Se utilizará código de diferentes fuentes. Algunos ejemplos serán inventados (y por lo tanto no hay garantía de que
el código tenga sentido... o de que si quiera funcione) y algunos otros serán de proyectos {\it de verdad}.

Se utilizará git como el sistema de control de versiones para los ejemplos. La versión 2.26.2 de paquetes de
{\it debian testing} al momento de escribir esta línea.\footnote{Algunos ejemplos están escritos con versiones anteriores
y la salida de los comandos puede que no sea consistente en las operaciones. Eso se debe a que operaciones como rebase están
recibiendo grandes cambios en git. Pidos disculpas por cualquier inconveniente o inconsistencia.}

Cuando muestre contenido de archivos, se podrá ver el número de la linea del archivo a la izquierda. Cuando se ejecuten comandos
y se muestre su salida, no habrá números de líneas, a menos que se especifique lo contrario.

Cuando se ejecuten comandos se verá el \$ habitual que se muestra {\it normalmente} en los terminales\footnote{ Diferentes personas
podrían tener cosas diferentes configuradas en su {\bf prompt}} seguido por el comando ejecutado. En algunas ocasiones se mostrará ña
salida de más de un comando. En esas ocasiones, cada comando irá precedido por su correspondiente signo de \$.

No esperen ver muchas fotografías en el libro. Es un libro de texto en la mayor parte. Cuando analizo los conflictos, se muestran
en texto plano y no estaré usando ningún IDE para ayudarme a resolver el conflicto.
