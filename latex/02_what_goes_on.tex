% copyright 2020 Edmundo Carmona Antoranz
% Released under the terms of Creative Commons Attribution-ShareAlike 4.0 International Public License

Let's dive a little bit into the things that take place when there is a content conflict. In order to understand this section
the best, you should have a {\bf crystal-clear} understanding of what these concepts are in git {\bf specifically}:

\begin{itemize}
	\item {\bf revision}\footnote{If you are not able to explain what {\bf a branch} is in git in just 5 words
	(if you don't count the ``A branch in git is...'' opener... and that's right, only 5 words will suffice),
	then you probably should dig a little to see what a revision is and how it differs from {\bf a branch}.}
	\item {\bf staging area}, aka {\bf index}\footnote{Some people think they are different things but they are {\bf not}.
	It's two different ways to call {\it the same thing}.}
\end{itemize}

If you are not clear on what they are, then stop right now and come back when you are. I promise I won't move forward with
the explanations until you come back. {\bf Pinky promise!}

\section{No Conflict}
First, let's see how git works on a normal revision, before we check how it behaves on conflicts when merging. Let's work
with branches from \hyperref[example_09]{example 09} (before we did the merges). Right after checking out {\bf example9/branchA},
we run {\bf git status}:

\begin{lstlisting}[style=console_style,
	basicstyle=\small,
	caption={\bf git status} on clean working tree]
$ git status
On branch example9/branchA
nothing to commit, working tree clean
\end{lstlisting}

How does git know that there are {\bf no changes} in my working tree? Git compares {\bf the working tree} with {\bf the index}:

\begin{lstlisting}[style=console_style,
	basicstyle=\small,
	caption={\bf git ls-files} on clean working tree]
$ git ls-files -s
100644 2f78cf5b66514f2506d9af5f3dadf3dee7aa6d9f 0       .gitignore
100755 ad957360597fb9bd3e83d4bfa869f6e19b7fbf2b 0       example.py
100644 475f980e6c6e24f8fc4a144e498fa1c1c59da370 0       module.py
\end{lstlisting}

git is listing permissions for the files, object ID, index {\bf stage}\footnote{More on what this number means later.} and the
file name for each one of the files in the current revision. git compares the files in the working tree and sees that they haven't
changed and so it reports no changes. 
% TODO make sure this is correct: It doesn't actually check the content of every file when we run git status, as that would be too time-consuming..

Next thing, let's modify {bf example.py} somehow. After modifying and saving, git realizes that the file has changed because
it does not match what is on {\bf index} anymore. {\bf Index} information won't change until we add the file, of course. So let's
add the file in the {\bf index} and check what {\bf git ls-files} has for us then:

\begin{lstlisting}[style=console_style,
	basicstyle=\small,
	caption={\bf git ls-files} on a modified file]
$ git add example.py
$ git status
On branch example9/branchA
Changes to be committed:
  (use "git restore --staged <file>..." to unstage)
        modified:   example.py

$ git ls-files -s
100644 2f78cf5b66514f2506d9af5f3dadf3dee7aa6d9f 0       .gitignore
100755 b30ade699b80860abd0ed6d84cff9df94e801d34 0       example.py
100644 475f980e6c6e24f8fc4a144e498fa1c1c59da370 0       module.py
\end{lstlisting}

We can see that the id of the object for {\bf example.py} has changed\footnote{Do not worry if the id of the object for {\bf example.py}
in {\bf index} is different for you from this. This id depends on file content and it's more than likely that the modification you did
won't be the same that I did for this example.} but other than that, there's no other difference. How can git know what the ids of
those objects in {\bf index} were before adding them? git is also checking the information about {\bf HEAD}:

\begin{lstlisting}[style=console_style,
	basicstyle=\small,
	caption={\bf git ls-tree} on {\bf example9/branchA}]
$ git ls-tree -r HEAD
100644 blob 2f78cf5b66514f2506d9af5f3dadf3dee7aa6d9f    .gitignore
100755 blob ad957360597fb9bd3e83d4bfa869f6e19b7fbf2b    example.py
100644 blob 475f980e6c6e24f8fc4a144e498fa1c1c59da370    module.py
\end{lstlisting}

And we can see how the ids of the 3 files were the same as the ids that were in the index when we checked out {\bf example9/branchA}.
Now that the id of {\bf example.py} has changed between {\bf HEAD} and what's on the index, git knows that there's a new content associated for {\bf example.py} in {\bf index} that is ready to be committed. That's why it's showing up in {\bf Changes to be committed} now.

Let's modify the file yet again {\it without adding it} and let's see what happens:

\begin{lstlisting}[style=console_style,
	basicstyle=\small,
	caption=After modifying staged file]
$ git status
On branch example9/branchA
Changes to be committed:
  (use "git restore --staged <file>..." to unstage)
        modified:   example.py

Changes not staged for commit:
  (use "git add <file>..." to update what will be committed)
  (use "git restore <file>..." to discard changes in working directory)
        modified:   example.py

$ git ls-tree -r HEAD
100644 blob 2f78cf5b66514f2506d9af5f3dadf3dee7aa6d9f    .gitignore
100755 blob ad957360597fb9bd3e83d4bfa869f6e19b7fbf2b    example.py
100644 blob 475f980e6c6e24f8fc4a144e498fa1c1c59da370    module.py
$ git ls-files -s
100644 2f78cf5b66514f2506d9af5f3dadf3dee7aa6d9f 0       .gitignore
100755 b30ade699b80860abd0ed6d84cff9df94e801d34 0       example.py
100644 475f980e6c6e24f8fc4a144e498fa1c1c59da370 0       module.py
\end{lstlisting}

This is interesting. {\bf example.py} is showing up in {\bf Changes to be committed} because the id for the file has changed between
{\bf HEAD} and {\bf index}. But it's also showing up in {\bf Changes not staged for commit} because git notices that the content of
the file on {\bf the working tree} is not matching the object on {\bf index}. If we added it now and we checked again:

\begin{lstlisting}[style=console_style,
	basicstyle=\small,
	caption=After adding file again]
$ git add example.py 
$ git status
On branch example9/branchA
Changes to be committed:
  (use "git restore --staged <file>..." to unstage)
        modified:   example.py

$ git ls-tree -r HEAD
100644 blob 2f78cf5b66514f2506d9af5f3dadf3dee7aa6d9f    .gitignore
100755 blob ad957360597fb9bd3e83d4bfa869f6e19b7fbf2b    example.py
100644 blob 475f980e6c6e24f8fc4a144e498fa1c1c59da370    module.py
$ git ls-files -s
100644 2f78cf5b66514f2506d9af5f3dadf3dee7aa6d9f 0       .gitignore
100755 1858db13464ae33329a90cacbe294e749ea9a1a5 0       example.py
100644 475f980e6c6e24f8fc4a144e498fa1c1c59da370 0       module.py
\end{lstlisting}

Notice how the id of the object associated to {\bf example.py} in {\bf index} has changed now (from
{\bf b30ade699b8....} to {\bf 1858db13464....}). The ids of the objects are still
different between {\bf index} and {\bf HEAD }and that's why the file is showing up in {\bf Changes to be committed}. The file content now
matches what is on {\bf index} and that's why git doesn't need to say anything else about the file. If we created a revision at this moment,
what would be the result?

\begin{lstlisting}[style=console_style,
	basicstyle=\small,
	caption=committing]
$ git commit -m "Some test"
[example9/branchA b2f9a40] Some test
 1 file changed, 6 insertions(+)
$ git status
On branch example9/branchA
nothing to commit, working tree clean
$ git ls-tree -r HEAD
100644 blob 2f78cf5b66514f2506d9af5f3dadf3dee7aa6d9f    .gitignore
100755 blob 1858db13464ae33329a90cacbe294e749ea9a1a5    example.py
100644 blob 475f980e6c6e24f8fc4a144e498fa1c1c59da370    module.py
$ git ls-files -s
100644 2f78cf5b66514f2506d9af5f3dadf3dee7aa6d9f 0       .gitignore
100755 1858db13464ae33329a90cacbe294e749ea9a1a5 0       example.py
100644 475f980e6c6e24f8fc4a144e498fa1c1c59da370 0       module.py
\end{lstlisting}

Notice how the id of {\bf example.py} has changed on {\bf HEAD}\footnote{Now on a different revision} to what was on index before we
committed ({\bf 1858db13464ae33329a90cacbe294e749ea9a1a5}). Now the ids of the files is the same between {\bf HEAD} and {\bf index}
and the file in {\bf the working tree} matches what's on {\bf index}, so git says there's nothing to report in git status.

Let's reset the branch before we play with conflicts:

\begin{lstlisting}[style=console_style,
	basicstyle=\small,
	caption=Resetting]
$ git reset --hard HEAD~
HEAD is now at dfde76c A little refactor (moving condition to another function)
\end{lstlisting}

\section{Some conflict}
Let's see what happens when there are conflicts. Checkout {\bf example9/branchB} and merge {\bf example9/branchA}:

\begin{lstlisting}[style=console_style,
	basicstyle=\small,
	caption=Merging]
$ git merge example9/branchA
Auto-merging example.py
CONFLICT (content): Merge conflict in example.py
Automatic merge failed; fix conflicts and then commit the result.
$ git status
On branch example9/branchB
You have unmerged paths.
  (fix conflicts and run "git commit")
  (use "git merge --abort" to abort the merge)

Changes to be committed:
        modified:   module.py

Unmerged paths:
  (use "git add <file>..." to mark resolution)
        both modified:   example.py

$ git ls-tree -r HEAD
100644 blob 2f78cf5b66514f2506d9af5f3dadf3dee7aa6d9f    .gitignore
100755 blob 500616dc70f4847f244d29d827a192b7fa03de93    example.py
100644 blob 2d1fe2ea52267ab6e75cca853c393fc6929a0e45    module.py
$ git ls-files -s
100644 2f78cf5b66514f2506d9af5f3dadf3dee7aa6d9f 0       .gitignore
100755 b0a7eae10629dec61246f86c08f2432f0e276675 1       example.py
100755 500616dc70f4847f244d29d827a192b7fa03de93 2       example.py
100755 ad957360597fb9bd3e83d4bfa869f6e19b7fbf2b 3       example.py
100644 475f980e6c6e24f8fc4a144e498fa1c1c59da370 0       module.py
\end{lstlisting}

We can see that {\bf example.py} is showing up in {\bf Unmerged paths}, and {\bf module.py} is listed in {\bf Changes to be committed}.
{\bf .gitignore} is not listed. {\bf .gitignore} hasn't changed between {\bf HEAD}, {\bf index} and what we have in {\bf the working tree}
so nothing to report on it. {\bf module.py} has changed as part of the merge process {\it without any conflicts} and that's why it is present
in {\bf HEAD} and in {\bf index} with {\it different} ids. The file currently on {\bf the working tree} is just like it is on {\bf index}
and that's why the file is listed in {\bf Changes to be committed}. The interesting stuff is on {\bf example.py}. Instead of having
a single item for it in {\bf index}, there are 3 and each one of them has a different {\bf index stage number}. When there is {\bf content
conflict} on a file (like in this case), git will hold 3 versions of the file in {\bf index}. Stage number 1 is the file as it is in
{\bf the common ancestor}\footnote{during a merge. You know how that {\bf meaning} changes depending on the operation}.
Stage number 2 is the file as it is in {\bf HEAD}. Stage number 3 is the file as it is in {\bf the other branch}. See for yourselves:

\begin{lstlisting}[style=console_style,
	basicstyle=\small,
	caption=Checking the ids of {\bf example.py} in different revisions]
$ git ls-tree $( git merge-base HEAD MERGE_HEAD ) example.py
100755 blob b0a7eae10629dec61246f86c08f2432f0e276675    example.py
$ git ls-tree HEAD example.py
100755 blob 500616dc70f4847f244d29d827a192b7fa03de93    example.py
$ git ls-tree MERGE_HEAD example.py
100755 blob ad957360597fb9bd3e83d4bfa869f6e19b7fbf2b    example.py
\end{lstlisting}

Stage number 0, as we had seen before, means that the file is either unchanged or added to {\bf index}.

Let's solve the conflict as we had done before (which means that both files have to be modified) and then let's see the
output of the git commands:

\begin{lstlisting}[style=console_style,
	basicstyle=\small,
	caption=Status after modifying the files]
$ git status
On branch example9/branchB
You have unmerged paths.
  (fix conflicts and run "git commit")
  (use "git merge --abort" to abort the merge)

Changes to be committed:
        modified:   module.py

Unmerged paths:
  (use "git add <file>..." to mark resolution)
        both modified:   example.py

Changes not staged for commit:
  (use "git add <file>..." to update what will be committed)
  (use "git restore <file>..." to discard changes in working directory)
        modified:   module.py

$ git ls-tree -r HEAD
100644 blob 2f78cf5b66514f2506d9af5f3dadf3dee7aa6d9f    .gitignore
100755 blob 500616dc70f4847f244d29d827a192b7fa03de93    example.py
100644 blob 2d1fe2ea52267ab6e75cca853c393fc6929a0e45    module.py
$ git ls-files -s
100644 2f78cf5b66514f2506d9af5f3dadf3dee7aa6d9f 0       .gitignore
100755 b0a7eae10629dec61246f86c08f2432f0e276675 1       example.py
100755 500616dc70f4847f244d29d827a192b7fa03de93 2       example.py
100755 ad957360597fb9bd3e83d4bfa869f6e19b7fbf2b 3       example.py
100644 475f980e6c6e24f8fc4a144e498fa1c1c59da370 0       module.py
\end{lstlisting}

Even though the ids of the objects in {\bf index} and {\bf HEAD} hasn't changed, the content of the files has changed on
{\bf the working tree}. Given that {\bf module.py} has an id that is different between {\bf index} and {\bf HEAD}, git lists
it in {\bf Changes to be committed}. Given that the content of the file on {\bf the working tree} is different from {\bf index},
it is also listed in {\bf Changes not staged for commit}. {\bf example.py} hasn't been added to index and so it is still conflicted
so we still have the same 3 versions of the file in index and is listed in {\bf Unmerged paths}.

Let's {\bf add module.py} first and see what happens:

\begin{lstlisting}[style=console_style,
	basicstyle=\small,
	caption=After adding {\bf module.py}]
$ git add module.py 
$ git status         
On branch example9/branchB
You have unmerged paths.
  (fix conflicts and run "git commit")
  (use "git merge --abort" to abort the merge)

Changes to be committed:
        modified:   module.py

Unmerged paths:
  (use "git add <file>..." to mark resolution)
        both modified:   example.py

$ git ls-tree -r HEAD
100644 blob 2f78cf5b66514f2506d9af5f3dadf3dee7aa6d9f    .gitignore
100755 blob 500616dc70f4847f244d29d827a192b7fa03de93    example.py
100644 blob 2d1fe2ea52267ab6e75cca853c393fc6929a0e45    module.py
$ git ls-files -s    
100644 2f78cf5b66514f2506d9af5f3dadf3dee7aa6d9f 0       .gitignore
100755 b0a7eae10629dec61246f86c08f2432f0e276675 1       example.py
100755 500616dc70f4847f244d29d827a192b7fa03de93 2       example.py
100755 ad957360597fb9bd3e83d4bfa869f6e19b7fbf2b 3       example.py
100644 42004985f8888627d3985174325a235401568e0b 0       module.py
\end{lstlisting}

For this file, it behaved like it did when there was no conflict. The ids for the object are different between {\bf HEAD} and
{\bf index} and that's why it is listed in {\bf Changes to be committed}. The file in {\bf the working tree} is like it is in
{\bf index} now and that's why git knows there's nothing else to report about the file. {\bf example.py} is still conflicted
and so there are no changes for it from what we had seen before.

Let's add {\bf example.py} now:

\begin{lstlisting}[style=console_style,
	basicstyle=\small,
	caption=After adding {\bf example.py}]
$ git add example.py 
$ git status
On branch example10/branchB
All conflicts fixed but you are still merging.
  (use "git commit" to conclude merge)

Changes to be committed:
        modified:   example.py
        modified:   module.py

$ git ls-tree -r HEAD
100644 blob 2f78cf5b66514f2506d9af5f3dadf3dee7aa6d9f    .gitignore
100755 blob 500616dc70f4847f244d29d827a192b7fa03de93    example.py
100644 blob 2d1fe2ea52267ab6e75cca853c393fc6929a0e45    module.py
$ git ls-files -s
100644 2f78cf5b66514f2506d9af5f3dadf3dee7aa6d9f 0       .gitignore
100755 ad957360597fb9bd3e83d4bfa869f6e19b7fbf2b 0       example.py
100644 42004985f8888627d3985174325a235401568e0b 0       module.py
\end{lstlisting}

Now that the (originally) conflicted file has been added to {\bf index}, we have a single item for it in {\bf index}. Given that
the ids are not the same between {\bf HEAD} and {\bf index}, git reports it in {\bf Changed to be committed} and given that the
content of the files in the working tree is the same as it is in {\bf index}, git doesn't say anything else about the files. Let's
create a new revision at this moment and see what happens:

\begin{lstlisting}[style=console_style,
	basicstyle=\small,
	caption=Committing after conflict]
$ git commit -m "A new revision"
[example9/branchB 9f80666] A new revision
$ git status
On branch example9/branchB
nothing to commit, working tree clean
$ git ls-tree -r HEAD
100644 blob 2f78cf5b66514f2506d9af5f3dadf3dee7aa6d9f    .gitignore
100755 blob ad957360597fb9bd3e83d4bfa869f6e19b7fbf2b    example.py
100644 blob 42004985f8888627d3985174325a235401568e0b    module.py
$ git ls-files -s
100644 2f78cf5b66514f2506d9af5f3dadf3dee7aa6d9f 0       .gitignore
100755 ad957360597fb9bd3e83d4bfa869f6e19b7fbf2b 0       example.py
100644 42004985f8888627d3985174325a235401568e0b 0       module.py
\end{lstlisting}

As it had happened before when there was no conflict, when we create a new revision, the ids in {\bf HEAD} are updated to the
same ids of the objects that {\bf index} had right before the revision was created. Given that now {\bf index} and {\bf HEAD} have
the same ids, there are no {\bf Changes to be committed} and given that there are no differences between {\bf index} and
{\bf working tree}, there's nothing else to report.

If you stop to think about it, other than having the 3 items listed in {\bf index} while the file was conflicted, there were no
other differences in terms of what was held in {\bf index} and the revisions. All in all, git does {\bf not} save any information
whatsoever related to the conflicts that happened on a merge when a new revision is created. It simply creates a new revision with
multiple parents and the files and their (final) contents on the revision just like it would do on any other revision:

\begin{lstlisting}[style=console_style,
	basicstyle=\small,
	caption=Checking revision information]
$ git cat-file -p HEAD
tree 396032b1546d75672f3a85c13a858d3b187d2046
parent 3cd9cfd24b13fa2381c5cc8009275b961ea7a26b
parent dfde76c316ff0a070ddc7560f86b7279b73ed807
author Developer A <dev.a@localhost> 1592110970 -0600
committer Developer A <dev.a@localhost> 1592111469 -0600

A new revision
$ git cat-file -p HEAD^{tree}
100644 blob 2f78cf5b66514f2506d9af5f3dadf3dee7aa6d9f    .gitignore
100755 blob ad957360597fb9bd3e83d4bfa869f6e19b7fbf2b    example.py
100644 blob 42004985f8888627d3985174325a235401568e0b    module.py
\end{lstlisting}

