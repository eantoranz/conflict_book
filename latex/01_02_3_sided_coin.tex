% copyright 2020 Edmundo Carmona Antoranz
% Released under the terms of Creative Commons Attribution-ShareAlike 4.0 International Public License

\section{Conflicts are a 3-sided coin}

Let's work on our first example from a real project. In this case, \hyperref[git_repo]{git itself}.

\subsection{Example 3 - a git conflict}
\label{example_03}

From \hyperref[git_repo]{git's repo}, checkout revision {\bf 80648bb3f2} and merge {\bf 20a5fd881a} \footnote{These
are the parents of revision {\bf 0df82d99da}}. You end up with a conflict in {\bf pack-bitmap.c}:
% FIXME might need to make it a little narrower
\begin{lstlisting}[style=c_style, firstnumber=671, caption={\bf example 3} - Conflict in {\bf pack-bitmap.c}]
	struct bitmap *objects = bitmap_git->result;

<<<<<<< HEAD
	ewah_iterator_init(&it, type_filter);
=======
	if (bitmap_git->reuse_objects == bitmap_git->pack->num_objects)
		return;

	init_type_iterator(&it, bitmap_git, object_type);
>>>>>>> 20a5fd881a

	for (i = 0; i < objects->word_alloc &&
\end{lstlisting}

I hear you this time. Let's keep {\it all the lines}, you say? Which block should come first? {\bf HEAD} or
{\bf the other branch}'s? Maybe the call to {\bf ewah\_iterator\_init()} should be placed between the conditional and
the call to {\bf init\_type\_iterator()} on {\bf the other branch}? Perhaps all the lines should be removed, actually?
Maybe we just need to keep only {\bf a part of it}? Let me show you some scenarios so that you can see what I'm talking about.

\subsubsection{Example 3 - Scenario 1}
For the sake of the argument, {\bf assume} that the file on {\bf the common ancestor} \footnote{Roughly speaking, the last revision
present on both branches that are being merged} is like this, at around those lines:\footnote{This is not the real contents
of {\bf the common ancestor}, by the way. I just want you to assume that this be the case}

\begin{lstlisting}[style=c_style, firstnumber=671, caption={\bf example 3} - Scenario 1 - common ancestor code]
	struct bitmap *objects = bitmap_git->result;

	ewah_iterator_init(&it, type_filter);
	if (bitmap_git->reuse_objects == bitmap_git->pack->num_objects)
		return;

	init_type_iterator(&it, bitmap_git, object_type);

	for (i = 0; i < objects->word_alloc &&
\end{lstlisting}

If this were the case, that would tip us that, {\it based on the conflict}, each one of the branches removed {\it one piece of 
the original code}. {\bf HEAD} removed the conditional from line 674 and the call to {\bf init\_type\_iterator()}. {\bf The
other branch} removed the call to {\bf ewah\_iterator\_init()}.

Then conflict resolution would be:
\begin{lstlisting}[style=c_style, firstnumber=671, caption={\bf example 3} - Scenario 1 - conflict resolution]
	struct bitmap *objects = bitmap_git->result;

	for (i = 0; i < objects->word_alloc &&
\end{lstlisting}

\subsubsection{Example 3 - Scenario 2}
Now assume that the common ancestor file looks like this:

\begin{lstlisting}[style=c_style, firstnumber=671, caption={\bf example 3} - Scenario 2 - common ancestor code]
	struct bitmap *objects = bitmap_git->result;

	ewah_iterator_init(&it, type_filter);
	if (bitmap_git->reuse_objects == bitmap_git->pack->num_objects)
		return;

	for (i = 0; i < objects->word_alloc &&
\end{lstlisting}

In this case, {\bf HEAD} removed the conditional started on line 674, {\bf the other branch} removed the call to
{\bf ewah\_iterator\_init()} and added a call to {\bf init\_type\_iterator()}. That points to having this as the conflict resolution:

\begin{lstlisting}[style=c_style, firstnumber=671, caption={\bf example 3} - Scenario 2 - conflict resolution]
	struct bitmap *objects = bitmap_git->result;

	init_type_iterator(&it, bitmap_git, object_type);

	for (i = 0; i < objects->word_alloc &&
\end{lstlisting}

\subsubsection{Example 3 - Scenario 3}
Finally, assume that the common ancestor file looks like this:

\begin{lstlisting}[style=c_style, firstnumber=671, caption={\bf example 3} - Scenario 3 - common ancestor code]
	struct bitmap *objects = bitmap_git->result;

	if (bitmap_git->reuse_objects == bitmap_git->pack->num_objects)
		return;

	for (i = 0; i < objects->word_alloc &&
\end{lstlisting}

In this scenario, {\bf HEAD} removed the conditional started on line 673, added a call to {\bf ewah\_iterator\_init()}.
{\bf The other branch} kept the conditional but added a later call to {\bf init\_type\_iterator()}. This begs for {\it yet another}
different resolution. We need to remove the conditional and keep the two new calls. Which one should come before? That's
a fairly good question to ask and seeing the code alone won't be able to tell us. You might need more background knowledge
to be able to solve it properly, or check the requirements that brought in each call\footnote{That can be time consuming, right?}
. Let's assume, for the sake of moving forward with the example, that we figure out that we should place the call
to {\bf init\_type\_iterator()} before the call to {\bf ewah\_iterator\_init()}. Then conflict resolution would be:

\begin{lstlisting}[style=c_style, firstnumber=671, caption={\bf example 3} - Scenario 3 - conflict resolution]
	struct bitmap *objects = bitmap_git->result;

	init_type_iterator(&it, bitmap_git, object_type);

	ewah_iterator_init(&it, type_filter);

	for (i = 0; i < objects->word_alloc &&
\end{lstlisting}

So 3 different scenarios for {\it the same conflict}. Each scenario meant a different conflict resolution. Now, what I want you
to realize is this {\bf hard cold truth}: {\bf the common ancestor} \footnote{along with the conflict itself, of course} is driving
conflict resolution. If you do {\it not} consider {\bf the common ancestor} you would be.... what would be the best word to describe
it? Oh, yes! {\bf Guessing!} And I don't care how educated your guess is (in terms of background knowledge of the code you are
working with). You would have to have a memory that beats that provided by git in terms of knowing what code looked like in the past
so that I stop calling it so.

Having cleared that up, let's continue with the current example. What does {\bf the common ancestor} look like in this case,
for real? Finding that out is a process that involves more than one step so let's start.

\subsubsection{Example 3 - Solving the conflict for real}
First things first. What is the common ancestor?

\begin{lstlisting}[style=console_style, caption={\bf example 3} - finging common ancestor]
$ git merge-base HEAD MERGE_HEAD
d0654dc308b0ba76dd8ed7bbb33c8d8f7aacd783
\end{lstlisting}

{\bf The common ancestor} is revision {\bf d0654dc308b0ba76dd8ed7bbb33c8d8f7aacd783}. On that revision, is the file called the same?
{\bf Let's hope so!} Did the content change much? Perhaps some lines were {\it added} or {\it deleted} before the block we are working
with? Let's hope it didn't change that much! \footnote {I don't know about you, but I'm not used to crossing fingers
when I have to do development work... at least, not this often.} Let's give it a try:

\begin{lstlisting}[style=console_style, basicstyle=\small, caption={\bf example 3} - checking ancestor code]
$ git blame -s -L 671,681 d0654dc30 -- pack-bitmap.c
fff42755efc 671)        while (roots) {
fff42755efc 672)                struct object *object = roots->item;
fff42755efc 673)                roots = roots->next;
fff42755efc 674) 
3ae5fa0768f 675)                if (find_pack_entry_one(object->oid.hash, bitmap_git->pack) > 0)
fff42755efc 676)                        return 1;
fff42755efc 677)        }
fff42755efc 678) 
fff42755efc 679)        return 0;
fff42755efc 680) }
fff42755efc 681)
\end{lstlisting}

{\bf Oops!} We ran out of luck! \footnote{Feeling lucky? Don't sit down to solve conflicts, please. Las Vegas sounds like a more
suitable place.}  The file hadn't been renamed ({\bf great!}) its {\bf previous} content did change ({\bf bummer!}) and so the
section we need to look at is not at the same position of the conflict we are dealing with. After surfing the file a little bit,
we find the block we care about, some 40 lines before where we were hoping it would be:

\begin{lstlisting}[style=console_style, basicstyle=\small, caption={\bf example 3} - checking ancestor code\, 2nd attempt]
$ git blame -s -L 631,638 d0654dc308b0ba76dd8ed7bbb33c8d8f7aacd783 -- pack-bitmap.c
3ae5fa0768f 631)        struct bitmap *objects = bitmap_git->result;
3ae5fa0768f 632) 
3ae5fa0768f 633)        if (bitmap_git->reuse_objects == bitmap_git->pack->num_objects)
fff42755efc 634)                return;
fff42755efc 635) 
fff42755efc 636)        ewah_iterator_init(&it, type_filter);
fff42755efc 637) 
fff42755efc 638)        while (i < objects->word_alloc && ewah_iterator_next(&filter, &it)) {
\end{lstlisting}

And {\it now} we could compare what each branch did. On {\bf HEAD}, we {\it removed} the conditional started on
line 633.\footnote{Line 633 of the file in {\bf the common ancestor}}. On {\bf the other branch}, we {\it removed} the call
to {\bf ewah\_iterator\_init()} and we added a call to {\bf init\_type\_iterator()}. Which means that in our conflict resolution
we would keep the call to {\bf init\_type\_iterator()} only because the other parts that were present on {\bf the common ancestor}
are gone:

\begin{lstlisting}[style=c_style, firstnumber=671, caption={\bf example 3} - final resolution]
	struct bitmap *objects = bitmap_git->result;

	init_type_iterator(&it, bitmap_git, object_type);

	for (i = 0; i < objects->word_alloc &&
\end{lstlisting}

And by comparing this with {\bf 0df82d99da} you should find no meaningful differences. \footnote{A formatting difference? That's fine! Not publishing this work anywhere, right?}

I hope this time you really got why it's so important to be able to see {\bf the common ancestor} on a conflict. As I said before, if you
don't consider the common ancestor, you are making an {\it educated guess} at best, a {\bf total disaster} at worst. I agree it
is {\it a lot of work} if we need to go check what {\bf the common ancestor} code looks like every single time we have a conflict.
\begin{itemize}
	\item Finding {\bf the common ancestor}
	\item Hope the file was not renamed
	\item Hope we are lucky so the file hasn't changed that much so we can find the original spot fast
\end{itemize}

What if the file had been renamed? Can we figure out what the original name was easily?

Before you burn the book in frustration, let me show you a little git {\it trick}. git can actually do the work of showing you what
{\bf the common ancestor} looks like {\it without } additional work on your part. By setting {\bf merge.conflictStyle} to
{\bf diff3} \footnote{ Run {\bf git config merge.conflictStyle diff3}. Use {\bf --global} if you want it to apply globally on
your user's config, not on a per-repo basis}, git will kindly add what the common ancestor looked on the section of code that is
related to the conflict. Let me show you the current conflict after applying this option for your own amusement:

\subsubsection{Example 3 - Solving the conflict for real with {\bf diff3}}
\begin{lstlisting}[style=c_style, firstnumber=671, caption={\bf example 3} - conflict with {\bf diff3} applied]
	struct bitmap *objects = bitmap_git->result;

<<<<<<< HEAD
	ewah_iterator_init(&it, type_filter);
||||||| d0654dc308
	if (bitmap_git->reuse_objects == bitmap_git->pack->num_objects)
		return;

	ewah_iterator_init(&it, type_filter);
=======
	if (bitmap_git->reuse_objects == bitmap_git->pack->num_objects)
		return;

	init_type_iterator(&it, bitmap_git, object_type);
>>>>>>> 20a5fd881a

	for (i = 0; i < objects->word_alloc &&
\end{lstlisting}

{\bf Wow!} We get to see {\bf the common ancestor} code between the sections of each branch. git is also kind enough to tell us
{\bf the common ancestor} revision. Now we can see what each one of the branches did.

In order to solve our conflict, first, start working from {\bf HEAD}:
\begin{lstlisting}[style=c_style, firstnumber=673, caption={\bf example 3} - step 1]
<<<<<<< HEAD
	ewah_iterator_init(&it, type_filter);
||||||| d0654dc308
\end{lstlisting}

Then consider how {\bf the other branch} changed {\it from} {\bf the common ancestor}:

\begin{lstlisting}[style=c_style, firstnumber=675, caption={\bf example 3} - step 2]
||||||| d0654dc308
	if (bitmap_git->reuse_objects == bitmap_git->pack->num_objects)
		return;

	ewah_iterator_init(&it, type_filter);
=======
	if (bitmap_git->reuse_objects == bitmap_git->pack->num_objects)
		return;

	init_type_iterator(&it, bitmap_git, object_type);
>>>>>>> 20a5fd881a
\end{lstlisting}

The call to {\bf ewah\_iterator\_init()} was removed and a call to {\bf init\_type\_iterator()} was
added. So we replicate that on {\bf HEAD}:

\begin{lstlisting}[style=c_style, firstnumber=673, caption={\bf example 3} - step 3]
<<<<<<< HEAD
	init_type_iterator(&it, bitmap_git, object_type);
||||||| d0654dc308
\end{lstlisting}

And we can now remove the remaining conflict block pieces and markers:

\begin{lstlisting}[style=c_style, firstnumber=671, caption={\bf example 3} - Done!]
	struct bitmap *objects = bitmap_git->result;

	init_type_iterator(&it, bitmap_git, object_type);

	for (i = 0; i < objects->word_alloc &&
\end{lstlisting}

See? {\bf No hassle!}

But come on! It is {\it possible} to survive solving conflicts without seeing {\bf the common ancestor} block, right?
But {\it of course!} Just like it's possible to develop software without using any unit tests whatsoever,
{\it if you know what I mean}. It's not like we need it like oxygen or water. But I already told you the what it is
like when you decide not to look at {\bf the common ancestor}: It's a {\bf guess}. Using this option to see
{\bf the common ancestor} code on a conflict is {\bf the single most important tip} in this handbook.

\subsection{Tips}
\begin{itemize}
	\item {\bf Always} consider code of {\bf the common ancestor}.
	\item Set {\bf merge.conflicStyle} to {\bf diff3} to get it shown directly in a conflicting section of code.
\end{itemize}

\subsection{Exercises}
\subsubsection{Exercise 3}
From \hyperref[git_repo]{git's repo}, checkout revision {\bf fe870600fe} and merge {\bf 1bdca81641}
\footnote{These are the parents of revision {\bf f8cb64e3d4}}. Solve both conflicts (there are 2 conflicting files,
one conflict section on each one of the files). Solution is \hyperref[example_03]{here}.

